\documentclass[12pt,a4paper]{article}
\usepackage{amsmath}
\usepackage[margin=0.5in]{geometry}
\usepackage{setspace} % For line spacing
\onehalfspacing % Apply 1.5 line spacing
\usepackage{fancyhdr} % For custom headers/footers
\pagestyle{fancy} % Enable fancy pagestyle
\fancyhf{} % Clear default headers/footers
\fancyhead[L]{} % Clear left header
\fancyhead[C]{Leyes de Conservacion - Fisica 1 - Ay1 Unica} % Center header
\fancyhead[R]{} % Clear right header
% \renewcommand{\headrulewidth}{0pt} % Remove header rule (line)
\begin{document}

% \title{Leyes de Conservacion - Fisica 1 - Ay1 Unica}
% \author{Ali Martin Z.A.}


En este documento se desarrolla un ejercicio de Leyes de Conservacion enmarcado en la materia de Fisica 1 para alumnos de las carreras Cs. Fisicas, Cs. Quimicas, Cs. de la Atmosfera y Oceanografia. Se espera abordar y afianzar los conceptos de \textit{momento lineal, momento angular, energía: sus conservaciones y las implicancias}, asi tambien como \textit{centro de masa, su posicion y velocidad}, y \textit{fuerzas internas y externas}. El problema elegido demuestra la potencia de este metodo, ya que la dinamica es complicada, pero aun asi podemos decir muchas cosas del sistema. Se busca tambien estudiar algunos casos limites del problema, y casos particulares interesantes; analisis considerado importante para la materia.  Este ejercicio esta pensado para la segunda clase practica del tema "Leyes de conservacion", y por lo tanto se espera que los alumnos ya hayan visto en las clases teoricas las leyes de conservacion del momento lineal, momento angular, y energia mecanica, y tambien hayan tenido una clase practica donde se abordan los primeros problemas relacionados a momento lineal.
Este es un problema introduce por primera vez la conservacion de momento angular (en el marco de las clases practicas); un problema estandar de la materia, pero con mucha riqueza conceptual, y es clave para el desarrollo madurativo de los conceptos mas importantes. Se debe demostrar que no es necesario resolver el problema en su totalidad para poder entender cualitativamente como es la dinamica, y mas aun, en este problema no es necesario utilizar las ecs. de Newton para encontrar la ecuacion de movimiento. Complementariamente, se presenta un codigo donde se muestra la evolucion del problema, donde se puede ver tambien como cambia la dinamica al considerar condiciones iniciales mas complicadas. 
La clase consta de una resolucion en el pizarron, donde se buscara el interes e interaccion de los alumnos haciendo preguntas durante la explicacion. Al finalizar, se mostrara la simulacion. El cronograma esta pensado para realizarse en 1 hora, de la cual se proponen 5 minutos para presentar el problema y los objetivos claramente, seguidos de 45 minutos de resolucion, y finalmente 10 minutos para mostrar la animacion y dar lugar a preguntas. Se plantea esta duracion para que haya tiempo de resolver un segundo problema, y para dar tiempo de consultas al finalizar con las resoluciones.


\textbf{Enunciado y Explicacion}: El problema elegido consta de 2 masas $m_1$ y $m_2$ unidas por un resorte ($k$,$\ell_0$). Las masas estan sobre una mesa sin rozamiento. Inicialmente las masas estan separadas en una longitud $d=\ell_0$, la masa $m_1$ tiene una velocidad $v_1$ y simultaneamente se le imprime una velocidad $v_2$ a la masa 2 tal que el centro de masa del sistema tiene una velocidad nula. El esquema del problema se presenta en la Figura \ref{fig:enunciado}. Dejar bien en claro los datos y los objetivo del problema es muy importante para los alumnos de Fisica 1. Asi que se tomara especial cuidado en remarcar primero que es lo que buscamos, y cuales son los datos del problema. Los objetivos del problema son, en primera instancia, encontrar la velocidad $v_2$ que cumpla con la condicion inicial y la posicion del centro de masa. El siguiente paso sera ver cuales cantidades son conservadas en el sistema, y que implicaciones tienen sobre la dinamica. Una vez considerado esto, buscaremos las magnitudes conservadas en funcion de los datos del problema. Teniendo en cuenta que la dinamica del problema es complicada, como siguiente objetivo se obtienen ecuaciones para el momento angular $L_{CM}(t)$ y la energia mecanica $E_m(t)$ en funcion de la posicion $r_1(t)$, para obtener la velocidad angular $\omega$ y una ecuacion diferencial para $\frac{dr_1}{dt}$ en funcion de $r_1(t)$ y los datos del problema. Al final nos encontraremos con una expresion interesante que no sabemos resolver, pero podemos simular numericamente.
Antes de seguir adelante, se preguntara a los alumnos cual es el movimiento \textit{cualitativo} que esperan del problema, y se discutiria brevemente de forma grupal que es entonces lo que podemos esperar.

De vital importancia para esta materia es decir explicitamente cual es el sistema de referencia que se va a utilizar. Para eso se preguntara a los alumnos \textit{cuales son las opciones que tenemos}, en este caso se resaltara que por lo discutido anteriormente nos conviene usar coordenadas polares. Pero para hacerlo de manera mas formal, resolveremos la primera parte desde un sistema de referencia x-y (fijo) en un lugar arbitrario. Y buscaremos ponernos en el sistema centro de masa. Esto normalmente no es necesario pero me parece instructivo, y sirve para afianzar conceptos de la clase teorica en un ejemplo concreto. Primero vemos que para que la $\vec{v}_{CM}=\dfrac{\sum_i m_i\vec{v}_i}{\sum_i m_i}=0$, necesitamos que $\vec{v}_2=-\frac{m_1}{m_2}v_1\hat{x}$. Y visto desde un punto fijo arbitrario el $\vec{r}_{CM}=\dfrac{\sum_i m_i\vec{r}_i}{\sum_i m_i}=\vec{r}_2+\dfrac{\ell_0m_1}{m_1+m_2}\hat{y}$, donde usamos que $\vec{r_1}=\vec{r_2}+\ell_0\hat{y}$. Entonces preguntamos \textit{que significa estar en el sistema centro de masa?}. Esto significa que $\vec{r}_{CM}=0$. Entonces, si nos paramos en el CM, vemos que conseguimos $\vec{r}_2=-\ell_0\frac{m_1}{m_1+m_2}\hat{y}$. Dejamos en claro que a partir de ahora vamos a trabajar en el sistema CM. Para ver porque nos gustaria trabajar desde este punto, vamos a ver que magnitudes se conservan. Comenzando por el momento lineal $\frac{d\vec{p}}{dt}=\sum F^{ext}= (-m_1g+N_1-m_2g+N_2)\hat{z}=0$ por vinculos. Entonces, se conserva el momento lineal, y por lo tanto $\vec{v}_{CM}(t)=0 \implies \vec{r}_{CM}(t)=\text{cte}=0$. Vemos entonces la ventaja de trabajar sobre el centro de masa: como se conserva el momento lineal y en $t=0$ este esta quieto, permanece en reposo $\forall t$. La conservacion del momento angular esta dada por la suma de torques externos $\dfrac{d L_{CM}}{dt}=\sum M^{ext}=\sum r_i\times F_i^{ext}=0$ ya que los pesos y las normales se cancelan de a pares. Esto implica que el momento angular se conserva. Un comentario interesante es que en este ejercicio en particular, el momento angular se conserva desde cualquier sistema de referencia, ya que las fuerzas se cancelan de a pares por una condicion de vinculo, que es independiente del sistema de referencias. Recordemos que esto no siempre es asi y hay que tener cuidado. Tambien resaltaria que estamos considerando el resorte como parte de nuestro sistema, y por lo tanto no cuenta como una fuerza externa. Preguntaria \textit{que ocurre si no consideramos el resorte como parte del sistema?}. Se discute grupalmente que es lo que pasaria en esta situacion, notando que igual se conservarian los momentos lineales y angulares. Entonces cual es la diferencia? Para eso miramos la conservacion de la energia. $\dfrac{dE_m}{dt}=\sum W^{NC}=0$ ya que no hay rozamiento y la fuerza elastica es conservativa. Pero si no tomamos al resorte como parte del sistema, lo que sucede es que la energia potencial elastica le saca energia a las particulas, y luego se las devuelve. Entonces se podria decir que en ese caso el resorte actua como una fuerza no conservativa. Si hacemos la integral de trabajo de esta fuerza, veremos su caracter oscilatorio. Si bien es una fuerza no conservativa, le termina devolviendo la energia al sistema. 

Volvemos al ejercicio. Nos piden las expresiones para el momento angular y la energia mecanica en funcion de los datos. Las cuentas son directas y obtenemos $L_{CM}(0)=-\ell_0v_1m_1\hat{z}$ y 
$E_m(0)=\frac{1}{2}m_1v_1(1+\frac{m_1}{m_2})+\frac{1}{2}k\ell_0^2$. Notablemente vemos que $L_{CM}$ no depende de $m_2$. Para explicar esto, damos un paso para atras en las cuentas que se hacen en el pizzaron, y vemos que esto pasa porque la condicion inicial es tal que $m_1|v_1|=m_2|v_2|$ y entonces podemos reescribir $L_{CM}=-m_2\underbrace{\frac{m_1}{m_2}v_1}_{v_2}\underbrace{\frac{\ell_0m_1}{m_1+m_2}\hat{z}}_{r_2}-m_1\underbrace{\frac{m_2\ell_0}{m_1+m_2}}_{r_1}v_1\hat{z}=-m_2v_2r_2\hat{z}-m_1v_1r_1\hat{z}$ que es justo lo que esperamos. Notamos tambien que segun la convension utilizada y la condicion inicial el signo es el correcto.

Nuestro siguiente gran objetivo es encontrar expresiones para $L_{CM}(t)$ y $\omega(t)$ en funcion del tiempo, y tambien de la posicion $r_1(t)$ en ese instante. Antes de esto, lo que nos pide el ejercicio es lo siguiente: si para un instante $t$ las velocidades de las masas $m_1$ y $m_2$ son $\vec{v}_1(t)$ y $\vec{v}_2(t)$, y las posiciones $\vec{r}_1(t)$ y $\vec{r}_2(t)$ respectivamente, de expresiones para $\vec{v}_2(t)$ en funcion de $\vec{v}_1(t)$ y para $\vec{r}_2(t)$ en funcion de $\vec{r}_1(t)$. Para un instante arbitrario pasamos a polares y la conservacion del momento lineal nos permite decir que \textbf{vectorialmente}
$\vec{r}_{CM}(t)=\dfrac{m_1\vec{r}_1^{CM}(t)+m_2\vec{r}_2^{CM}(t)}{m_1+m_2}=0$ y obtenemos que $\vec{r}_2(t)=-\frac{m_1}{m_2}\vec{r}_1(t)$ y analogamente $\vec{v}_2(t)=-\frac{m_1}{m_2}\vec{v}_1(t)$. Antes de pasar a buscar $L_{CM}(t)$ vamos a dar un paso para adelante y vamos a descomponer $\vec{v}(t)=v_\theta(t) \hat{\theta}+v_r(t) \hat{r}$. Calculando obtenemos 
$L_{CM}(t)=m_1(1+\frac{m_1}{m_2})r_1(t)v_\theta(t)\hat{z}$. Para obtener $\omega(t)$ usamos que $v_\theta=r\cdot\omega$ (que aca vale por ser un movimiento en un plano), y usando la conservacion de $L_{CM}$, podemos escribir 
$\omega(t)=-\dfrac{m_2}{m_1+m_2}\dfrac{v_1(0)\ell_0}{r_1^2(t)}\hat{z}$ en funcion de $r_1(t)$ y los datos. Notemos que la frecuencia depende del tiempo, se preguntaria a los alumnos \textit{porque la frecuencia no es constante?} Intuitivamente, se puede ver que las masas oscilan en la direccion radial, consecuentemente, si miramos el momento angular $L \propto r^2\omega $ si variamos $r$ tenemos que variar $omega$ para que $L$ se conserve. Finalmente nos piden encontrar la energia $E_m(t)$ en funcion de datos y de $\frac{d r_1(t)}{dt}$ y $r_1(t)$. Que ecuacion diferencial obtenemos para $r_1(t)$? La expresion general para la energia mecanica es 
$E_m(t)=\frac{1}{2}m_1(\frac{d}{dt} \vec{r}_1(t))^2+\frac{1}{2}m_2(\frac{d}{dt} \vec{r}_2(t))^2+\frac{1}{2}k(\vec{r_1}(t)+\vec{r_2}(t))^2$. Reemplazando todo lo que hicimos antes podemos llegar a que 
$E_m(t)=\frac{1}{2}m_1(\dot r_1^2(t)+r_1^2(t)\omega^2(t))+\frac{1}{2}\frac{m_1^2}{m_2}(\dot r_1^2(t)+r_1^2(t)\omega^2(t))+\frac{1}{2}kr_1^2(t)(1+\frac{m_1}{m_2})^2$. Como la energia se conserva, podemos decir que $E_m(t)=E_0$ que conocemos en funcion de los datos del problema. Entonces, con esto tenemos nuestra ecuacion diferencial, que despejando $\dot r_1$ tenemos
\begin{equation}\label{eq:dif}
\frac{d}{dt}r_1(t)=\pm \sqrt{\frac{E_0}{\frac{m_1}{2}(1+\frac{m_1}{m_2})}-\left(\frac{m_2}{m_1+m_2}\right)^2\frac{v_1(0)^2\ell_0^2}{r_1^2(t)}-\frac{k}{m_1}r_1^2(t)\left(1+\frac{m_1}{m_2}\right)}
\end{equation}
Hasta aca llega el ejercicio. Vamos a interpretar un poco lo que tenemos para entender como podemos seguir si quisieramos. Tenemos una ecuacion diferencial para $r_1(t)$. Supongamos que la sabemos resolver, entonces que podemos hacer con esto? Solo es la dinamica radial de la masa 1. Pero que tiene que ver esto con el giro? No solo oscila en $\hat{r}$ sino que tambien gira, y ademas hay 2 masas. Lo bueno es que sabemos que giran a la misma velocidad$^{[1]}$ $\theta_1=\theta_2+\pi\implies \omega(t)=\omega_1=\omega_2$ $^{[2]}$. Por otro lado tambien sabemos que $r_2(t)=-\frac{m_1}{m_2}r_1(t)$. Asi que si resolvemos la dinamica de $r_1(t)$ y $\omega(t)$ podemos deducir gracias a las conservaciones como se mueve $r_2(t)$. Y tambien tenemos que:
\begin{equation}\label{eq:omega}
\omega(t)=-\frac{m_2}{m_1+m_2}\frac{v_1(0)\ell_0}{r_1^2(t)}\hat{z}
\end{equation}
Entonces esta ecuacion responde nuestra pregunta. Si conocemos $r_1(t)$ entonces podemos reemplazar en \ref{eq:omega} y obtenemos la velocidad angular. Ahora tendriamos que integrar para recuperar $\theta(t)$. La integral que tenemos que resolver se resuelve en mecanica clasica, asi que en el codigo ya esta puesta la solucion y vamos a mirar como es la dinamica. 

Para complementar la clase recomiendo los capitulos xxx del libro xxx, particularmente paginas xxx, y tambien del libro yyy paginas yyy. Si quieren ver como resolver la integral pueden ver el libro de Goldstein Classical Mechanics paginas ggg.


\end{document}