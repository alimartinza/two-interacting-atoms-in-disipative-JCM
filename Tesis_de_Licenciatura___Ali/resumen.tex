\newenvironment{abstract}%
{\thispagestyle{empty} \cleardoublepage\null \thispagestyle{empty} \vfill\begin{center}
\bfseries \abstractname \end{center} }%
{\thispagestyle{empty} \vfill\null }


\selectlanguage{spanish}
\begin{abstract}
El modelo de Janyes-Cummings de un \'atomo es un ejemplo paradigm\'atico para el estudio de la interaccion entre la radiacion y la materia que sirve para estudiar fenómenos de  informaci\'on cu\'antica. El modelo describe de 
manera sencilla la interacci\'on entre fotones y materia de manera puramente cu\'antica, y es utilizado en diversas aplicaciones. Desde el punto de vista experimental, permitió, entre otras cosas, realizar teleportación cuántica utilizando cavidades semiconductoras. Recientemente, con el creciente interés en las tecnologías cuánticas y la introducción de los transmones como arquitectura de cabecera para su desarrollo e innovación, este modelo tuvo renovadas aplicaciones, ya que la dinámica de los transmones es casi igual a la de un átomo en una cavidad descrita por el modelo de Jaynes-Cummings. Esto hace que estudiar estos modelos de electrodinámica de cavidades tenga un interés y una aplicación directa en la computación cuántica. 
La principal ventaja que trae la electrodinámica de circuitos superconductores sobre las demás arquitecturas  propuestas, radica en su escalabilidad, en los rangos de acoplamientos que se pueden acceder y en su capacidad de reducir errores mediante códigos de corrección, que logra mitigar hasta cierto punto el ruido o las pérdidas introducidos por el entorno. Este último punto es en el que se concentra la tesis.
Es interesante encontrar alguna cantidad u objeto que no sea sensible a los efectos del entorno; un "sensor cuántico" que de información del sistema. Un posible candidato es la fase geométrica. Esta es una fase que acumulan los estados cuánticos al evolucionar, relacionada con la trayectoria que recorre en el espacio de estados. 

En el presente trabajo extendemos el modelo de Jaynes-Cummings a dos qubits interactuantes, inmersos en una cavidad no lineal (con medio Kerr). El estudio se concentra en entender la dinámica del sistema, analizar la dinámica de entrelazamiento entre ambos átomos y observar su comportamiento ante los efectos de un entorno a temperatura cero. Finalmente, se estudia la fase geométrica y su sensibilidad frente al entorno. 

Se encontraron ciertas relaciones entre los parámetros el problema que predicen zonas de interés. Principalmente, sirven para predecir aproximadamente que combinación de parámetros se debe utilizar para lograr que la fase geométrica sea robusta ante los efectos destructivos del entorno. Este resultado, es  muy útil para diagramar implementaciones del modelo utilizando circuitos superconductores. 


\end{abstract}

%\selectlanguage{english}
%\begin{abstract}



%\end{abstract}
\selectlanguage{spanish}



