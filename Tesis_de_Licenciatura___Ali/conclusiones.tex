\chapter{Conclusiones}
\label{ch6:conclusiones}


%CAMBIAR ESTO PARA PERSONALIZARLO A MI GUSTO
\pagestyle{fancy}
\fancyhf{}
\fancyhead[LE]{\nouppercase{\rightmark\hfill}}
\fancyhead[RO]{\nouppercase{\leftmark\hfill}}
\fancyfoot[LE,RO]{\hfill\thepage\hfill}

En esta sección resumiremos los contenidos y principales resultados de la Tesis. En primer lugar, y después de la Introducción general al trabajo, 
en el capítulo \ref{ch:fg}  presentamos la fase geométrica (FG), un objeto ampliamente estudiado en diferentes áreas de la física actual. Comenzamos con la fase de Berry para sistemas bajo evoluciones cíclicas y adiabáticas, y posteriormente, se generalizó la definición de la FG al caso de sistemas abiertos bajo evoluciones arbitrarias (no-cíclicas, no-adibáticas). La FG tiene muchas propiedades importantes, y en particular es de interés en este trabajo, estudiar las condiciones para su robustez frente al efecto del entorno. Estas condiciones fueron utilizadas en los siguientes capítulos, en diferentes implementaciones.

Luego, en la sección \ref{ch3_jcm}, mostramos los principales aspectos del modelo de Jaynes-Cummings (un átomo dentro de una cavidad con un sólo modo del campo electromagnético). Analizamos la dinámica del modelo, donde el Hamiltoniano se diagonaliza en sub-espacios de 2x2, pudiéndose estudiar el problema de manera exacta. La baja dimensionalidad del problema permite representar las trayectorias (estados) en el espacio de fases en la esfera de Bloch. Esto nos permitió interpretar la FG. Si el medio de la cavidad es lineal, entonces la FG es robusta en el caso resonante ($\Delta=0$), trazando trayectorias máximas sobre la esfera de Bloch \cite{Viotti2022}. Extendimos el modelo incluyendo un medio no lineal, lo cual nos permitió demostrar que el detunning $\Delta$ y el parámetro del medio $\chi$ están relacionados. En este caso, el medio y el detunning están en igualdad de condiciones, y modificar la frecuencia de la cavidad permite eliminar completamente el efecto de las no linealidades del medio. Encontramos la condición de robustez que incluye a ambos parámetros. Este resultado es interesante ya que tiene implicancias experimentales importantes.

Luego, en el capítulo \ref{ch4_dinamica} estudiamos la dinámica de poblaciones generada por dos átomos acoplados dentro de la cavidad de JC. Para este análisis se introdujo un apantallamiento "no físico", para re-obtener los resultados del capítulo anterior al trazar parcialmente sobre el átomo apantallado. Los resultados sivieron para ver que el modelo estaba bien extendido y que no aparecían inconsistencias. Al liberar el apantallamiento, se logró capturar la complejidad del problema que era el objeto de estudio. En particular se estudió la dinámica de entrelazamiento entre los dos átomos y la dependencia con los parámetros del problema. Pudimos ver que al agregar un tercer estado dinámicamente relevante, sumado a las no linealidades de la cavidad, la estructura energética se hace mucho más compleja. Consecuentemente, el comportamiento del entrelazamiento en función de los parámetros no es sencilla, y demostramos que hay ciertas condiciones (\ref{ec4:condicion 1})- (\ref{ec4:condicion 3}) que predicen comportamientos particulares e interesantes. Estas condiciones se dan cuando las energías de los estados de la base están degeneradas. Por un lado, predicen zonas donde la concurrencia presenta oscilaciones coherentes y de mayor amplitud, en general acompañadas de SDE. Por otro, predicen zonas donde el entrelazamiento decae lentamente en el tiempo, es decir, un caso robusto ante los efectos del entorno.  

Finalmente, en el capítulo 5, se estudió la FG en el modelo completo de dos átomos en una cavidad disipativa no-lineal. Primero se mostró la dependencia de la FG con los parámetros del problema, comparando las condiciones iniciales en el subespacio de $N=1$ y $N=2$ excitaciones, donde encontramos  nuevamente que el espacio de $N=2$ presenta estructuras muy complejas. Luego, estudiando las correcciones a la FG inducidas por el entorno: $\delta\phi=\phi_d-\phi_u$, se corroboró que las condiciones de degeneración, no solo conservan el entrelazamiento, sino que también conservan la fase geométrica. Es decir, para estos puntos, se demostró que la corrección a la FG es nula: $\delta\phi=0$. Por lo tanto, se encontraron situaciones donde la FG es robusta frente a la disipación y decoherencia. 

Es importante destacar que las condiciones de robustez sufren una modificación por el efecto del entorno. Los puntos exactos en el espacio de parámetros donde el sistema presenta robustez, dependen suavemente del acoplamiento con el entorno. Hicimos este análisis al final del capítulo. 

Si bien las condiciones halladas existen y son, en principio útiles para las implementaciones experimentales del modelo, aún no pudimos encontrar la razón cuantitativa fundamental del por qué estas condiciones funcionan. Podemos especular acerca del origen topológico asociado a las FGs, pero  claramente entender estos aspectos requiere de un estudio profundo en el futuro próximo. 
%Un comportamiento interesante, es que para todos los estados iniciales estudiados, la condicion $\Delta-3\chi+2(k-J)=0$

La riqueza del modelo presentado en el capítulo 4 abre muchas posibilidades de estudios posteriores. Por ejemplo, ¿qué pasa si el estado inicial es superposición de estados de la base? ¿Cuál es la razón fundamental que lleva a la robustez de la FG? 

%\textcolor{blue}{completar/sacar: En el caso de 1 átomo con medio lineal se vio que la condición de robustez significa que las curvas sobre la esfera de Bloch son máximas, ya que recorren el meridiano más grande. Cuando el medio no es lineal esto ya no es así.} \textcolor{orange}{hacer grafico de bloch para la evolucion robusta de 1 átomo en medio kerr en el capítulo 3}

Vimos que los autoestados son máximamente entrelazados en el caso de un átomo en la cavidad y en  el subespacio de $N=1$ para dos átomos, cuando se cumple la condición de robustez $\Delta-\chi+2(k-J)=0$. En el caso de un átomo este era un estado de Bell $\ket{u_{1,2}}=\frac{1}{\sqrt{2}}(\ket{e0}\pm\ket{g1)}$, mientras que  en el caso de dos átomos para el subespacio de $N=1$, los autoestados son también entrelazados $\ket{u_{1,2}}=\frac{1}{\sqrt{2}}(\ket{\phi_1^{(1)}}\mp\ket{\phi_2^{(1)}})$.

Es importante determinar si el entrelazamiento es una condición  para la robustez de la FG. Un indicio de que están relacionadas se estableció en los capítulos 4 y 5: concretamente el entrelazamiento entre los átomos y la FG son especialmente robustos ante el efecto del entorno cuando se cumplían las \textbf{mismas condiciones}. Las condiciones (\ref{ec4:condicion 1})-(\ref{ec4:condicion 3}) predicen correctamente dónde se encontrarán los puntos donde el entorno no tiene efecto sobre estas cantidades.


%Esto y más, surge al analizar en profundidad el problema. Pero si se remite a los hechos, entonces lo que se encontró en este trabajo es que las condiciones \ref{ec4:condicion 1}, \ref{ec4:condicion 2} y \ref{ec4:condicion 3} predicen correctamente en donde se encontrarán los puntos donde el entorno no tiene efecto sobre estos observables.

Como perspectivas para el futuro próximo, esta Tesis ha dejado varios resultados de interés para analizar y profundizar. En primer lugar, podemos mencionar que sería importante buscar una razón analítica fundamental por la cual las condiciones de robustez funcionan. El problema que se presenta es que  las expresiones son muy complicadas de interpretar. Quizás se puede recurrir a una  combinación entre análisis analíticos y numéricos para encontrar dicha razón fundamental. 

En segundo lugar, es importante profundizar sobre el entendimiento del entrelazamiento en sistemas tripartitos. Utilizar y comparar diferentes definiciones en este sistema, puede ayudar a comprender si el entrelazamiento total del sistema contribuye a la robustez. Como hemos visto, la concurrencia, que es parte del entrelazamiento total, parece estar relacionada. Por lo tanto, no se puede descartar que el entrelazamiento total sea la causa fundamental de la robustez. Estudiar el entrelazamiento en sistemas tripartitos y su relación con la condición de robustez para la FG es muy interesante e importante para el diseño de implementaciones de este sistema en el laboratorio (por ejemplo, usando arquitecturas de circuitos superconductores). Otra potencial aplicación de la FG con relación al entrelazamiento se ve en la detección de la fidelidad en compuertas cuánticas \cite{Ganesh2025}.

Finalmente, y en relación a posibles implementaciones utilizando circuitos superconductores, en este modelo se puede estudiar el régimen de acoplamiento ultra-fuerte. Este régimen es de relevancia para el control de transmones en implementaciones de información y/o  computación cuántica. El circuito superconductor análogo al de este trabajo, consta de dos transmones acoplados a una guía de ondas coplanar, donde los transmones actúan como átomos artificiales, y la guía de ondas como cavidad resonante. Este problema es el bloque fundamental de la electrodinámica cuántica de circuitos, donde la cavidad puede estar también sometida a un forzado clásico externo (campos de control).

