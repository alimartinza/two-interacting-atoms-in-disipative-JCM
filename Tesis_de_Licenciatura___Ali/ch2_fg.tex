\chapter{Fase Geométrica}
\label{ch:fg}

%CAMBIAR ESTO PARA PERSONALIZARLO A MI GUSTO
\pagestyle{fancy}
\fancyhf{}
\fancyhead[LE]{\nouppercase{\rightmark\hfill}}
\fancyhead[RO]{\nouppercase{\leftmark\hfill}}
\fancyfoot[LE,RO]{\hfill\thepage\hfill}

Este cap\'itulo presenta uno de los objetos de estudio del trabajo. La fase geométrica es un objeto relevante en el ámbito de la informaci\'on cu\'antica, ya que como se ver\'a m\'as adelante, recupera informaci\'on sobre la trayectoria del sistema en el espacio de Hilbert. Su potencial recae en el hecho de que se han encontrado situaciones \cite{Viotti2022} en donde esta fase es robusta frente a los efectos del entorno, y por lo tanto es un candidato formidable para la metrología y medición de sistemas cuánticos.  
El cap\'itulo está estructurado de manera que en primer lugar se tratar\'a una descripción general de las fases geom\'etricas (FG) en el contexto de sistemas aislados, descritos consecuentemente mediante estados puros. Analizar este caso antes de centrar la atención en sistemas cuánticos abiertos permitirá asimilar nociones y ganar intuición sobre las fases geométricas en el marco de una teoría más simple. A lo largo del capítulo se trabajarán expresiones válidas bajo ciertas hipótesis, partiendo del caso menos general, y llegando al caso más general. Por lo tanto, al final del capítulo se presentará una definición particular para sistemas abiertos en evoluciones no-cíclicas ni unitarias, la cual se usará en los próximos capítulos.

\section{R\'egimen adiab\'atico y fase de Berry} \label{sec2:adiabatico}
La fase de Berry \cite{Berry1984} es una manifestacion fundamental del teorema adiab\'atico. Esta representa la fase acumulada por el autoestado de un Hamiltoniano $H(t)$ que var\'ia lentamente en un ciclo, que está relacionada con el circuito descrito por $H(t)$ en un dado espacio de par\'ametros. 

Para ver esto, se considera un Hamiltoniano $H(R(t))$ que depende explícitamente del tiempo a través de un parámetro $R=(R_1,R_2,\dots)$. Dado este Hamiltoniano, formalmente se pueden encontrar los autoestados instantáneos del sistema $\ket{\psi_n(R(t))}$ que satisfacen
\begin{equation}
    H(R(t))\ket{\psi_n(R(t))}=E_n(R(t))\ket{\psi_n(R(t))},
\end{equation}
suponiendo además que los autovalores satisfacen $E_1<E_2<\dots E_n$ de forma que no hay degeneraci\'on. Se considera que la evoluci\'on temporal de un estado cualquiera $\ket{\psi(t)}$ está dada por la ecuación de Schr\"odinger
\begin{equation}
    i \hbar \ket{\dot \psi(t)}=H(R(t))\ket{\psi(t)}.
\end{equation}
Desarrollando el estado en funci\'on de los autoestados instantáneos del Hamiltoniano, se puede resolver formalmente el problema
\begin{equation}
    \ket{\psi(t)} = \sum_n c_n(t)\ket{\psi_n(R(t))},
\end{equation}
donde los coeficientes \( c_n(t) \) satisfacen:
\[
i \hbar \dot{c}_n(t) = \left( E_n - i \hbar \langle \psi_n | \dot{\psi}_n \rangle \right) c_n(t) - i \hbar \sum_{m \neq n} \langle \psi_n | \dot{\psi}_m \rangle c_m(t).
\]

En el régimen adiabático, donde el Hamiltoniano cambia lentamente en comparación con las escalas internas del sistema, se desprecia el término de acoplamiento cruzado:
\[
\dot{c}_n(t) \approx -\frac{i}{\hbar} \left( E_n - i \hbar \langle \psi_n | \dot{\psi}_n \rangle \right) c_n(t).
\]

El estado resultante es:
\[
| \psi(t) \rangle = e^{-\frac{i}{\hbar} \int_0^t E_n(R(t')) \, dt'} e^{i \phi_n(t)} | \psi_n(R(t)) \rangle,
\]
donde \( \phi_n(t) = i \int_0^t \langle \psi_n(R(t')) | \nabla_R | \psi_n(R(t')) \rangle \cdot \dot{R}(t') \, dt' \) es la fase geométrica acumulada.

Para circuitos cerrados en el espacio de parámetros, la fase geométrica se expresa como:
\begin{equation}\label{ec2:fg berry}
    \phi_n(C) = i \oint_C \langle \psi_n(R) | \nabla_R | \psi_n(R) \rangle \cdot dR,    
\end{equation}
independiente de la velocidad con que se recorre el circuito. Sin embargo, la hipótesis para llegar a este resultado es que la velocidad de la evoluci\'on sea suficientemente lenta para que se puedan despreciar las transiciones no adiab\'aticas a otros niveles de energ\'ia. Por lo tanto, para llegar a este resultado no es totalmente independiente de la velocidad con la que se recorre el circuito en el espacio de parámetros.

\section{Fase de Aharonov-Anandan}\label{sec2:fase AA}

La formulación de Aharonov y Anandan permite definir una fase geométrica que es independiente de la evolución adiabática. Su propuesta se basa únicamente en la trayectoria del estado en el espacio proyectivo de rayos, sin referencia explícita al Hamiltoniano.

Considérese el espacio de Hilbert $\mathcal{H}$, y dentro de este, el subespacio \( N_0 \) que contiene vectores normalizados \( | \psi \rangle \). El espacio proyectivo \( P \) se define como el conjunto de clases de equivalencia bajo la relación \( | \psi \rangle \sim e^{i\alpha} | \psi \rangle \).  Estas colecciones $\xi = \{e^{i\alpha}\ket{\psi} \; ; \; 0 \leq \alpha \leq 2\pi\}$ denominadas rayos, agrupan en un \'unico elemento (la clase) todos los objetos equivalentes. Cada clase de equivalencia se denomina un rayo, y el mapeo \( \Pi : N_0 \to P \) proyecta un vector al rayo correspondiente.

Durante una evolución cíclica, el estado al tiempo inicial \( | \psi(0) \rangle \) y al tiempo final \( | \psi(T) \rangle \) pertenecen al mismo rayo, por lo que:
\[
| \psi(T) \rangle = e^{i\phi} | \psi(0) \rangle.
\]
Los estados solo pueden diferir en una fase total $\phi$. Para determinar la fase geométrica, se descompone \( \phi \) en dos contribuciones: una parte dinámica y una parte geométrica.

La relación entre el estado físico \( | \psi(t) \rangle \) y su clase de equivalencia \( \xi \in P \) se escribe como:
\[
| \psi(t) \rangle = e^{i f(t)} | \xi(t) \rangle,
\]
donde \( f(t) \) es una función que recoge la fase acumulada. Sustituyendo esta relación en la ecuación de Schrödinger:
\[
i \hbar \frac{\partial}{\partial t} | \psi(t) \rangle = H | \psi(t) \rangle,
\]
se obtiene una ecuación para \( f(t) \):
\[
\hbar \dot{f}(t) = -\langle \xi(t) | H | \xi(t) \rangle + i \hbar \langle \xi(t) | \dot{\xi}(t) \rangle.
\]

La fase total acumulada entre los tiempos \( 0 \) y \( T \) es:
\[
\phi = f(T) - f(0) = -\frac{1}{\hbar} \int_0^T \langle \xi(t) | H | \xi(t) \rangle \, dt + \int_0^T i \langle \xi(t) | \dot{\xi}(t) \rangle \, dt.
\]

Aquí, el primer término es la fase dinámica:
\[
\phi_{\text{din}} = -\frac{1}{\hbar} \int_0^T \langle \xi(t) | H | \xi(t) \rangle \, dt = -\frac{1}{\hbar}\int_0^T dt \, \bra{\psi(t)}H\ket{\psi(t)},
\]
y el segundo término corresponde a la fase geométrica:
\begin{equation} \label{eq:fg AA}
    \phi_{\text{AA}} = \int_0^T i \langle \xi(t) | \dot{\xi}(t) \rangle \, dt.
\end{equation}

Esta última expresión muestra que la fase geométrica depende únicamente de la trayectoria en el espacio proyectivo \( P \) y no del Hamiltoniano o la velocidad de la evolución. Al ser independiente de estos factores, refleja una propiedad puramente geométrica de la curva trazada por el estado en \( P \).



\subsection{Interpretación geométrica y caso no-cíclico}\label{sec2:no ciclico}
En esta sección se mostrará la interpretación geométrica y la generalización al caso no cíclico, demostrada por Samuel y Bhandari \cite{Bhandari1988}. Esta definición no requiere de la condición de ciclo cerrado, y tampoco requiere que el estado conserve su norma, como por ejemplo en una medición y colapso de la función de onda. Para esto es necesario dotar al espacio de Hilbert de geometría donde la fase surge de la estructura del espacio.

Para darle estructura al espacio, lo que ya hicimos antes es considerar un fibrado, donde definimos una clase de equivalencia para estados que difieren en una fase global. Para darle mayor estructura tenemos que introducir el concepto de conexión, que nos permitirá comparar elementos pertenecientes a fibras distintas mediante una regla de transporte paralelo. La regla de transporte paralelo nos dice que
\begin{equation} \label{ec2:transporte paralelo}
    \text{Im} \braket{\psi(t)}{\dot \psi(t)}=0. 
\end{equation}

Considérese una curva \( C: t \in [0, T] \to \ket{\psi(t)} \) sobre \( N_0 \), horizontal, y su vector tangente $\ket{\dot{\psi}(t)}/\braket{\psi(t)}{\psi(t)}$. La conexión natural
\begin{equation}
A = \frac{\text{Im} \bra{\psi(t)} \dot{\psi}(t) \rangle}{\bra{\psi(t)} \psi(t) \rangle},
\end{equation}
transforma, frente a transformaciones \( U(1) \) de gauge \( \ket{\psi(t)} \to e^{i\alpha(t)} \ket{\psi(t)} \), según
\begin{equation} \label{ec2:transformación de gauge}
A \to A + \dot{\alpha}(t).
\end{equation}

Dado que \( C \) es horizontal por definición, la ley de transporte paralelo de la Ec. (\ref{ec2:transporte paralelo}) impone que la conexión se anule a lo largo de la trayectoria del estado que le da origen. Si el vector de estado \( \ket{\psi(t)} \) está, además, asociado a una evolución cíclica en el sentido de Aharonov-Anandan, entonces retorna al rayo inicial en algún instante \( T \).

Considérese, en este escenario, la integral de la conexión \( A \) sobre el camino construido a partir de la curva \( \ket{\psi(t)} ; t \in [0, T] \), cerrada uniendo \( \ket{\psi(T)} \) con \( \ket{\psi(0)} \) sobre el rayo. Como se ha discutido, la curva \( \ket{\psi(t)} \) es horizontal por definición y, por lo tanto, la conexión se anula \( A = 0 \) sobre ella. Por otra parte, la integral sobre el tramo vertical que cierra el camino da como resultado la diferencia de fase entre \( \ket{\psi(T)} \) y \( \ket{\psi(0)} \):
\begin{equation}
\oint A dl_{N_0} = \int_C A + \int_{\text{rayo}} A = \text{arg} \bra{\psi(0)} \psi(T) \rangle.
\end{equation}

Es decir, la integral sobre el camino total (cerrado), es la diferencia de fase total entre el estado inicial y final. Por otra parte, la integral de la conexión \( A \) sobre una curva cerrada en \( N_0 \) es invariante por efecto de la ley de transformación Ec. (\ref{ec2:transformación de gauge}). La holonomía de la curva \( C \subset P \) asociada a la conexión \( A \) es entonces:
\begin{equation}
g(C) = e^{i \oint_C A} = e^{i\phi_{\text{AA}}}.
\end{equation}
En el caso de una evolución no cíclica, el vector que describe el sistema no vuelve a su rayo de partida. Para este caso se establece una manera de comparar estados de diferentes fibras. Dicha comparación se hace a través de la fase de \textit{Pancharatnam} \cite{Pancha1956}, definida para dos estados no-ortogonales cualesquiera como
\begin{equation}
    \phi_P = \arg \braket{\psi_1}{\psi_2}.
\end{equation}
Para hacer la generalización al caso no-cíclico, tenemos que dar un concepto de distancia, y para esto tenemos que hablar de líneas geodésicas. No vamos a meternos en detalle en esto, pero lo importante es que la fase en el caso no cíclico consiste de la diferencia entre la fase dinámica y la fase de Pancharatnam
\begin{equation}
    \phi_{SB}=-\phi_P-\frac{1}{\hbar}\int_0^Tdt\bra{\psi(t)}H\ket{\psi(t)}.
\end{equation}
Este método se puede utilizar para generalizar al caso no unitario, en el sentido de un estado puro que no conserva su norma. Este tipo de evolución puede suceder cuanto estamos teniendo en cuenta mediciones en el sistema, colapsos de la función de onda no conservan la norma según la regla de colapso de la mecánica cuántica. En este caso, si consideramos el estado inicial $\ket{\psi_0}$ sobre el cual se realizan mediciones sucesivas, de forma tal que la N-ésima proyección es otra vez el estado inicial, el estado final del sistema está dado por
\begin{equation}
    \ket{\psi_0}\braket{\psi_0}{\psi_{N-1}}\dots\braket{\psi_2}{\psi_1}\braket{\psi_1}{\psi_0}.
\end{equation}
Según el criterio de Pancharatnam los estados inicial y final tienen una diferencia de fase bien definida, dada por el argumento del número complejo que acompaña al estado $\ket{\psi_0}$.


\section{Enfoque cinemático}\label{sec2:cinematico}

En la mayoría de las discusiones sobre la fase geométrica, el punto de partida es la ecuación de Schrödinger para algún sistema cuántico particular caracterizado por un dado Hamiltoniano. Sin embargo, la fase geométrica es consecuencia de la cinemática cuántica, esto es, independiente del detalle respecto del origen dinámico de la trayectoria descrita en el espacio de estados físicos. Mukunda y Simon (\cite{Mukunda1993-1},\cite{Mukunda1993-2}) resaltaron la independencia de la fase geométrica respecto del origen dinámico de la evolución proponiendo un enfoque cinemático en el cual la trayectoria descrita en el espacio de estados físicos es el concepto fundamental para la fase geométrica. En su desarrollo, se parte de la consideración de una curva uniparamétrica y suave \( C \subset N_0 \), conformada por una dada secuencia de estados \( \ket{\psi(t)} \):
\begin{equation}
C = \{ \ket{\psi(t)} \in N_0 \mid t \in [0, T] \subset \mathbb{R} \},
\end{equation}
donde no se hace ninguna suposición respecto de si \( C \) es una curva abierta o cerrada, ni del origen dinámico de la secuencia de estados. Se observa luego detenidamente la cantidad \( \bra{\psi(t)} \dot{\psi}(t) \rangle \) construida a partir de esta curva. La condición de unitariedad implica que esta cantidad sea imaginaria pura, lo que puede escribirse como
\begin{equation}
\bra{\psi(t)} \dot{\psi}(t) \rangle = i \, \text{Im} \bra{\psi(t)} \dot{\psi}(t) \rangle.
\end{equation}

Por otra parte, aplicando una transformación \( U(1) \) de gauge
\begin{equation} \label{ec2:transformacion u1}
C \to C': \ket{\psi'(t)} = e^{i\alpha(t)} \ket{\psi(t)}, \quad t \in [0, T],
\end{equation}
la cantidad analizada transforma según
\begin{equation}
    \text{Im} \bra{\psi(t)} \dot{\psi}(t) \rangle \rightarrow  \text{Im} \bra{\psi(t)} \dot{\psi}(t) \rangle + \dot{\alpha}(t).
\end{equation}

Lo que se quiere conseguir es una funcional que sea invariante ante transformaciones $U(1)$ (Ec. (\ref{ec2:transformacion u1})), es decir, toma mismos valores para curvas $C$ y $C'$
\begin{equation} \label{ec2:fg cinematica unitaria}
    \phi_u[C] \equiv \text{arg} \braket{\psi(0)}{\psi(T)} - \Im \int_0^T dt \braket{\psi(t)}{\dot \psi(t)}
\end{equation}
Está permitido definir este funcional de la curva $C$ en el espacio de rayos, ya que es invariante ante reparametrizaciones. Algo importante de remarcar es que, si se aplica una transformación unitaria arbitraria a nuestro estado, entonces al cambiar el Hamiltoniano también cambiará la curva que describe el estado inicial en el espacio de Hilbert, y por lo tanto se puede mostrar que la fase geométrica cambia. Por suerte, en el caso que la transformación no depende del tiempo, entonces se demuestra que la fase no cambia. 

\section{Fases geométricas en sistemas abiertos}\label{sec2:sistemas abiertos}
Las secciones anteriores tratan la fase geométrica en diferentes casos, ascendientes en generalidad, ya que se logra relajar condiciones e hipótesis, y se llegó a una expresión general que satisface propiedades importantes, como invarianza ante transformaciones de fase global $U(1)$ y a reparametrizaciones monótonas. También dependen únicamente de la trayectoria descrita por el estado físico en el espacio de rayos y no del Hamiltoniano que genera dicha trayectoria, y finalmente son interpretables en términos puramente geométricos. 

Sin embargo, estamos asumiendo que el estado es puro durante toda su evolución, restricción que es una idealización y experimentalmente es necesario tener en cuenta que todo sistema físico está en contacto con un entorno. Se requiere entonces una descripción en términos de estados mixtos y evoluciones no unitarias. Muchos esfuerzos [\cite{Uhlmann1}-\cite{Singh2003}] se concentraron en definir la fase geométrica acumulada por un estado mixto, incluso existen reportes experimentales \cite{Du2003}. Otra ruta explorada considera el efecto del entorno como correcciones que permitan mantener las nociones de fase geométrica del caso unitario. Trabajos de este tipo introducen el efecto del entorno mediante un Hamiltoniano no hermítico \cite{Carollo2003,Carollo2005}, y otros estudian modificaciones a la fase de Berry por ruido clásico en el campo magnético \cite{DeChiara2003}, o por un entorno cuántico \cite{Whitney2003,Whitney2005}, tanto desde lo teórico como lo experimental \cite{Berger2013,Berger2015}.

El marco en el cual una fase geométrica para sistemas cuánticos abiertos debe definirse es el siguiente: se supone que el efecto del entorno sobre el sistema de interés es tal que, bajo aproximaciones adecuadas, el sistema puede tratarse \textit{efectivamente} como un sistema aislado que experimenta un tipo de evolución lineal no unitaria:
\begin{equation}
    \Sigma:\rho(0)\rightarrow\Sigma_t[\rho(0)] \equiv \rho(t),
\end{equation}
que da cuenta tanto de la dinámica interna del sistema como de su interacción con el entorno, y satisface una ecuación maestra. Una consecuencia de este enfoque es que, en el caso general, un estado inicial puro evoluciona en un estado mixto $\rho(t)$. El operador densidad que representa el estado del sistema admite una descomposición $\{ \ket{\psi_k(t)},\omega_k(t)\}$ en estados puros $\ket{\psi_k(t)}$ pesados con probabilidades $\omega_k(t)$, que permite expresarla como
\begin{equation}
    \rho(t)=\sum_k\omega_k(t)\ketbra{\psi_k(t)}{\psi_k(t)}.
\end{equation}
La asociación $\rho(t)\rightarrow \{ \ket{\psi_k(t)},\omega_k(t)\}$ entre el operador densidad y el \textit{ensamble} de estados $\{\ket{\psi_k(t)}\}$ no es uno-a-uno, sino uno-a-muchos, lo que significa que en general existen diferentes ensambles, con diferentes estados y diferentes pesos, que sin embargo tienen la misma matriz densidad. Esto imposibilita la distinción entre estas situaciones solamente con la información que proporciona la matriz densidad.

Una estrategia recurrente en la literatura que aborda el problema de asociar una fase geométrica a un estado mixto $\rho(t)$ es descomponer formalmente la matriz densidad en una mezcla estadística como la de la ecuación anterior, y aplicar la fase unitaria Ec. (\ref{ec2:fg cinematica unitaria}) sobre cada elemento de la mezcla para asociar una fase a $\rho(t)$. Esto fue propuesto, desde una descripción en términos de operadores de saltos en \cite{Carollo2003,Carollo2005} y posteriormente en \cite{Sjoqvist2009}-\cite{Buric2009}. En una aproximación diferente al problema,  Tong et al. \cite{Tong2004}  propone una definición de fase geométrica que se vale de una purificación del estado, pero resulta independiente de la elección que se utilice para purificar. La siguiente sección desarrolla esta propuesta en particular.

\subsection{Enfoque cinemático en sistemas abiertos}
La introducción teórica concluye con esta sección, siguiendo la propuesta de Tong et al. \cite{Tong2004} para la fase geométrica en sistemas cuánticos abiertos. Para ésto, se considera un sistema y el espacio de Hilbert $\mathcal{H}$ de dimensión $N$. La evolución del estado puede describirse como una curva $C \subset \mathcal{P}$
\begin{equation}
    C:t\in[0,T] \rightarrow \rho(t) = \sum_{k=1}^N\omega_k(t)\ketbra{\psi_k(t)}{\psi_k(t)},
    \label{ec2:mapa rho}
\end{equation}
donde $\omega_k(t)\geq 0$ y $\ket{\psi_k(t)}$ son los autovalores y autoestados, respectivamente, de la matriz densidad $\rho(t)$ del sistema. Por simplicidad se asume que las funciones $\omega_k(t)$ que no son nulas, son no degeneradas en el intervalo de estudio $[0,T]$, y se refiere al trabajo original \cite{Tong2004} para su generalización al caso degenerado.

Para introducir una noción de fase geométrica bajo estas condiciones, se comienza por realizar una purificación del estado mixto, haciendo uso de un sistema auxiliar con un espacio de Hilbert de igual dimensión que el espacio original. El estado mixto se eleva entonces a un estado purificado de mayor dimensión
\begin{equation}
    \ket{\Psi(t)}=\sum_{k=1}^N\sqrt{\omega_k(t)}\ket{\psi(t)}\otimes\ket{a_k},
\end{equation}
donde $\ket{\Psi(t)}\in \mathcal{H}\otimes\mathcal{H}_{aux}$ es la purificación de $\rho(t)$, en el sentido de que la matriz densidad se recupera tomando traza parcial sobre el espacio auxiliar. 

La fase de Pancharatnam entre las purificaciones inicial y final puede escribirse como
\begin{equation} 
    \phi_P=\arg \left( \sum_{k=1}^{N} \sqrt{\omega_k(0)\omega_k(T)}\braket{\psi_k(0)}{\psi_k(T)} \right),
    \label{ec2:fase pan}
\end{equation}
Para extraer la fase asociada al sistema de interés, es necesario eliminar la dependencia en la purificación específica utilizada. Para esto, sabiendo que para cada instante $t\in[0,T]$ las bases $\{\ket{\psi_k(t)}\}$ y $\{\ket{\psi_k(0)}\}$ son bases ortonormales del mismo espacio, existe entonces una transformación que lleva de un conjunto a otro $\ket{\psi_k(t)}=U(t)\ket{\psi_k(0)} ,  \; \forall k$. El paso esencial para arribar a una fase puramente geométrica es el de notar que en realidad, existe una clase de equivalencia de mapas unitarios $\tilde U(t)$ que realizan todas la misma curva $C$. Específicamente, la expresión de la Ec. (\ref{ec2:mapa rho}) que defina le curva es manifiestamente invariante ante transformaciones de gauge $U(1)$, de forma que dos transformaciones unitarias $U(t)$ y $U'(t)$ que mapeen $\{\ket{\psi_k(0)}\}$ en $\{\ket{\psi_k(t)}\}$ o en $\{e^{i\alpha_k(t)}\ket{\psi_k(t)}\}$ resultan equivalentes. Los mapas $U(t)$ que en la clase de equivalencia tienen la forma 
\begin{equation}
    \tilde U(t) = U(t)\sum_{k=1}^{N}e^{i\alpha_k(t)}\ketbra{\psi_k(0)}{\psi_k(0)}.
    \label{ec2:op evo}
\end{equation}
En particular, puede identificarse el mapa $U^\parallel(t)$ que satisface la condición de transporte paralelo para cada $\ket{\psi_k(t)}$, es decir
\begin{equation}
U^\parallel(t)=U(t) | \bra{\psi_k(0)}U^\dagger(t)\dot U(t)\ket{\psi_k(0)}=0 \; |  \; \forall k,
\end{equation}
y definir la fase geométrica como la diferencia de fase Ec. (\ref{ec2:fase pan}) para este mapa particular. Sustituyendo en la Ec. (\ref{ec2:op evo}) que describe la relación de equivalencia entre operadores, se obtiene que 
\begin{equation}
    \alpha_k(t)=i\int_{0}^{t}dt'\bra{\psi_k(0)}U^\dagger(t')\dot U(t)\ket{\psi_k(0)},
\end{equation}
y en consecuencia, la fase geométrica resulta

\begin{equation}
    \phi_g[C]=\arg \left( \sum_{k=1}^{N} \sqrt{\omega_k(0)\omega_k(t)} \braket{\psi_k(0)}{\psi_k(T)}e^{-\int_{0}^{T}dt\braket{\psi_k(t)}{\dot\psi_k(t)}} \right).
    \label{ec2:fg general}
\end{equation}
La definición propuesta satisface las condiciones que rigen sobre una noción geométrica razonable para un estado mixto, que son: (i) Efectivamente es una fase, dado que su definición a través de la función argumento impone una periodicidad bien definida, (ii) es manifiestamente invariante de gauge ya que toma el mismo valor para cualquier operador unitario $U(t)$ en la clase de equivalencia descrita por la Ec. (\ref{ec2:op evo}), y por lo tanto depende únicamente por el camino $C$ trazado por la matriz $\rho(t)$ del sistema y (iii) cuando la evolución es unitaria, se recuperan los resultados anteriores para estados iniciales puros, y para estados iniciales mixtos \cite{Singh2003} y \cite{Sjoqvist2000}. Finalmente (iv) es accesible experimentalmente, por ejemplo usando interferometría o tomografía cuántica.

Será de utilidad para su aplicación, el caso particular en que el sistema se encuentre inicialmente en un estado puro $\ket{\psi(0)}$. En tal situación, la descomposición de la matriz densidad del sistema en el instante inicial solo tendrá un autovalor distinto de cero: $\omega_+(0)=1$. En consecuencia, la sumatoria en la Ec. (\ref{ec2:fg general}) posee un único termino no nulo, y la formula se reduce a 
\begin{equation}
    \phi_g[C]=\arg{\braket{\psi(0)}{\psi_+(T)}}-\Im \int_{0}^{T}dt \braket{\psi_+(t)}{\dot\psi_+(t)}.
    \label{ec2:fg general puro}
\end{equation}
Esta expresión admite la interpretación de fase geométrica acumulada por el autoestado $\ket{\psi_+(t)}$. Esta formulación ha sido ampliamente usada por su interés teórico y experimental en diferentes sistemas \cite{fg1}-\cite{fg6}.
