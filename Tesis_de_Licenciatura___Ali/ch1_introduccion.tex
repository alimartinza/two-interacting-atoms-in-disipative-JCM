\chapter{Introducción}
\label{ch:intro}

%CAMBIAR ESTO PARA PERSONALIZARLO A MI GUSTO
\pagestyle{fancy}
\fancyhf{}
\fancyhead[LE]{\nouppercase{\rightmark\hfill}}
\fancyhead[RO]{\nouppercase{\leftmark\hfill}}
\fancyfoot[LE,RO]{\hfill\thepage\hfill}

La mecánica cuántica ha transformado nuestra comprensión de muchos sistemas físicos, incorporando conceptos fundamentales como la superposición y el entrelazamiento. Estos fenómenos no solo han sido fundamentales para el desarrollo de teorías físicas avanzadas, sino que también han dado lugar a una nueva era en tecnología de la información cuántica, donde el entrelazamiento y la superposición son elementos fundamentales, utilizados como recursos, y elevando la capacidad de los sistemas. Un ejemplo crucial de esta revolución son las computadoras cuánticas, que aprovechan las superposiciones de estados para ejecutar algoritmos físicamente viables. 

%Estos algoritmos permiten reducir significativamente los tiempos de cómputo, transformando problemas complejos de decisión, cuya solución no se puede encontrar rápidamente con computadoras clásicas, en problemas que pueden resolverse en tiempos polinómicos. Los problemas de decisión pertenecen a una clase conocida como tiempo polinómico no determinista, que son aquellos cuya solución puede ser verificada rápidamente, pero cuya resolución directa puede ser extremadamente difícil y consumir mucho tiempo. Los problemas más difíciles dentro de esta clase, llamados problemas NP-completos, se cree que no pueden resolverse de manera eficiente, lo que hace que las computadoras cuánticas representen una alternativa prometedora \cite{Shor1999}. 

En este contexto, los sistemas de electrodinámica cuántica de cavidades han emergido como una plataforma crucial para la implementación de procesos cuánticos controlados. En estos sistemas, se confina al campo electromagnético en una cavidad óptica de alta calidad, permitiendo interacciones controladas entre la radiación y átomos individuales o átomos artificiales de dos niveles, como los qubits superconductores. Esta capacidad de manipulación precisa ha convertido a la electrodinámica de cavidades en un campo clave para el desarrollo de la computación cuántica y las simulaciones cuánticas controladas. 

Uno de los modelos más importantes en este ámbito es el modelo de Jaynes-Cummings (JCM) \cite{JCoriginal}, que describe la interacción entre el campo electromagnético y un sistema atómico de dos niveles. Su relevancia en la física cuántica radica en que proporciona un marco teórico claro para entender la dinámica de sistemas cuánticos abiertos y su impacto en la coherencia cuántica. Experimentos recientes con circuitos superconductores han logrado simular de manera efectiva el modelo de Jaynes-Cummings, permitiendo el estudio de la dinámica de fotones y átomos artificiales en condiciones altamente controladas. En particular, el JCM permite explorar la transferencia de excitaciones entre la luz y la materia a nivel cuántico, lo que ha sido crucial también en experimentos con circuitos superconductores y cavidades o resonadores de microondas. Su extensión a configuraciones de múltiples átomos ha permitido una exploración detallada de la interacción entre átomos y la generación de estados entrelazados, lo cual es fundamental para el desarrollo de tecnologías cuánticas. 

En este contexto, la fase geométrica (FG) se ha consolidado como un concepto central en la mecánica cuántica. Introducida inicialmente por Berry \cite{Berry1984} en el caso de sistemas cerrados con evoluciones adiabáticas, este concepto ha sido extendido a escenarios más generales, incluyendo evoluciones no adiabáticas y sistemas abiertos. Su importancia nace de su capacidad para proporcionar información sobre la estructura del espacio de estados, además de su potencial para aplicaciones prácticas en la computación cuántica y metrología cuántica \cite{Ericsson2000,Johnsson2020,Shapere1989}. En sistemas de Jaynes-Cummings, la fase geométrica se manifiesta en la evolución del estado del sistema cuando se recorren trayectorias cerradas en el espacio de parámetros, lo que permite inferir información sobre la coherencia cuántica y la interacción con el entorno. En experimentos recientes con qubits superconductores acoplados a cavidades, se ha logrado medir con precisión la acumulación de fase geométrica, lo que ha permitido validar predicciones teóricas y explorar su posible uso en puertas lógicas cuánticas robustas. Además, la FG se ha relacionado con efectos topológicos, y mucho trabajo en el área ha logrado resultados tanto experimentales como teóricos. En particular, aplicaciones en la computación cuántica incluyen implementaciones para la creación de compuertas lógicas utilizando computación cuántica topológica \cite{Vedral2003,Wilczek1984,Zee1988}. Esfuerzos recientes también buscan consolidar a la FG como método para estudiar el entrelazamiento entre dos átomos \cite{Ganesh2025}. Esto es útil para medir la fidelidad de compuertas cuánticas, ya que los métodos actuales son ineficientes para sistemas de muchos cuerpos. Estos se basan en mediciones proyectivas, lo cual crece exponencialmente con el número de partículas, en cambio, la FG es solo un parámetro y no escala con el tamaño del sistema.

Por otro lado, el entrelazamiento cuántico es un recurso fundamental para la información cuántica, ya que permite la implementación de protocolos como la criptografía cuántica y la teleportación. En particular, la dinámica del entrelazamiento en sistemas de Jaynes-Cummings ha sido objeto de numerosos estudios. Trabajos pioneros como los de Eberly y Yu \cite{Yu2009} han analizado fenómenos como la "muerte súbita del entrelazamiento", lo que ha permitido comprender mejor la influencia del entorno en la evolución de sistemas cuánticos. Además, se han desarrollado estrategias para generar y controlar estados altamente entrelazados en estos sistemas, lo cual tiene importancia directa en la implementación de redes cuánticas y simulaciones de materiales exóticos. Experimentos recientes han utilizado circuitos superconductores para generar estados entrelazados de manera eficiente, demostrando la viabilidad de estos sistemas para la implementación de algoritmos cuánticos y la creación de canales seguros de comunicación cuántica.

La electrodinámica cuántica de cavidades es un campo de investigación estrechamente vinculado con la computación cuántica. La posibilidad de manipular la interacción luz-materia en cavidades de alta calidad ha permitido el desarrollo de experimentos en arquitecturas de circuitos superconductores con qubits y/o con sistemas híbridos, proporcionando una plataforma versátil para la implementación de compuertas lógicas cuánticas y la creación de arquitecturas escalables para la computación cuántica. A medida que los experimentos han avanzado, se han abierto nuevas vías para explorar el entrelazamiento, la coherencia y la implementación de algoritmos cuánticos en estos sistemas. De manera particular, sistemas de electrodinámica de cavidades han sido utilizados para la implementación de simulaciones cuánticas de materiales exóticos, lo que abre la posibilidad a un mayor entendimiento de las fases topológicas de la materia y a nuevas formas de diseñar dispositivos cuánticos robustos ante errores y frente al ruido o decoherencia inducida por los entornos.

En esta tesis, se estudiará la dinámica de entrelazamiento y la fase geométrica en un sistema de dos átomos acoplados a una cavidad, modelado mediante la extensión del modelo de Jaynes-Cummings. Se analizarán las condiciones bajo las cuales el entrelazamiento se preserva o se disipa, así como la influencia de los acoplamientos externos en la acumulación de la fase geométrica. Este análisis permitirá obtener una visión más clara de los mecanismos que afectan la coherencia cuántica en estos sistemas.

El presente trabajo no solo contribuye a una mejor comprensión de la mecánica cuántica fundamental, sino que también tendrá aplicaciones potenciales en el desarrollo de tecnologías cuánticas avanzadas. Los resultados obtenidos podrían ser relevantes para el diseño de dispositivos cuánticos de alta fidelidad, la optimización de protocolos de computación cuántica y la exploración de nuevos esquemas para la simulación de sistemas cuánticos complejos. Además, el estudio de la fase geométrica en sistemas abiertos contribuirá a la identificación de nuevas estrategias para la implementación de operaciones cuánticas resistentes al ruido y la decoherencia, un aspecto clave para el desarrollo de computadoras cuánticas funcionales.
