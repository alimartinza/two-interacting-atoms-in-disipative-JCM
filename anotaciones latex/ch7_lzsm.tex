\chapter{Landau-Zener-Stückelberg-Majorana}
\label{ch7_lzsm}


%CAMBIAR ESTO PARA PERSONALIZARLO A MI GUSTO
\pagestyle{fancy}
\fancyhf{}
\fancyhead[LE]{\nouppercase{\rightmark\hfill}}
\fancyhead[RO]{\nouppercase{\leftmark\hfill}}
\fancyfoot[LE,RO]{\hfill\thepage\hfill}

\section{Introducción}
En este capitulo vamos a tratar el problema de Landau-Zener-Stückelberg-Majorana (LZSM). Primero vamos a ver los aspectos teoricos generales, concentrandonos en la solucion analitica dada por Zener en 1932 \cite{zener1932non} resolviendo la ecuacion diferencial de Weber en funcion de las funciones $U(a,z)=D_{-a-\frac{1}{2}}(z)$ de parabolic cylinder. Vamos a prestar atencion a la fase geometrica. Mediante la solucion analitica se puede ver que las fases son importantes en la evolucion, especialmente cuando consideramos multiples pasajes por el punto de maximo acercamiento ya que puede haber interferencia constructiva o destructiva (llamada Coherent Destruction of Tunneling, CDT) dependiendo de las fases adquiridas en cada pasaje. La fase de Stokes es una fase que se adquiere al atravesar el punto de maximo acercamiento (en realidad esta fase es la fase que se acumula en el limite que comenzamos y terminamos la evolucion muy lejos del punto de maximo acercamiento, y es la fase total acumulada, osea no depende del tiempo digamos, solo es el valor asintotico luego de una pasada.) y es importante para entender la interferencia. El estudio de esta fase es interesante, y lo que queremos hacer es contrastar esta fase con la fase geometrica. Por un lado, podemos hacer una evolucion adiabatica y ciclica, y ver si podemos obtener la fase de Berry. Luego, vamos a intentar de calcular la fase geometrica generalizada para una evolucion no ciclica y no adiabatica, y en particular podemos tomar el limite de evolucion adiabatica y ciclica para ver si recuperamos la fase de Berry, y sus correcciones no adiabaitcas. Finalmente, haremos un analisis numerico del sistema bajo un atajo adiabatico. Utilizando este atajo, queremos ver como se modifica la fase geometrica y si aparecen nuevas correcciones, o si las correcciones no adiabaticas se anulan. 
  
Esto me lo dijo el CoPilot: Luego vamos a ver como se puede implementar este modelo en un sistema de dos niveles acoplados a un modo de campo cuantico, y como se puede resolver numericamente el problema usando la libreria QuTiP \cite{johansson2012qutip,johansson2013qutip2}. Finalmente vamos a ver algunos resultados numericos y compararlos con los resultados teoricos.

\section{El modelo LZSM}

\begin{equation}
H(t) = -\frac{\Delta}{2}\sigma_x - \frac{vt}{2}\sigma_z
\end{equation}

Base diabatica o computacional: $\{|0\rangle, |1\rangle\}$ con $|0\rangle = \begin{pmatrix}1 \\ 0\end{pmatrix}$ y $|1\rangle = \begin{pmatrix}0 \\ 1\end{pmatrix}$.
Estos son autoestados del Hamiltoniano cuando $\Delta=0$ en cuyo caso sus energias son $E_{0,1}(t)=\mp \frac{vt}{2}$.

Los estados adiabaticos son los autoestados del Hamiltoniano en todo momento $H(t)\ket{E_\pm(t)}=E_\pm(t)\ket{E_\pm(t)}$, con energias $E_\pm(t) = \pm \frac{1}{2}\sqrt{(vt)^2 + \Delta^2}$.

Los autoestados son 
\begin{equation}
    \ket{E_\pm(t)}=\gamma_\mp(t)\ket{0}\mp \gamma_\pm(t)\ket{1}
\end{equation}
con $\gamma_\pm(t)=\sqrt{\frac{1}{2}\left(1\pm \frac{vt}{\sqrt{(vt)^2+\Delta^2}}\right)}$. Vemos que $\Delta E(t)=E_+(t)-E_-(t)=\sqrt{(vt)^2+\Delta^2}$ es el gap entre los niveles, y que el minimo gap es $\Delta E_{min}=\Delta$ en $t=0$.

\section{Solución analítica del problema LZSM}

Para resolver el problema vamos a escribir el estado en la base diabatica como $\ket{\psi(t)}=\alpha(t)\ket{0}+\beta(t)\ket{1}$, y la ecuacion de Schrodinger nos da el sistema de ecuaciones diferenciales
\begin{equation}
    i\hbar\frac{d}{dt}\ket{\psi(t)}=H(t)\ket{\psi(t)}
\end{equation}

\begin{align}
    i\hbar\dot{\alpha}(t) &= -\frac{vt}{2}\alpha(t) - \frac{\Delta}{2}\beta(t) \\
    i\hbar\dot{\beta}(t) &= -\frac{\Delta}{2}\alpha(t) + \frac{vt}{2}\beta(t)
\end{align}
De aqui podemos obtener una ecuacion de segundo orden para $\alpha(t)$ y $\beta(t)$, para la cual necesitamos cambiar de variables $\tau = \sqrt{\frac{v}{2\hbar}}t$ y definimos el parametro de adiabaticidad $\delta=\frac{\Delta^2}{4 \hbar v}$. Entonces, tenemos
\begin{align}
    & \frac{d^2}{d\tau^2}\alpha(\tau) + \left(-i + 2\delta + \tau^2\right)\alpha(\tau) = 0 \\
    & \frac{d^2}{d\tau^2}\beta(\tau) + \left(i + 2\delta + \tau^2\right)\beta(\tau) = 0
\end{align}

Luego cambiamos de variable $z=\sqrt{2}e^{-i\frac{\pi}{4}}\tau$ y obtenemos la ecuacion diferencial de Weber
\begin{align}
    &\frac{d^2}{dz^2}\alpha(z) + \left(-i\delta-\frac{1}{2}-\frac{z^2}{4}\right)\alpha(z)=0 \\
    &\frac{d^2}{dz^2}\beta(z) + \left(-i\delta+\frac{1}{2}-\frac{z^2}{4}\right)\beta(z)=0
\end{align}
Las soluciones son las \textit{parabolic cylinder functions} $U(a,z)=D_{-a-\frac{1}{2}}(z)$ \cite{abramowitz1964handbook} 
\begin{align}
    &\alpha(z)=A_+ U(\frac{1}{2}+i\delta, z) + A_- U(\frac{1}{2}+i\delta, -z) \\
    &\beta(z)=B_+ U(-\frac{1}{2}+i\delta, z) + B_- U(-\frac{1}{2}+i\delta, -z)
\end{align}
donde las constantes $A_\pm$ y $B_\pm$ se determinan por las condiciones iniciales. Reemplazando en la ecuacion original, se obtiene la relacion entre las constantes
\begin{equation}
    B_\pm = -\frac{\sqrt{2\delta}}{i} e^{\mp i\frac{\pi}{4}} A_\pm
\end{equation}
y dada una condicion inicial en $z=z_i$, se puede determinar $A_\pm$.
\begin{align}
    A_+=\frac{\Gamma(1+i\delta)}{\sqrt{2\pi}} \left[ \alpha(z_i)U(-\frac{1}{2}+i\delta, -z_i) - \beta(z_i) \sqrt{\delta} e^{i\frac{\pi}{4}} U(\frac{1}{2}+i\delta, -z_i) \right] \\
    A_-=\frac{\Gamma(1+i\delta)}{\sqrt{2\pi}} \left[ \alpha(z_i)U(-\frac{1}{2}+i\delta, z_i) + \beta(z_i)\sqrt{\delta} e^{i\frac{\pi}{4}} U(\frac{1}{2}+i\delta, z_i) \right]
\end{align}

y entonces la solucion a todo tiempo para los coeficientes $\alpha(z)$ y $\beta(z)$ es
\begin{equation}
    \begin{pmatrix}
    \alpha(z) \\
    \beta(z)
    \end{pmatrix} =
    \begin{pmatrix}
        \Xi_{11} & \Xi_{12} \\
        \Xi_{21} & \Xi_{22}
    \end{pmatrix} 
    \begin{pmatrix}
    \alpha(z_i) \\
    \beta(z_i)
    \end{pmatrix}
\end{equation}

\begin{align}
    \Xi_11=&\frac{\Gamma(1+i\delta)}{2\pi} \left[ U(-\frac{1}{2}+i\delta, -z_i)U(\frac{1}{2}+i\delta, z) + U(-\frac{1}{2}+i\delta, z_i)U(\frac{1}{2}+i\delta, -z) \right] \\
    \Xi_{12}=&\frac{\Gamma(1+i\delta)}{2\pi} \sqrt{\delta} e^{i\frac{\pi}{4}} \left[ U(\frac{1}{2}+i\delta, z_i)U(\frac{1}{2}+i\delta, -z) - U(\frac{1}{2}+i\delta, -z_i)U(\frac{1}{2}+i\delta, z) \right] \\
    \Xi_{21}=&\frac{\Gamma(1+i\delta)}{2\pi} \sqrt{\delta} e^{-i\frac{\pi}{4}} \left[ U(-\frac{1}{2}+i\delta, z_i) U(-\frac{1}{2}+i\delta, -z)- U(-\frac{1}{2}+i\delta, -z_i)U(-\frac{1}{2}+i\delta, z) \right] \\
    \Xi_{22}=&\frac{\Gamma(1+i\delta)}{2\pi} \left[ U(\frac{1}{2}+i\delta, -z_i)U(-\frac{1}{2}+i\delta, z) + U(\frac{1}{2}+i\delta, z_i)U(-\frac{1}{2}+i\delta, -z) \right]
\end{align}
\section{Fase de Stokes}
Tomando la condicion inicial en $t_i \to -\infty$ y el tiempo final en $t_f \to +\infty$, y usando las propiedades asintoticas de las funciones $U(a,z)$, 
\begin{equation}
    \lim_{|z|\to\infty} U(\frac{1}{2}+i\delta,z) \approx e^{-\frac{z^2}{4}} z^{-i\delta+\frac{1}{2}} - \frac{\sqrt{2\pi}}{\Gamma(1+i\delta)} e^{-\pi\delta+i\pi} e^{\frac{z^2}{4}} z^{i\delta} \,\, \text{si } \arg(z)=\frac{-3\pi}{4},
\end{equation}
\begin{equation}
    \lim_{|z|\to\infty} U(\frac{1}{2}+i\delta,z) \approx e^{-\frac{z^2}{4}} z^{-i\delta+\frac{1}{2}} \,\, \text{si } \arg(z)=\frac{\pi}{4}.
\end{equation}
Ademas podemos escribir
\begin{equation}
    \Gamma(i\delta)=\sqrt{\frac{\pi}{\delta\sinh(\pi\delta)}} e^{-i \arg\Gamma(i\delta)}
\end{equation}
con $\arg\Gamma(i\delta)=-\arg\Gamma(-i\delta)=-\frac{\pi}{2}-\arg(\Gamma(1-i\delta))$. Juntando esto se obtienen los coeficientes de la matriz de transmicion 
\begin{align}
    \Xi_{11} &\approx e^{-\pi\delta} \equiv T \\
    \Xi_{12} &\approx -\frac{\sqrt{2\pi}}{\Gamma(-i\delta)} e^{-\frac{\pi\delta}{2}-i\frac{\pi}{4}} \\
\end{align}
donde $T$ es la probabilidad de permanecer en el mismo estado diabatico. Le llamamos $T$ al coeficiente de transmicion, ya que el estado diabatico $\ket{0}$, que inicialmente es el estado fundamental, luego se convierte en el estado excitado, por lo tanto hacemos una transicion entre los estados adiabaticos. Saltamos del nivel mas bajo de energia al mas alto con una taza de transicion $T$.

Con estos resultados se deriva el metodo de la matriz de transmicion, que asume que la evolucion se puede aproximar como adiabatica excepto en el punto de maximo acercamiento, donde hay una transicion no adiabatica brusca, timpo impulso. Este metodo da resultados correctos si el punto inicial y final de la evolucion estan lejanos al punto de $\epsilon=0$, y se puede tambien implementar facilmente para ver que sucede en el caso de multiples pasajes.
\subsection{Pasajes multiples}
Consideremos que el sistema pasa dos veces por el punto de maximo acercamiento, es decir, hacemos una evolucion desde $t_i$ hasta $t_f$ con $t_f=-t_i=\tau >>1$, y luego el camino inverso, podemos recorrer el camino tantas veces como querramos.   
\section{Fase geométrica en el problema LZSM}

Mi idea es la siguiente. Primero, utilizar el resultado exacto para intentar de calcular la fase geometrica en el caso general

\begin{equation}
    \phi_g(t) = \arg\langle \psi(t_i)|\psi(t)\rangle + i \int_{t_i}^{t} dt' \langle \psi(t')|\dot{\psi}(t')\rangle = \arg\langle \psi(t_i)|\psi(t)\rangle - \Im \int_{t_i}^{t} dt' \langle \psi(t')|\dot{\psi}(t')\rangle.
\end{equation}

Por ahora avance un poco, pero creo que eventualmente hay que hacer integraciones numericas. Por ahora estoy considerando que el tiempo inicial es muy lejano y entonces el estado inicial $\ket{1}$ es el estado \textit{ground}. Con esto solo quedan pocos terminos, pero que igual son complejos. La generalizacion a un estado inicial cualquiera debe ser mas complicada mas que nada por el primer termino del argumento, la integral va a tener muchos terminos que van a ser similares entre si, una vez que se resuelva un termino los demas deberian salir parecido. 
Escribir formalmente como es la fase geometrica para un estado inicial arbitrario es facil, dado $\ket{\psi(t_i)}=\alpha_0\ket{0}+\beta_0\ket{1}$ usando el resultado exacto es facil ver que 
\begin{equation}
    \ket{\psi(t)} = (\alpha_0\Xi_{11}(t)+\beta_0\Xi_{12}(t))\ket{0} + (\alpha_0\Xi_{21}(t)+\beta_0\Xi_{22}(t))\ket{1}
\end{equation}
Entonces 
\begin{equation}
    \begin{aligned}
        \phi_g(t) &= \arg[\alpha_0^*(\alpha_0\Xi_{11}(t)+\beta_0\Xi_{12}(t)) + \beta_0^*(\alpha_0\Xi_{21}(t)+\beta_0\Xi_{22}(t))] \\
    & - \Im \int_{t_i}^{t} dt' \left\{ |\alpha_0|^2(\Xi_{11}^*(t')\dot{\Xi}_{11}(t')+\Xi_{21}^*(t')\dot{\Xi}_{21}(t')) \right. \\
    &+ |\beta_0|^2(\Xi_{12}^*(t')\dot{\Xi}_{12}(t')+\Xi_{22}^*(t')\dot{\Xi}_{22}(t')) \\
    & +\beta_0^*\alpha_0(\Xi_{12}^*(t')\dot{\Xi}_{11}(t')+\Xi_{22}^*(t')\dot{\Xi}_{21}(t')) \\
    & \left. + \alpha_0^*\beta_0(\Xi_{11}^*(t')\dot{\Xi}_{12}(t')+\Xi_{21}^*(t')\dot{\Xi}_{22}(t')) \right\}
    \end{aligned}
\end{equation}
Entonces estuve intentando de simplificar un poco esto, para el caso particular que el estado inicial es el estado ground, osea $\alpha_0=0$ y $\beta_0=1$. Algunas mejoras se pueden hacer, como escribir la derivada $\dot \Xi$ en funcion de otras cosas, pero tampoco es demasiado util. Pero haciendo numerics podria funcionar ver que onda. 

Entonces la idea para mi seria la siguiente. Ver si puedo mejorar mas las expresiones y llegar a algo un poco mas cerrado. Luego, lo que estoy bastante seguro que se va a poder es parecido a lo del TM, que seria tomar limites asintoticos, y ver si aparecen correcciones no adiabaticas. Aca estoy pensando que la fase de Berry seria la fase de Stokes, pero no estoy seguro. Esto tengo que revisarlo mejor.

Una vez hecho esto, lo que quiero hacer es ver que pasa si implementamos un atajo adiabatico. La idea es la siguiente. El atajo adiabatico nos permite hacer una evolucion que imita a la evolucion adiabatica, pero en un tiempo finito. Entonces, si hacemos el atajo adiabatico, y tomamos el tiempo final muy grande, deberiamos recuperar la fase de Berry. Pero si tomamos un tiempo final finito, deberiamos ver correcciones no adiabaticas. La pregunta es si estas correcciones son las mismas que las que aparecen en la evolucion sin atajo adiabatico, o si son diferentes. 

\section{Atajo adiabático en el problema LZSM}
La implementacion del atajo adiabatico es sencilla, consideramos el Hamiltoniano H(t), y le aplicamos una transformacion unitaria $U(t)$ tal que $H' = U(t) H(t) U^\dagger(t)$ es diagonal, es decir, la transformacion unitaria $U(t)$ consta de los autoestados instantaneos del Hamiltoniano.
Ahora, le agregamos el termino de \textit{conter-driving} $\tilde H(t) = H(t)+ H_{CD}(t)$, y si le aplicamos la misma transformacion unitaria, entonces
\begin{equation}
    \tilde H' = U(t) \tilde H(t) U^\dagger(t) - i\hbar  U(t) \dot U^\dagger(t) = UHU^\dagger+\left(UH_{CD}U^\dagger - i\hbar  U(t) \dot U^\dagger(t)\right)
\end{equation}
Como impusimos que $UHU^\dagger$ es diagonal, entonces es suficiente pedir que el segundo termino sea cero para que las poblaciones de los autoestados se mantengan constantes. 
Tambien podemos escribirlo en terminos de los autoestados instantaneos $H(t)\ket{\psi_n(t)}=E_n(t)\ket{\psi_n(t)}$, y se obtiene
\begin{equation}
    H_{CD} = i \hbar \sum_{m\neq n }\sum_n \dfrac{\ket{\psi_m}\bra{\psi_m} \dot H \ket{\psi_n}\bra{\psi_n}}{E_n-E_m}
\end{equation}
Para el Hamiltoniano de LZSM $H(t)=-(\Delta(t)\sigma_x+\epsilon(t)\sigma_z)/2$ se obtiene que 
\begin{equation}
    H_{CD}(t)=\frac{\hbar}{2}\dot\theta(t)\sigma_y
\end{equation}
donde $\theta(t)=\arctan\frac{\Delta(t)}{\epsilon(t)}$. En el caso de driving lineal que estamos considerando, entonces se obtiene que el hamiltoniano total es 
\begin{equation}
    \tilde H(t)=-\frac{\Delta}{2}\sigma_x-\frac{vt}{2}\sigma_z+\frac{v\Delta}{\Delta^2+v^2t^2}\sigma_y
\end{equation}
Resolver los autoestados y autovalores no debe ser demasiado complicado, pero resolver la dinamica de un estado inicial cualquiera debe ser muy dificil, ya que las ecuaciones diferenciales para los parametros deben ser aun mas complicados que los de la ecuacion de Weber. Pero si podemos resolver numericamente sin mayores dificultades creo yo, y podemos ver la FG. No creo que se pueda obtener nada demasiado analitico, pero voy a probar cuando llegue ahi. 

Esto seria entonces lo que tenia planeado hacer para este proyecto.

