% PAQUETES NECESARIOS
\usepackage[labelsep=period]{caption}
\usepackage{subcaption}
\usepackage{comment}
\usepackage{float}
\usepackage{graphicx}
\usepackage{cancel}
\usepackage[utf8]{inputenc}                 % para poder esribir con acentos
\usepackage[T1]{fontenc}                    % encoding T1 para fonts
\usepackage[english,spanish,es-nodecimaldot]{babel}         % multilenguaje

\usepackage{graphicx}                       % inclusion de figuras 
\usepackage{amsthm, amsmath, amssymb}       % fonts y environments para math
\usepackage{setspace}\onehalfspacing        % espaciado entre lineas
\usepackage[loose,nice]{units}              % units in upright fractions
\usepackage[usarlogouba]{DF-MSc-titlepage}               % titlepage al estilo df.uba.ar
%\usepackage{indentfirst}                    % indentar el inicio de seccion
\usepackage{lipsum}                         % para generar texto generico
%\usepackage{aas_macros}                     % macros para nombre de journals

\usepackage{empheq}                                 %para hacer cajas de texto con color
\usepackage[most]{tcolorbox}

\newtcbox{\mymath}[1][]{%
    nobeforeafter, math upper, tcbox raise base,
    enhanced, colframe=blue!30!black,
    colback=blue!30, boxrule=1pt,
   #1}

\newcommand{\boxedeq}[2]{\begin{empheq}[box={\fboxsep=6pt\fbox}]{equation}\label{#1}#2\end{empheq}}
\newcommand{\coloredeq}[2]{\begin{empheq}[box={\mymath[colback=red!50!black!50!white!50]}]{equation}\label{#1}#2\end{empheq}}


\usepackage{bookmark}                       % para que genere bookmarks en el pdf
\usepackage{fancyhdr}                       % para los headers & footers
\usepackage{emptypage}                      % saca headers and footers de paginas en blanco 
\usepackage{color}
\usepackage{ulem}
\usepackage[margin=1in]{geometry}           % geometria de la pagina 
\usepackage{physics}                        % brakets and such
% CUSTOMIZACIONES PROPIAS
\graphicspath{{./figuras/}}                 % define el directorio de figuras
%\usepackage[Conny]{fncychap}                % para definir estilos de capitulos
%\renewcommand{\vec}[1]{\mathbf{#1}} 	    % vectores como bold
\geometry{bindingoffset=1cm}                % espacio en el borde interno para el encuadernado
\geometry{textwidth=390pt}                  % cuerpo del texto fijo en 390pt
\addto\captionsspanish{\renewcommand{\listtablename}{Índice de tablas}}
\usepackage[svgnames]{xcolor}
\usepackage{hyperref}                       % para vinculos en el pdf
\usepackage{cancel}
\usepackage{bm}
\hypersetup{
    colorlinks=true,
    linkcolor=blue,
    filecolor=magenta,      
    urlcolor=cyan,
    citecolor=Green,
    hyperindex=true,
    pdfauthor={Camila Cristiano},
    pdftitle={Tesis de Licenciatura},
    }

\makeatletter
\def\thickhrulefill{\leavevmode \leaders \hrule height 1ex \hfill \kern \z@}
\def\@makechapterhead#1{%
  %\vspace*{50\p@}%
  \vspace*{10\p@}%
  {\parindent \z@ \centering \reset@font
        \thickhrulefill\quad
        \scshape \@chapapp{} \thechapter
        \quad \thickhrulefill
        \par\nobreak
        \vspace*{10\p@}%
        \interlinepenalty\@M
        \hrule
        \vspace*{10\p@}%
        \Huge \bfseries #1\par\nobreak
        \par
        \vspace*{10\p@}%
        \hrule
    \vskip 40\p@
    %\vskip 100\p@
  }}
\def\@makeschapterhead#1{%
  %\vspace*{50\p@}%
  \vspace*{10\p@}%
  {\parindent \z@ \centering \reset@font
        \thickhrulefill
        \par\nobreak
        \vspace*{10\p@}%
        \interlinepenalty\@M
        \hrule
        \vspace*{10\p@}%
        \Huge \bfseries #1\par\nobreak
        \par
        \vspace*{10\p@}%
        \hrule
    \vskip 40\p@
    %\vskip 100\p@
  }}
\DeclareMathOperator*{\mcm}{mcm}