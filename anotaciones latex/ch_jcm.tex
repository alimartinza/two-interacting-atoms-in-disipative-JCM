\chapter{Fase geométrica en JCM no lineal}
\label{ch:jcm}


%CAMBIAR ESTO PARA PERSONALIZARLO A MI GUSTO
\pagestyle{fancy}
\fancyhf{}
\fancyhead[LE]{\nouppercase{\rightmark\hfill}}
\fancyhead[RO]{\nouppercase{\leftmark\hfill}}
\fancyfoot[LE,RO]{\hfill\thepage\hfill}


\section{Solucion unitaria y autoenergias}

Para generalizar el resultado anterior, se le agrega un medio no lineal a la cavidad. Este medio se lo conoce como medio Kerr, y agrega un término en el Hamiltoniano de la cavidad Ec. (\ref{ec3:hamiltoniano inicial}):
\begin{equation}
    \hat H_C=\epsilon \hat n \rightarrow \hat H_C^{\text{Kerr}}=\epsilon \hat n (1+\frac{\chi}{\epsilon}\hat n),
\end{equation}
donde $\chi$ es el parámetro que caracteriza al medio de la cavidad Kerr. Este medio hace que la energía de la cavidad dependa de la cantidad de fotones de manera no lineal, donde $\chi$ es la frecuencia/energía característica de la dependencia cuadrática. Si se realiza nuevamente la RWA, se arriba a las mismas conclusiones sobre la forma del Hamiltoniano de la Ec. (\ref{ec3:hamiltoniano jcm}) y se obtiene
\begin{equation}
    \hat H=\frac{\Delta}{2}\hat \sigma_z+\chi \hat n^2+g(\hat a^\dagger \hat \sigma_-+\hat a \hat \sigma_+),
\end{equation}
y en forma matricial, en el subespacio de n excitaciones $\{\ket{gn},\ket{e,n-1}\}$:
\begin{equation}
    H^{(n)} = \begin{pmatrix}
        -\frac{\Delta}{2}+\chi n^2 & g \sqrt{n} \\
        g \sqrt{n} & \frac{\Delta}{2}+\chi (n-1)^2 
    \end{pmatrix}.
\end{equation}
Resolviendo, se obtiene que ahora los autovectores son
\begin{equation}\label{ec3:autoestado kerr}
    \ket{\psi_\pm^n}=\frac{1}{N_\pm} \left( (-\frac{\Delta}{2}+\chi(n-1/2)\mp\frac{\Omega_{n,\chi}}{2}) \ket{gn} + g\sqrt{n} \ket{e,n-1}  \right),
\end{equation}
donde $\Omega_{n,\chi}=\sqrt{(\chi(2n-1)-\Delta)^2+4g^2n}$, $N_\pm=\sqrt{(-\frac{\Delta}{2}+\chi(n-1/2)\mp\Omega_{n,\chi}/2)^2+g^2n}$, y las autoenergías son 
\begin{equation}\label{ec3:autoenergia kerr}
    E_\pm^n=\chi(n-\frac{1}{2})^2 +\frac{\chi}{4} \pm \frac{\Omega_{n,\chi}}{2}
\end{equation}
Se puede ver que el resultado con $\chi=0$ se reduce al caso visto anteriormente, que representa una cavidad con un medio lineal.

Si se quiere resolver la dinámica de este problema para un estado inicial cualquiera, lo que se tiene que hacer es desarrollar este estado inicial en función de los autoestados del problema. Entonces se tendría para un estado arbitrario con un número total de excitaciones definido, que suponemos igual a 1 por simplificación (la generalización es inmediata):
\begin{equation}
    \ket{\psi}(t)=U(t)(\braket{\psi_+^1}{\psi(0)}\ket{\psi_+^1}+\braket{\psi_-^1}{\psi(0)}\ket{\psi_-^1})=c_+e^{-iE_+t}\ket{\psi_+}+c_-e^{-iE_-t}\ket{\psi_-}.
\end{equation}
Resulta interesante notar que al sacar factor común la fase $e^{-iE_+t}$, la cantidad relevante sigue siendo $\Omega_{n,\chi}$, que es la frecuencia de Rabi para medios tipo Kerr. Por otro lado, el producto interno que da lugar a los coeficientes $c_\pm$ depende de $\chi$, por lo tanto las amplitudes de probabilidad de encontrar al estado temporalmente evolucionado en algún otro estado haciendo una medición proyectiva, depende de $\chi$. Concluyendo que el medio Kerr modifica las amplitudes de oscilación de las poblaciones del estado.

Entonces se analiza la relación entre $\pm \Omega_{n,\chi}/2$ y el detunning, teniendo en cuenta que ahora el medio puede tener $\chi \neq 0$. 

\begin{figure}[h]
    \centering
    \begin{subfigure}[h]{0.49\textwidth}
        \centering
        \includegraphics[width=\textwidth]{figuras/ch3/relacion energia detunning jcm simple kerr.png}
        \caption{}
        \label{fig3:relacion energia detunning kerr 1}
    \end{subfigure}
    \hfill
    \begin{subfigure}[h]{0.49\textwidth}
        \centering
        \includegraphics[width=\textwidth]{figuras/ch3/relacion energia detunning jcm simple kerr 2.png}
        \caption{}
        \label{fig3:relacion energia detunning kerr 2}
    \end{subfigure}
    \caption{Gráfico de la frecuencia de Rabi $\Omega_{N,\chi}$ en función del detunning $\Delta$ para (\ref{sub@fig3:relacion energia detunning kerr 1}) $N=1$ y (\ref{sub@fig3:relacion energia detunning kerr 2}) $N=2$.}
    \label{fig3:relacion energia detunning kerr}
\end{figure}

Se observa en la Figura \ref{fig3:relacion energia detunning kerr} como se modifica la frecuencia $\pm \Omega(N,\chi)$ para diferentes valores de $\chi$, donde en el panel (\ref{sub@fig3:relacion energia detunning kerr 1}) se muestran las energías para $N=1$ y en (\ref{sub@fig3:relacion energia detunning kerr 2}) para $N=2$, en función del detunning. En colores (de más oscuro a más claro) se aprecia cómo el aumento de $\chi \in [0,2g]$ afecta a las curvas: al aumentar $\chi$, las curvas se desplazan hacia la derecha en una cantidad $\chi(n-1/2)$. 

Este comportamiento se puede predecir mirando la forma de la autoenergía Ec. (\ref{ec3:autoenergia kerr}), ya que lo que estamos haciendo es desplazando la raíz mediante un cambio de variables $\Delta \rightarrow \Delta - \chi(2n-1)$. Este desplazamiento depende del número de excitaciones $n$. 
Dados dos valores diferentes de $\chi$ es de interés saber si al aumentar $\Delta$, aumentan o disminuyen las energías de los estados, por ejemplo, si se comparan dos casos, uno con $\chi_1=0$ y otro con $\chi_2=0.5g$, y dado un valor de detunning $\Delta=2g$, cuál de los dos casos tiene una mayor frecuencia?

Para esto se busca la intersección entre dos curvas con diferentes $\chi$, que llamamos $\chi_1$ y $\chi_2$, con $\chi_1<\chi_2$. Mediante un cálculo se obtiene que la intersección es para $\Delta=(2n-1)\frac{\chi_1+\chi_2}{2}$. Es decir, si $\Delta<(2n-1)\frac{\chi_1+\chi_2}{2}$ entonces la frecuencia de $\chi_2$ es mayor que la de $\chi_1$, y viceversa si $\Delta>(2n-1)\frac{\chi_1+\chi_2}{2}$.

Ahora, habiendo entendido esto, podemos ver que el efecto del medio es modificar la frecuencia y también la amplitud de la oscilación. En ambos casos, el mínimo de frecuencia y máximo de amplitud se alcanza cuando se cumple que 
\begin{equation}
    \Delta-\chi(2n-1)=0.
\end{equation}
Esta condición se da en el mínimo de energía, y por lo tanto es inmediato concluir que la frecuencia es mínima es este lugar. Además, cuando se cumple esta condición, los autoestados son autoestados de $\hat \sigma_x$, y los estados de la base evolucionan según
\begin{equation}
    \begin{aligned}
        \ket{gn(t)} &=\left[ A_g \left( \frac{e^{-iE_+t}}{N_+^2}+\frac{e^{-iE_-t}}{N_-^2}\right)-B_g\left( \frac{e^{-iE_+t}}{N_+^2}-\frac{e^{-iE_-t}}{N_-^2}\right) \right] \ket{gn} \\
        & + g\sqrt{n}\left[C_g\left( \frac{e^{-iE_+t}}{N_+^2}+\frac{e^{-iE_-t}}{N_-^2}\right)-D_g\left( \frac{e^{-iE_+t}}{N_+^2}-\frac{e^{-iE_-t}}{N_-^2}\right)\right]\ket{e,n-1},
    \end{aligned}
\end{equation}
con
\begin{equation*}
    \begin{aligned}
        A_g &= (-\Delta/2+\chi(n-1/2))^2+\Omega_{n,\chi}^2/4, \\
        B_g &= \Omega_{n,\chi}(-\Delta/2+\chi(n-1/2)), \\
        C_g & = g\sqrt{n}(-\Delta/2+\chi(n-1/2)), \\
        D_g & = g\sqrt{n}\Omega_{n,\chi}. 
    \end{aligned}
\end{equation*}
Para que las oscilaciones sean coherentes, se puede ver que cuando $\Delta-\chi(2n-1)=0$ las constantes $B_g=C_g=0$, $N_+=N_-=\sqrt{2}g\sqrt{n}$, y así el estado evoluciona como:
\begin{equation}
    \ket{gn(t)}=e^{-i(\chi(n-1/2)^2+\chi/4)}\left[\cos(\Omega t/2)\ket{gn}-\sin(\Omega t/2)\ket{e,n-1}\right],
\end{equation}
y la evolución de $\ket{e,n-1(t)}$ debe ser el estado ortogonal. Si no se cumple la condición, entonces las oscilaciones no serán coherentes y su amplitud no será 1.

Una propiedad importante que se observa cuando se cumple la condición $\Delta-\chi(2n-1)=0$, es que el autoestado del Hamiltoniano es máximamente entrelazado (MES por sus siglas en inglés). Para ver esto se parte de la Ec. (\ref{ec3:autoestado kerr}) y la expresión para $\Omega_{n,\chi}$. Por un lado, es fácil ver que $\Omega_{n,\chi}=2g\sqrt{n}$ y se obtiene 
\begin{equation}
    \ket{\psi_\pm^n}=\frac{1}{\sqrt{2}}(\ket{gn}\mp \ket{e,n-1}),
    \label{ec3:bell}
\end{equation}
que son estados de Bell. 

\section{Modelo de JC disipativo} 
\label{sec3:jcm disipativo}

Después del análisis de la fase geométrica acumulada por el sistema átomo-cavidad en la situación ideal de completo aislamiento, se aborda ahora el estudio para el escenario más realista en el que el mismo sistema se encuentra en interacción con un entorno, es decir, un modelo de JC disipativo.

Siguiendo la Ref. \cite{TesisViotti}, en esta sección se estudia en detalle la fase geométrica acumulada en un modelo de Jaynes-Cummings disipativo, como caso paradigmático dentro del campo de la electrodinámica en cavidades. Se considera que los principales mecanismos por los cuales el sistema “átomo + modo” interactúa con el entorno son el flujo de fotones a través de las paredes de la cavidad, y el continuo e incoherente bombeo del sistema de dos niveles, lo que conforma un escenario frecuente en electrodinámica de cavidades semiconductoras \cite{Khitrova2006}-\cite{DelValle2009}. 

Para poder modelar estos mecanismos, se emplea la ecuación maestra fenomenológica de Lindblad (ver Ap. \ref{ap_ecsmaestras})
\begin{equation}\label{ec3:lindblad}
\dot{\rho}(t) = -i [H, \rho(t)] + \frac{1}{2} \sum_\alpha \big( 2L_\alpha \rho(t) L_\alpha^{\dagger} - \{ L_\alpha^{\dagger}L_\alpha, \rho(t) \} \big),
\end{equation}
despreciando otros procesos con menor influencia en la dinámica como el desfasaje puro o el bombeo de fotones del entorno en la cavidad, donde además se ha considerado que el entorno se halla a temperatura cero. Los operadores de Lindblad

\begin{equation}
L_\gamma = \sqrt{\gamma} \ a,
\end{equation}
\begin{equation}
L_p = \sqrt{p} \ \sigma_+,
\end{equation}
representan la pérdida de fotones y el bombeo continuo e incoherente del átomo, respectivamente, con los parámetros $\gamma$ y $p$ denominados tasa de pérdida de fotones y amplitud del bombeo. 

El bombeo sobre el átomo es siempre secundario frente a la pérdida de fotones, lo cual nos da las relaciones $\frac{p}{g},\frac{p}{\gamma} \ll 1$, y l
\subsection{Solución y régimen de acoplamiento}\label{sec3:regimen acoplamiento}
En esta ocasión se resuelve el problema restringiéndose al subespacio donde el átomo puede estar en cualquiera de sus dos estados, y la cavidad tiene 1 o 2 fotones. En consecuencia, el problema se limita a un subespacio truncado cuya base son los estados $\{ \ket{0}=\ket{g,0} ; \ket{1}=\ket{e,0} ; \ket{2}=\ket{g,1} \}$. Desarrollar explícitamente el sistema de ecuaciones dadas por la ecuación de Lindblad Ec. (\ref{ec3:lindblad}), lleva a ecuaciones para los elementos $\rho_{0i}$ que quedan desacoplados de los demás:

\begin{equation}
    \begin{aligned}
        \dot \rho_{01} & =-\frac{p}{2} \rho_{01}+i\Delta\rho_{01}+ig\rho_{02} \\
        \dot \rho_{02} & =-\frac{p}{2} \rho_{02}-\gamma \rho_{02}+ig\rho_{01}
    \end{aligned}
\end{equation}
con lo cual, si inicialmente los elementos de matriz $\rho_{0i}(0)=0$, permanecerán así durante toda la evolución del sistema. Para hacer una analogía y comparar con el caso unitario, se estudia la condición inicial $\rho(0)=\ketbra{e,0}{e,0}$, de modo que la matriz $\rho(t)$ exhiba una estructura diagonal por bloques. El primer bloque de 1x1 representando al estado $\ket{0}$, y luego un bloque de 2x2 que describe la dinámica entre los estados $\ket{1}$ y $\ket{2}$. Las ecuaciones son

\begin{equation}
\begin{aligned}
\dot{\rho}_{00} &= -p \rho_{00} + \gamma \rho_{22} \\
\dot{\rho}_{11} &= -i g (\rho_{21} - \rho_{12}) + p \rho_{00} \\
\dot{\rho}_{22} &= -i g (\rho_{12} - \rho_{21}) - \gamma \rho_{22} \\
\dot{\rho}_{12} &= -i g (\rho_{22} - \rho_{11}) - i \Delta \rho_{12} - \frac{\gamma}{2} \rho_{12},
\end{aligned}
\end{equation}
se resuelven numéricamente para acceder al estado $\rho(t)$ a tiempo $t>0$. 

\begin{figure}[h]
    \centering
    \includegraphics[width=\textwidth]{figuras/ch3/poblaciones sc vs wc.png}
    \caption{Solución numérica al sistema de ecuaciones dada por la ecuación de Lindblad para el estado inicial $\ket{e0}\bra{e0}$. Estos gráficos se realizaron con $\Delta=2g$; a la izquierda se observa el regimiento de WC con $\gamma=0.1g$, donde el sistema átomo-cavidad está débilmente acoplado con el entorno, y a la derecha el de SC con $\gamma=2g$, donde las poblaciones y coherencias decaen sin oscilar. Las lineas sólidas son las poblaciones de los estados, en azul para el estado $\ket{g0}$, en morado para $\ket{e0}$ y en amarillo $\ket{g1}$, y la línea rayada representa la coherencia entre los estados con N=1 ($\ket{e0}$ y $\ket{g1}$). }
    \label{fig3:poblaciones e0}
\end{figure}

En la Figura \ref{fig3:poblaciones e0}a, se muestra el régimen de SC, donde el acoplamiento entre el átomo y la cavidad es mayor al acoplamiento con el entorno, según la relación entre los parámetros $\gamma/g=0.1$, y en el panel \ref{fig3:poblaciones e0}b, se muestra el caso del WK. En el primer caso, tanto las poblaciones como las coherencias presentan oscilaciones coherentes antes de decaer por la influencia del entorno. En cambio, para el caso de WK, estas oscilaciones coherentes no están presentes y el sistema llega a su estado asintótico en tiempos muy cortos. Las características de la dinámica de cada régimen, influyen profundamente en el estudio de la fase geométrica, haciendo posible únicamente su utilización en el caso de Strong Coupling, donde la dinámica presenta oscilaciones coherentes durante varios ciclos, antes de decaer, haciendo del régimen de SC el único escenario conveniente para su estudio. Antes de fundamentar esta afirmación, se realiza un estudio poblacional en el caso de una cavidad con medio Kerr.
\begin{figure}[h]
    \centering
    \includegraphics[width=\textwidth]{figuras/ch3/poblaciones kerr.png}
    \caption{Análisis poblacional para una cavidad con medio Kerr con $\chi=0.5g$. Para la simulación se utilizaron los mismos parámetros que en la Figura \ref{fig3:poblaciones e0}}
    \label{fig3:poblaciones kerr}
\end{figure}
Como se vio anteriormente, el efecto del medio Kerr sobre los autoestados y las autoenergías es, por un lado, desplazar los niveles de energía.
Si se comparan las Figuras \ref{fig3:poblaciones e0} con \ref{fig3:poblaciones kerr}, entonces se pueden observar dos diferencias. La primera es lo mencionado anteriormente; en el mismo tiempo, es decir entre $0\leq t/T \leq 25$, en el caso de $\chi=0$ se observan 25 oscilaciones, pero en el caso de $\chi=0.5g$ solo se observan 23 oscilaciones. Esto se debe a la condición que se encontró al final de la sección \ref{sec3:medio kerr}. En este caso se verifica que $\Delta=2g>(2n-1)\frac{\chi_1+\chi_2}{2}=\frac{g}{4}$, entonces la diferencia de energías entre los autoestados disminuye al aumentar $\chi$, y por lo tanto, las oscilaciones son más lentas para el caso de $\chi=0.5g$ en comparación con $\chi=0$.

a relación entre $\gamma$ y $g$ da lugar a dos regímenes que se diferencian con claridad \cite{Carmi1989}-\cite{Lodhal2015}. El régimen de acoplamiento fuerte (SC o Strong Coupling) aparece cuando la interacción átomo-cavidad es más fuerte que la disipación del entorno, es decir $\gamma /g <1$. En el caso contrario $\gamma/g>1$ estamos en el régimen de acoplamiento débil (WC o Weak Coupling). Para no generar confusiones, hay que destacar que, en general, cuando en la literatura se habla de acoplamientos fuertes y débiles, se refiere a la interacción entre las partes del mismo sistema, pero en este caso, se esta haciendo referencia a la interacción del sistema con el entorno \textit{en comparación} con la interacción interna del sistema.
\subsection{Estados estacionarios}
Si bien el bombeo es secundario, este tiene un efecto perturbador, ya que estamos pensando que este efecto ocurre aun cuanto $T=0$, para ver que es lo que pasa analicemos el estado estacionario pidiendo que $\dot\rho_{ij}=0$, y obtenemos que 
\begin{align}
    \rho_{00}&=\frac{\gamma}{p}\rho_{22} \\
    \rho_{22}&=\frac{-ig}{\gamma}\Re(\rho_{12}) \implies \rho_{22}=\Re(\rho_12)=0\\
    \Re\rho_{12}&=\frac{2\Delta}{\gamma}\Im\rho_{12} \implies \Im\rho_{12}=0 \\
    &\rho_{00}+\rho_{11}+\rho_{22}=1 \implies \rho_{11}=1 
\end{align}

Esto nos dice que el pumping nos rompe todo, ya que el estado fundamental no es el que uno espera. Esto desaparece si proponemos que $p=0$, algo que es facil de ver mirando las ecuaciones diferenciales.

Este es el efecto que para mi hace que la fase geometrica no sea robusta cuando tenemos pumping y la condicion inicial no esta sobre el circulo maximo.

Cuando $p=0$
\subsection{Fase geométrica en presencia de disipación}
\label{sec3:fg disipacion}
En esta sección se muestran algunos resultados conocidos acerca de la fase geométrica adquirida por el sistema, calculada siguiendo la definición de la Ec. (\ref{ec2:fg general}), y se analiza como esta se ve modificada con respecto del valor unitario por efecto del contacto con el entorno. Como el estado inicial es puro, la definición se reduce al caso particular descrito por la Ec. (\ref{ec2:fg general puro}).

Los autovalores y autovectores del operador densidad pueden escribirse formalmente  diagonalizando el subespacio de 2x2 de la matriz densidad:
\begin{equation}
    \rho(t)=\begin{pmatrix}
        \rho_{00} & 0 & 0 & 0 &\dots \\
        0 & \rho_{11} & \rho_{12} & 0 & \dots \\
        0 & \rho_{21} & \rho_{22} & 0 & \dots \\ 
        \vdots & 0 & 0 & \ddots & \dots 
    \end{pmatrix}
\end{equation}
donde se asume nuevamente una estructura diagonal por bloques, que se da cuando el estado inicial tiene un número definido de excitaciones, dando lugar a dos autovectores. De estos dos, es el de mayor energía el de interés para utilizar la definición de la fase geométrica Ec. (\ref{ec2:fg general puro})
\begin{equation}
    \ket{\psi_+}(t)=\frac{-(\rho_{22}-\epsilon_+)\ket{e,0}+\rho_{21}\ket{g,1}}{\left((\rho_{22}-\epsilon_+)^2+\rho_{21}\rho_{12} \right)^{1/2}},
\end{equation}
con $\epsilon_+=\frac{1}{2}(\rho_{11}+\rho_{22}+\sqrt{(\rho_{11}-\rho_{22})^2+4\rho_{12}\rho_{21}})$ el autovalor asociado. Recurriendo a este resultado, podemos escribir formalmente la fase geométrica en función de los elementos de matriz $\rho_{ij}(t)$ \cite{Viotti2022}:
\begin{equation}
    \phi_g(t)=\int_0^t dt' \frac{\Im \dot\rho_{21}\rho_{12}}{(\rho_{22}-\epsilon_+)^2+\rho_{12}\rho_{21}}.
\end{equation}

En general, esta fase diferirá de aquella acumulada en una evolución unitaria de forma que puede
escribirse, sin pérdida de generalidad, como $\phi_g=\phi_u+\delta\phi$, con $\delta\phi$ la diferencia entre la fase unitaria y
aquella modificada por la presencia del entorno. Caracterizar la corrección $\delta\phi$ permite relacionar este
objeto, perteneciente a la geometría misma del espacio de Hilbert, con los efectos de disipación y
decoherencia experimentados por el sistema, así como determinar bajo qué circunstancias $\delta\phi$ resulta
despreciable y se puede considerar que la fase geométrica es robusta (o no) al efecto del entorno.

\textcolor{red}{Pregunta: Aca no podria pasar que el $\epsilon_+$ que por definicion es el autovalor mas grande, no coincida con el estado que estamos mirando? Esta claro que inicialmente el $\epsilon_+$ coincide con el autoestado puntero, pero cuando este autovalor decae entonces podria pasar que eventualmente haya que mirar el $\epsilon_-$. Esto quiere decir que quizas esta formula esta mal cuando el sistema es abierto.}
\section{Vuelta atras al JCM de ludmila}

Escribiendo y revisando cosas, me encontre con algunos fenomenos interesantes aun en el caso de 1 atomo, el JCM de ludmi. Esto surgio basicamente de intentar de usar lo de la seccion anterior para ver si la trayectoria en el caso de 2 atomos seguia una linea geodesica. Para visualizarlo encontre lo anterior, pero me parece que no da mucha informacion, y despues tambien encontre una medida qe nos dice que tan lejos estamos de una linea geodesica, esto lo hace mediante la distancia de Wooters y utilizando el hecho de que durante una evolucion la cantidad $\Delta E \sim \bra{\psi}H^2\ket{\psi}-\bra{\psi}H\ket{\psi}^2$ no depende de la fase global (en realidad es un poco mas sofisticado el argumento pero mas o menos va por aca) y por lo tanto es un invariante sobre el espacio proyectivo o algo asi. Ahora que lo intento de escribir veo que no lo recuerdo con demasiado detalle, pero usando esto podemos decidir si la linea es una geodesica o no. Lo intente y me parecia un poco raro lo que me estaba dando asi que volvli para el caso de 1 atomo a ver como era. Mirando algunas condiciones inicales mas raras, y mirando los autoestados correspondientes a la matriz densidad en sistemas abiertos (autoestado que acumula la FG), me encontre algunas situaciones inesperadas.

Lo primero que busque fue ver que pasaba si la trayectoria era en una geodesica, pero no de circulo maximo, sino una geodesica mas pequeña. Para esto saque la condicion inicial del circulo maximo y se vio lo siguiente: aun en el caso resonante tenemos diferencias entre la FG (pero para tiempos muy largos), y el comportamiento del autoestado puntero es algo que no me esperaba. 

\begin{figure}
    \centering
    \includegraphics[width=0.5\linewidth]{ch3/jcm1 fg ci modificada 1.png}
    \caption{condicion inicial $\ket{e0}+\sqrt{2}\ket{g1}$, parametros $\Delta=0,\chi=0$. (a) grafico de autoenergias (para el caso disipativo) (b) es un zoom de (a). (c) FG comparando evolucion unitaria (azul) y disipativa (perdidas $\gamma=0.1g$ y dephasing $p=0.05\gamma$). (d) es un zoom de (c).}
    \label{fig3:jcm1-fg-ci1}
\end{figure}
\begin{figure}
    \centering
    \includegraphics[width=0.5\linewidth]{ch3/jcm1 bloch ci modificada 1.png}
    \caption{condicion inicial $\ket{e0}+\sqrt{2}\ket{g1}$, parametros $\Delta=0,\chi=0$. Trayectoria en esfera de Bloch. En negro se ve el estado dado por la evolucion unitaria, que parece darse sobre una geodesica. En amarillo se ve la evolucion del vector de bloch que representa la matriz densidad para el caso disipativo, vemos que pierde pureza lo cual interpretamos como probabilidad que va al estado fundamental del sistema a $T=0$ que es $\ket{g0}$, y en verde/celeste vemos la trayectoria del autoestado puntero que se corresponde con la evolucion con perdidas y dephasing.}
    \label{fig3:jcm1-bloch-ci1}
\end{figure}
Como vemos en la figura \ref{fig3:jcm1-bloch-ci1}, aun cuando la trayectoria es por una geodesica (o aunque sea es lo que parece ya que parece paralela al circulo maximo), el autoestado correspondiente se chanflea hacia el costado. Hay dos cosas mas que me parecen interesantes. Por un lado, lo que vemos es que la FG en la fig \ref{fig3:jcm1-fg-ci1} parece que es robusta aun cuando no esta pegando los saltos escalonados que yo habia ya relacionado con la robustez, pero aun asi en realidad vemos que se esta alejando ya que cuando alargamos el tiempo de simulacion ambas curvas se alejan. La otra cosa que me parecio interesante es que, si pensamos la FG en terminos de la interpretacion de AA del angulo subtendido por la trayectoria en el espacio de estados, entonces lo que yo veo en este caso es que el vector de bloch de la matriz densidad (amarillo) subtiende el mismo angulo que la trayectoria unitaria (negra). En este sentido, aunque quizas esta mal, tiene para mi incluso mas sentido esto que el autoestado puntero (le llamo autoestado puntero al autoestado que inicialmente tiene prob 1). 

Como explicamos el chanfleo del autoestado puntero? Si sacamos el dephasing obtenemos lo siguiente.


\begin{figure}
    \centering
    \includegraphics[width=0.5\linewidth]{ch3/jcm1 fg ci modificada 1 no dephasing.png}
    \caption{condicion inicial $\ket{e0}+\sqrt{2}\ket{g1}$, parametros $\Delta=0,\chi=0$. (a) grafico de autoenergias (para el caso disipativo) (b) es un zoom de (a). (c) FG comparando evolucion unitaria (azul) y disipativa (perdidas $\gamma=0.1g$ y dephasing $p=0.05\gamma$). (d) es un zoom de (c).}
    \label{fig3:jcm1-fg-ci1-no-dephasing}
\end{figure}
\begin{figure}
    \centering
    \includegraphics[width=0.5\linewidth]{ch3/jcm1 bloch ci modificada 1 no dephasing.png}
    \caption{condicion inicial $\ket{e0}+\sqrt{2}\ket{g1}$, parametros $\Delta=0,\chi=0$. Trayectoria en esfera de Bloch. En negro se ve el estado dado por la evolucion unitaria, que parece darse sobre una geodesica. En amarillo se ve la evolucion del vector de bloch que representa la matriz densidad para el caso disipativo, vemos que pierde pureza lo cual interpretamos como probabilidad que va al estado fundamental del sistema a $T=0$ que es $\ket{g0}$, y en verde/celeste vemos la trayectoria del autoestado puntero que se corresponde con la evolucion con perdidas y dephasing.}
    \label{fig3:jcm1-bloch-ci1-no-dephasing}
\end{figure}

Ahora no solo no se chanflea el autoestado puntero sino que tambien parece arreglarse la no robustez. No analice bien todos los casos y todo, pero parece que el dephasing es el que rompe la FG, y si lo pensamos mal y rapido lo que parece hacer el dephasing en la esfera de bloch es rotar el autoestado puntero (o algo asi) hacia el plano z-y, pero si la condicion inicial esta sobre este plano, entonces no tiene efecto? Por ejemplo la condicion inicial $\frac{1}{\sqrt{2}}(\ket{g1}+i\ket{e0})$ sigue la siguiente trayectoria en resonancia

\begin{figure}
    \centering
    \includegraphics[width=0.5\linewidth]{ch3/jcm1 bloch ci modificada 2.png}
    \caption{trayectoria de $\frac{1}{\sqrt{2}}(\ket{g1}+i\ket{e0})$ en resonancia}
    \label{fig3:jcm1-bloch-ci2}
\end{figure}

En este caso las FG si es escalonada. La condicion inicial $\frac{1}{\sqrt{2}}(\ket{g1}+\ket{e0})$ no evoluciona ya que es un autoestado, y por lo tanto no acumula fase geometrica. Por otro lado, su contraparte disipativa si acumula (poco pero acumula), porque los rates de perdida de fotones son diferentes para cada estado de la superposicion, entonces me parece que hace un pequeño circulo. 

Por completitud pongo una esfera de bloch fuera de resonancia para que se note la diferencia con estos casos y un poco de interpretacion de la diferencia con los angulos subtendidos


\begin{figure}
\centering
\begin{subfigure}{0.4\textwidth}
    \includegraphics[width=\textwidth]{ch3/jcm1 bloch no resonancia.png}
    \caption{Trayectoria unitaria (negro) y autoestado puntero (azul)}
    \label{fig3:jcm1-no-resonancia-1}
\end{subfigure}
\hfill
\begin{subfigure}{0.4\textwidth}
    \includegraphics[width=\textwidth]{ch3/jcm1 bloch no resonancia 2.png}
    \caption{Trayectoria unitaria (negro) y vector disipativo (amarillo)}
    \label{fig3:jcm1-no-resonancia-2}
\end{subfigure}
        
\caption{Esfera de Bloch fuera de resonancia. $\frac{1}{\sqrt{2}}(\ket{g1}+\ket{e0})$, $\Delta=0,\chi=g$}
\label{fig3:jcm-bloch-no-resonancia}
\end{figure}

Todo esto me hace pensar que todavia no entiendo bien el sistema de 1 atomo y la FG esta muy interesante porque hay un monton de factores que entran en juego.

\section{Entrelazamiento en JCM}

Ahora queria mirar un poco como es el entrelazamiento. Lo primero que hice fue agarrar el caso resonante, y calcular el entrelazamiento durante la evolucion. Como en el caso resonante las evoluciones son por curvas geodesicas y rotan al rededor del eje $x$ de la esfera de Bloch, lo que entonces nos alcanza es con mirar la condicion inicial
\begin{equation}
    \ket{\psi}=\cos(\frac{\theta}{2})\ket{e0}+\sin(\frac{\theta}{2})\ket{g1}
\end{equation}
e ir variando el angulo $\theta$. Esto nos dice como evoluciona el sistema en el caso resonante para practicamente todos los casos, ya que si le agregamos una fase compleja a la condicion inicial, en resumen lo que estamos haciendo es elegir otro origen temporal, ya que lo que termina haciendo esta cosa es agregarnos una fase compleja que va como 

