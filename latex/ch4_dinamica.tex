\chapter{Jaynes-Cummings de dos átomos, no lineal, medio Kerr}
\label{ch4_dinamica}

%CAMBIAR ESTO PARA PERSONALIZARLO A MI GUSTO
\pagestyle{fancy}
\fancyhf{}
\fancyhead[LE]{\nouppercase{\rightmark\hfill}}
\fancyhead[RO]{\nouppercase{\leftmark\hfill}}
\fancyfoot[LE,RO]{\hfill\thepage\hfill}

En este capitulo se extiende el modelo de Jaynes-Cummings presentado en el capitulo \ref{ch3_jcm}, agregandole nuevas cosas. Lo mas importante es que ahora vamos a tener dos átomos dentro de una misma cavidad. En la literatura en general, el JCM fue extendido para considerar dos cavidades donde cada una tiene su propio átomo, y usando una condición inicial entrelazada se puede hacer interactuar ambas cavidades REFS. El camino que se tomó en este trabajo, es un tanto fuera de lo convensional ya que no hay muchos estudios sobre este sistema. El principal obstaculo que presenta este problema, es que el espacio de Hilbert crece mucho y se torna inmanejable analiticamente; como bien ya sabemos, el JCM tiene subespacios de 2 dimensiones que no se mezclan, y utilizando esta estrategia vamos a ver que en este caso tenemos subespacios de 4x4 que tampoco se mezclan en el caso unitario. Esto nos permite encontrar algunas expresiones analiticas, pero en general se utilizaran métodos númericos para analizar la dinamica. \newline
Este capitulo entonces seguira un hilo conductor, partiendo desde el caso mas sencillo hasta llegar a analizar cuales son los efectos de los diferentes parámetros en el problema. 
	Primero vamos a considerar una cavidad perfecta, es decir sin disipaci\'on, agregandole el segundo átomo, vamos a intentar de entender cual es el efecto de este sobre el modelo de un solo átomo. Para esto haremos un analisis poblacional, y de observables como la entropia reducida, la concurrencia, las matrices de pauli. Una vez agregado el segundo átomo, vamos a prender las interacciones de a una y vamos a analizar cuales son sus efectos. Luego, vamos a comparar esto con el caso en donde la cavidad presenta perdidas. Principalmente, nos centraremos en un analisis del entrelazamiento, ya que esta es la cualidad mas interesante que tenemos en el ambito de la informaci\'on cu\'antica. 

\textcolor{blue}{Luego, se analizar\'a el problema para dos átomos, primero en el caso que estos no interact\'uan
directamente entre si, sino que lo hace indirectamente a travez de la cavidad. La comparativa entre
esta situación y la mas comun, donde los átomos interactuan mediante sus espines o sus momentos
dipolares, es muy rica porque nos permite discernir con claridad cual es el efecto de la cavidad
y cual de la interacción entre los átomos a la hora de entrelazarse e intercambiar energía.
El problema de dos átomos tiene una peculiaridad al elegir las condiciones iniciales, ya que la
dinamica depende de esta eleccion, y hay muchas diferentes configuraciones interesantes, por un lado
por la gran dimension del espacio, y por otro lado, esta la posibilidad de jugar con las simetrias.
Surge asi la pregunta de si es importante, o si tiene sentido, teniendo dos átomos indistinguibles
en una cavidad, que la condición inicial sea asimetrica ante intercambio. 
}

\section{Modelo de dos átomos y solucion unitaria}
En este trabajo, nos vamos a concentrar en una extension del modelo, donde vamos a ubicar dos átomos dentro de la cavidad. Estos átomos pueden interactuar entre si, y con la cavidad, y ademas agregaremos no-linealidades en el acoplamiento y en el medio.
Vamos a usar un modelo de Jaynes-Cummings para describir la interacci\'on entre el campo electromagn\'etico y los átomos. Adem\'as supondremos que el acoplamiento depende de la cantidad de fotones y los átomos podr\'an interactuar entre si mediante un termino tipo Ising y otro tipo dipolo-dipolo. Recordemos que para estamos asumiendo que vale la aproximaci\'on de onda rotante ($\omega_0 \sim \omega$) y $g << \omega,\omega_0$.
Entonces, el Hamiltoniano que describe este problema es el siguiente:

\begin{equation}
\begin{split}
     \hat H & =\underbrace{\hbar \omega_0 h(\hat n) \hat n }_{\hat H_F}+\underbrace{\frac{\hbar \omega}{2}(\hat\sigma_Z^{(1)}+\hat\sigma_Z^{(2)})}_{\hat H_A}   \\ 
     & + \underbrace{\hbar g(\hat\sigma_+^{(1)}\hat a f(\hat n)+\hat\sigma_-^{(1)}f(\hat n) \hat a^\dagger + \hat\sigma_+^{(2)}\hat a f(\hat n)+\hat\sigma_-^{(2)}f(\hat n) \hat a^\dagger)}_{H_{FA}} + \\ & \underbrace{2\hbar \kappa (\hat \sigma_-^{(1)}\hat \sigma_+^{(2)}+\hat \sigma_+^{(1)}\hat \sigma_-^{(2)}) + \hbar J \hat \sigma_Z^{(1)}\hat \sigma_Z^{(2)}}_{H_{AA}}
\end{split}
\end{equation}

donde $\hat a$ es el operador de aniquilaci\'on del fot\'on, $\omega_0$ y $\omega$ son las frecuencias del fot\'on y del átomo respectivamente, $g$ es la constante de acoplamiento, las constantes $J$ y $\kappa$ son los par\'ametros de Ising y de dipolo-dipolo para las interacciones átomo-átomo, y los operadores $\sigma^{(i)}$ son las matrices de Pauli que act\'uan sobre el átomo i-esimo. Finalmente, las funciones $h(\hat n)$ y $f(\hat n)$ son las que van a dar cuenta de la no linealidad dependiente del numero de fotones de la cavidad $\hat n = \hat a^\dagger \hat a$. 

Tomando un medio tipo Kerr la funci\'on $h(\hat n)=1+\frac{\chi}{\omega_0}\hat n$ \cite{}\textcolor{red}{(CITA)}, y la funci\'on $f(\hat n) =1$ si tomamos un acoplamiento lineal, y $f(\hat n) = \sqrt{\hat n}$ si consideramos un acoplamiento tipo Buck-Sukumar \cite{}\textcolor{red}{(CITA)}

En este punto es normal hacer una transformaci\'on unitaria  $K = \exp\left\{-i \omega t (\hat a^\dagger a + \sigma_z/2)\right\}$ para dejar el Hamiltoniano en funci\'on del Detuning $\Delta (\sim 0)$. 

\begin{equation}
\begin{split}
     \hat H_I & =\hbar \chi \hat n^2+\frac{\hbar \Delta}{2}(\hat\sigma_Z^{(1)}+\hat\sigma_Z^{(2)})   \\ 
     & + \hbar g(\hat\sigma_+^{(1)}\hat a f(\hat n)+\hat\sigma_-^{(1)}f(\hat n) \hat a^\dagger + \hat\sigma_+^{(2)}\hat a f(\hat n)+\hat\sigma_-^{(2)}f(\hat n) \hat a^\dagger) \\ 
 & + 2\hbar \kappa (\hat \sigma_-^{(1)}\hat \sigma_+^{(2)}+\hat \sigma_+^{(1)}\hat \sigma_-^{(2)}) + \hbar J \hat \sigma_Z^{(1)}\hat \sigma_Z^{(2)}
\end{split}
\end{equation}\label{eq4:H}
Este es el Hamiltoniano con el que vamos a trabajar, así que a partir de ahora vamos a olvidarnos del subíndice I. Obsérvese que el caso de $\chi=0$ es el caso de un medio lineal. 
\textcolor{blue}{ACA PUEDO AGREGAR UN ESQUEMA DE COMO SERIA.} En este esquema se ve como seria el experimento planteado. 

Este Hamiltoniano se puede resolver analíticamente para el caso de una cavidad sin perdidas. En analogía con el caso de 1 átomo, vamos a elegir la base de N excitaciones, donde esperamos que estos subespacios queden invariantes, es decir, que el Hamiltoniano sea diagonal por bloques, pero como ahora tenemos 2 átomos, tenemos que elegir una base que respete simetrías, para N excitaciones los estados de la base son $\left\{\ket{ggn},\frac{1}{\sqrt{2}}(\ket{eg,n-1}+\ket{ge,n-1}),\ket{ee,n-2},,\frac{1}{\sqrt{2}}(\ket{eg,n-1}-\ket{ge,n-1})\right\}$. En esta base, el Hamiltoniano se diagonaliza por bloques y el bloque de N excitaciones queda
El problema unitario se puede resolver analíticamente. Esto esta hecho en el paper de los autores \textit{O de los Santos-Sánchez
, C González-Gutiérrez and J Récamier}, titulado \textit{Nonlinear Jaynes–Cummings model for two
interacting two-level atoms} \cite{paper:santos}\textcolor{red}{(CITA)}. 
\textcolor{red}{Ac\'a tengo que pasar las cuentas a latex, pero las hice en papel para ver si entend\'ia todo.}
Para resolver el problema lo primero que hacemos es notar que el Hamiltoniano conserva el numero de excitaciones, es decir $[H,\hat N]=0$, y en esta situación es sabido que el Hamiltoniano de JC es diagonal por bloques si elegimos convenientemente la base, esta es la que agrupa los estados con misma cantidad de excitaciones $\hat N = \hat n + \hat \sigma_+^{(1)}\hat \sigma_-^{(1)}+\hat \sigma_+^{(2)}\hat \sigma_-^{(2)}$: 
\begin{equation}
\begin{split}
    & \left\{\ket{\Phi^{(n)}_1}=\ket{ggn},\ket{\Phi^{(n)}_2}=\frac{1}{\sqrt{2}}(\ket{egn-1}+\ket{gen-1}),\ket{\Phi^{(n)}_3}=\ket{een-2},\right. \\
& \left. \ket{\Phi^{(n)}_4}=\frac{1}{\sqrt{2}}(\ket{egn-1}-\ket{gen-1})\right\} 
\end{split}
\label{ec4:base}
\end{equation}
,donde se eligi\'o esta combinaci\'on particular porque el ultimo estado de la base, que es impar ante intercambio, queda desacoplado de los otros, simplificando el problema. Esto se ve al evaluar los elementos de matriz del Hamiltoniano $H_{i,j}=\bra{\Phi_i}\hat H \ket{\Phi_j}$, este queda en bloques, y el subespacio correspondiente a $n$ excitaciones $\hat H^{(n)}$ es una matriz de 4x4
\begin{equation}
    \frac{\hat H^{(n)}}{\hbar}=
    \begin{pmatrix}
     \chi n^2 - \Delta +  J & \sqrt{2} g f(n)\sqrt{n} & 0 & 0 \\
    \sqrt{2} g f(n)\sqrt{n} &  \chi (n-1)^2  -  J + 2 k & \sqrt{2} g f(n-1)\sqrt{n-1} & 0 \\
    0 & \sqrt{2} g f(n-1)\sqrt{n-1} &  \chi (n-2)^2 +  \Delta +  J & 0 \\
    0&0&0& \begin{aligned} 
                 & \chi (n-1)^2  \\ 
                 &-  J - 2 k
        \end{aligned}
    \end{pmatrix}
\end{equation}
Vemos claramente que el estados impar ante intercambio esta aislado, y entonces es autoestado del problema, y por lo tanto evoluciona solo y no se mezcla con los otros estados. Esto nos sirve porque ahora, para terminar de resolver el problema, tenemos que diagonalizar la matriz de 3x3. Cabe aclarar que esta matriz solo es valida para $n\geq 2$, ya que los subespacios con $N=0,1$ no tienen 4 estados. En estos casos la soluci\'on del problema de autovalores es mas sencilla aun, as\'i que solo dejaremos los resultados. A partir de ahora se usar\'a como convención $\hbar=1$. \newline
Para resolver el problema de autovalores de la matriz de 3x3 utilizamos la fórmula de Cardano para conseguir las ra\'ices triples que nos aparecen en el polinomio caracter\'istico, y entonces encontramos que los autovalores son
\begin{equation}
    E_j^{(n)}=-\frac{1}{3}\beta_n+2\sqrt{-Q_n}\cos{\left(\frac{\theta_n+2(j-1)\pi}{3}\right)}
    \label{ec4:autoenergias}
\end{equation}
para $j=1,2,3$, y donde 
\begin{equation}
    \theta_n=\cos^{-1}\left(\frac{R_n}{\sqrt{-Q_n^3}}\right)
\end{equation}
\begin{equation}
    \begin{aligned}
        Q_n & = \frac{3\gamma_n-\beta_n^2}{9} \\
        R_n & = \frac{9\beta_n\gamma_n-27\eta_n-2\beta_n^3}{54} \\
        \beta_n & = - \left( \chi(n^2+(n-1)^2+(n-2)^2)+J+2k\right) \\
        \gamma_n & = (\chi(n-1)^2 - J + 2k)(x(n-2)^2+\chi n^2+2J) \\ 
        & +(\chi (n-2)^2+\Delta+J)(x n^2-\Delta+J)-2g^2(n^{2a}+(n-1)^{2a}) \\ 
        \eta_n &= -(\chi n^2-\Delta+J)(\chi(n-2)^2+\Delta+J)(\chi(n-1)^2-J+2k) \\
        &+2g^2 \left[  \chi(n-2)^2n^{2a}+\chi n^2(n-1)^{2a}+\Delta\left(n^{2a}-(n-1)^{2a}\right) +J(n^{2a}-(n-1)^{2a})\right]
    \end{aligned} 
    \label{ec4:parametros solucion}
\end{equation}
donde $a=\frac{1}{2}$ se corresponde con acoplamiento lineal, es decir, $f(n)=1$, y $a=1$ a Buck-Sukumar $f(n)=\sqrt{n}$. Los autovalores ser\'an reales si $Q_n^3+R_n^2<0$.
Con esto podemos escribir los autovectores:
\begin{equation}
    \begin{split}
        \ket{u_j^{(n)}} &= \frac{1}{N_j^{(n)}} \bigg[ \left((E_j^{(n)} - H_{22}^{(n)})(E_j^{(n)}-H_{33}^{(n)}) - H_{23}^{(n)^2} \right) \ket{\Phi_1^{(n)}} \\ &+ H_{21}^{(n)}(E_j^{(n)}-H_{33}^{(n)})\ket{\Phi_2^{(n)}} + H_{23}^{(n)}H_{12}^{(n)}\ket{\Phi_3^{(n)}}\bigg]
    \end{split}
\end{equation}
Obviamente no nos olvidemos del estado $\ket{\Phi_4^{(n)}}$, que tambi\'en es autoestado, con autovalor $E_4^{(n)}=\chi(n-1)^2-J-2k$.
Para el subespacio de $N=0$ solo tenemos un vector $\ket{\Phi_1^{(0)}}=\ket{gg0}$ y su autovalor es $E_1^{(0)}=-\Delta+J$.
Para $N=1$ tenemos 3 vectores en el subespacio, y las autoenergías son
\begin{align}\label{ec4:energias n1}
    E_{1,2}^{(1)} &=\frac{\chi -\Delta}{2} +k \pm \sqrt{2g^2+(k-J+\frac{\Delta -\chi}{2} )^2}\\
    E_3^{(1)} & = -2k-J      
\end{align}
y sus autovectores
\begin{align}
    \ket{u_{1,2}^{(1)}}&=\frac{1}{N_{1,2}^{(1)}}(-\sqrt{2}g\ket{gg1}+ \left(\frac{\chi-\Delta}{2}+J-k \mp \sqrt{2g^2+(k-J+\frac{\Delta -\chi}{2} )^2} \right)\dfrac{\ket{eg0}+\ket{ge0}}{\sqrt{2}}\\
    \ket{u_3^{(1)}}&= \frac{1}{\sqrt{2}}(\ket{eg0}-\ket{ge0})
\end{align}
Con esto, podemos resolver analíticamente la evolución temporal de cualquier estado inicial.
Para esto solo tenemos que desarrollar el estado inicial en términos de los autovectores, y la evolución temporal esta dada por
\begin{equation}
\ket{\psi(t)}=e^{-iHt}\ket{\psi(0)}=\sum_{j,n} c_j^{(n)}e^{-iE_j^{(n)}t}\ket{u_j^{(n)}}
\end{equation}
donde $c_j^{(n)}=\braket{u_j^{(n)}}{\psi(0)}$.
La complejidad de estas expresiones hace complicado conseguir conclusiones interesantes,aun asi, algo que se puede notar, es la diferencia fundamental que se encuentra para las energías con un numero total de excitaciones $N=1$ y $N>1$. Si se observa el factor que esta antes de la raiz cuadrada,  se ve que para el caso en que $N \geq 2$ tenemos un $\frac{1}{3}\beta_n$ que solo depende de $\chi$, $J$, $k$ y $n$. Mientras tanto, en el caso de $N=1$, este factor depende del detunning $\Delta$. Esto es interesante, ya que uno podría pensar que la formula para $N$ excitaciones se puede generalizar para incluir $N=0,1$, pero la fundamental diferencia de tener mas o menos estados que interactúan entre si, da lugar a efectos fundamentalmente diferentes. Si uno mira en detalle las cuentas, se percata de que en el caso de $N=1$ este factor $\Delta$ aparece, ya que en la matriz Hamiltoniana el único estado con $N=1$ que tiene un termino que incluye al detunning, es el estado $\ket{gg1}$, y los otros dos estados al ser un átomo excitado y otro no, el termino de detunning se cancela. Por lo tanto, este termino con $\Delta$ sobrevive, al contrario que en todos los demás subespacios, ya que tenemos por un lado el termino del $\ket{ggn}$ que nos aporta un $\Delta$, y el termino de $\ket{ee,n-2}$ que nos aporta otro $\Delta$ pero con el signo cambiado, y elimina la contribución del primer estado a la energía. Esto es muy interesante, ya que para $N=1$, si aumentamos el detunning, no solo se separan los niveles de energía, sino que también hay una asimetría por el termino independiente. Para analizar esto en detalle, en la figura \ref{fig:relación energia detunning} se observan las energías de los diferentes niveles en función del detunning. 

\begin{figure}
    \centering
    \begin{subfigure}[h]{0.49\textwidth}
        \centering
        \includegraphics[width=\textwidth]{figuras/ch4/relacion_energia_detunning1.pdf}
        \caption{$k=J=0$}
        \label{fig:relación energia detunning 1}
    \end{subfigure}
    \hfill
    \begin{subfigure}[h]{0.49\textwidth}
        \centering
        \includegraphics[width=\textwidth]{figuras/ch4/relacion_energia_detunning2.pdf}
        \caption{$J\neq 0$}
        \label{fig:relación energia detunning 2}
    \end{subfigure}
       \caption{Relación entre energía y detunning para los diferentes niveles de energía del problema. Las lineas solidas muestran la energía de los estados del JC doble con N=0 (negro, solido), N=1 (verde oscuro y lima, solido) y N=2 (rojo, naranja, amarillo y gris; solido). Cambien se muestran los niveles de energía del JC de un átomo para N=1 (negro; rayado) y N=2 (rojo; rayado). Obsérvese que las energías del JC de un átomo están multiplicadas por 2.}
       \label{fig:relación energia detunning}
\end{figure}
En esta figura \ref{fig:relación energia detunning 1} se observan las energías de los primeros niveles para el modelo de un átomo, mostrados con lineas rayadas, y de dos átomos, con lineas solidas; para esta figura se tomaron átomos que no interactúan ($k=J=0$) y una cavidad lineal ($\chi=0$). Se puede ver que, si bien el modelo de dos átomos tiene estructuras mas complicadas, son similares a las de 1 átomo. En primer lugar, los estados con N=2 (rojo y naranja; solido) tienen una forma igual a la de JC de 1 átomo, si bien esta un poco desfasada, es interesante ver como las lineas tienen una coincidencia muy grande, recordando que en el gráfico las lineas rayadas están multiplicadas por 2, esto nos da una interpretación bastante buena, y es que la energía de dos átomos no interactuantes en una cavidad es igual (o muy parecida) a dos veces la energía de 1 átomo en una cavidad. \textcolor{red}{Esto tengo que chequear con cuentas} \textcolor{blue}{Creo que esto se debe al corrimiento Lamb, ya que ahora tenemos dos átomos que interactúan con el vacío, entonces el corrimiento es 1 unidad mas grande en los extremos, que es justamente lo que vemos en el gráfico, cuando el detunning es muy negativo, la energ\'ia tiende a ser igual a la de un JC simple con 2 excitaciones, y cuando el detunning es muy positivo, entonces tiende a la de 1 excitaci\'on; esta asimetr\'ia para $\Delta>0$ y $\Delta<0$ se observar\'a en resultados posteriores.} Por otro lado, se puede observar lo que se había comentado anteriormente, que la energ\'ia de los estados con $N=1$ tienen un t\'ermino fuera de la raíz, que hace que sea mas asim\'etrico a\'un. Normalmente, en el JC de 1 átomo, ya que todos los niveles de energía tienen una forma funcional igual, este termino de afuera de la raíz se le puede agregar o quitar como un offset en la energía del estado fundamental, la diferencia con este caso es que, no todos los niveles de energía presentan esto, entonces si agregamos un offset, igualmente habría una diferencia. 

Otra cosa interesante de notar es que si la cavidad es lineal, entonces los estados antisimetricos de diferentes excitaciones $\frac{1}{\sqrt{2}}(\ket{eg,n}-\ket{ge,n})$ y $\frac{1}{\sqrt{2}}(\ket{eg,n'}-\ket{ge,n'})$, estan degenerados en energía.

Una vez estudiados los niveles de energía y comparados con el caso de 1 átomo, vamos a proseguir con la dinámica del problema, que en el caso unitario puede resolverse analíticamente, pero a\'un así, nos concentraremos en simulaciones numéricas.
Para comenzar, vamos a intentar de recuperar el caso de un átomo, asimetrizando el acoplamiento uno de los dos átomos que tenemos en la cavidad, y haciendo tender este a cero, es decir, vamos a trabajar con $k=J=0$ y vamos a agregar un parámetro adimensional $\alpha$ que solamente actúa sobre el átomo 2, y sirve de apantallamiento. Este parámetro $\alpha$ acompañara a las constantes de acoplamiento, por ejemplo el acoplamiento entre el átomo y la cavidad $g\rightarrow g\alpha$, tal que si $\alpha \rightarrow 0$ entonces el átomo quedara desacoplado de la cavidad. 

\section{Dinámica con apantallamiento}
\label{sec4:dinamica apantallamiento}
Lo primero que se tiene que hacer es recuperar los resultados anteriores. Para aclarar, en la figura \ref{fig4:diagrama esquematico} se muestra un esquema de como es el problema que se esta trabajando, con los nombres que se le darán a las partes del sistema. Llamaremos átomo B al que esta apantallado mediante el parámetro adimensional $\alpha$, el indice A se referirá al otro átomo, y C a la cavidad. Entonces, para recuperar los resultados anteriores, se propone que $\alpha=0$ y la interacción entre los átomos $k=J=0$. De esta manera, se elige en analogía con el caso de 1 átomo, como estado inicial cualquier estado donde el átomo A sea excitado, y la cavidad C no tenga ningún fotón; por lo tanto se elige el estado inicial mas sencillo posible que cumple estas condiciones $\ket{\psi_0}=\ket{eg0}$. Si bien este apantallamiento no tiene un significado físico, y experimentalmente es imposible lograr estas condiciones, realizar este estudio sirve para entender cualitativamente los efectos de cada parámetro del problema, y también entender que el entrelazamiento entre los dos átomos, lleva a efectos impredecibles. La complejización del problema de 1 átomo al de 2 átomos es muy grande, y por eso es necesario ir de a poco.
\begin{figure}[H]
    \begin{minipage}[c]{0.67\textwidth}
        \includegraphics[width=\textwidth]{figuras/ch4/diagrama esquematico.png}
    \end{minipage}\hfill
    \begin{minipage}[c]{0.3\textwidth}
    \caption{Esquema del problema de estudio. Se nombran a las partes para referenciarlas fácilmente. Los átomos los llamamos A y B, donde el átomo B es el que sufre el apantallamiento que utilizaremos para recuperar los resultados anteriores. La cavidad la llamaremos C, esta puede contener una cantidad arbitraria de excitaciones, pero nos concentraremos principalmente en 0,1 y 2 excitaciones. Ambos átomos son de dos niveles, y en principio son idénticos e indistinguibles, pero se le agrega un apantallamiento artificial.}
    \label{fig4:diagrama esquematico}
  \end{minipage}
\end{figure}
Utilizando esta condición inicial se realiza una simulación numérica y se observan las poblaciones, y se espera recuperar la misma dinámica que en el caso de 1 átomo, ya que el átomo B no interactúa con ninguna de las otras partes del sistema A y C. Para poder representar el estado del sistema sobre una esfera de Bloch, se realiza una traza parcial sobre el átomo B, y así se obtiene la figura \ref{fig4:bloch delta}, donde se muestran 3 trayectorias correspondientes a diferentes valores del detunning, la linea azul es el caso resonante $\Delta=0$, y las trayectorias morada y naranja se corresponde con $\Delta=0.5g$ y $\Delta=2g$ respectivamente.
\begin{figure}[H]
    \begin{minipage}[c]{0.67\textwidth}
        \includegraphics[width=\textwidth]{figuras/ch4/bloch eg0 bloch AC a=0 d=2.0 x=0.0 k=0.0 J=0.0 gamma=0.0 p=0.0.png}
    \end{minipage}\hfill
    \begin{minipage}[c]{0.3\textwidth}
    \caption{
         } \label{fig4:bloch delta}
  \end{minipage}
\end{figure}
Se observa como la dinámica entre estos dos estados es exactamente igual que la observada en la figura \ref{fig3:bloch cinematica}, ademas, como todos los puntos están sobre la superficie de la esfera, los estados son puros, diciéndonos que el estado global es separable, y entonces haber trazado sobre el átomo B no tuvo efecto sobre la dinámica entre el átomo A y la cavidad. Para corroborar esto se realiza un análisis poblacional mas general. \textcolor{red}{hacer gráfico con las probabilidades y ver que la probabilidad de eg0 + gg1 =1}
Lo siguiente que podemos analizar, que no se tenia la posibilidad cuando se tiene 1 átomo, es que se puede considerar una condición inicial entrelazada. Si bien los átomos no interactúan, y el átomo B esta aislado del universo, se puede entrelazar los átomos y luego se apagan las interacciones del átomo B. Por ejemplo, si se considera el estado inicial entrelazado $\ket{\psi_0}=(\ket{eg0}+\ket{ge0})/\sqrt{2}$, se obtiene 
\begin{figure}[H]
    \begin{minipage}[c]{0.67\textwidth}
        \includegraphics[width=\textwidth]{figuras/ch4/bloch eg0+ge0 bloch AC a=0 d=2.0 x=0.0 k=0.0 J=0.0 gamma=0.0 p=0.0.png}
    \end{minipage}\hfill
    \begin{minipage}[c]{0.3\textwidth}
    \caption{
         } \label{fig4:bloch delta}
  \end{minipage}
\end{figure}
Ahora, los estados no están sobre la superficie, lo que se interpreta como que estamos en presencia de un estado mixto. Al trazar sobre el átomo B, efectivamente se considera como si este fuese parte de un entorno. Al olvidarse de la dinámica del segundo átomo, se puede interpretar como que este se lleva un 50\% de probabilidad de llevarse la excitación, ya que no sabemos si inicialmente el átomo A o el átomo B es el que tiene la excitación. Entonces efectivamente tenemos un 50\% de probabilidad de que el estado de la cavidad sea $\ket{g0}$, y no evoluciona, y un 50\% de probabilidad de que la excitación este dentro de la cavidad, y por lo tanto vemos que la dinámica es la misma que en el caso anterior, pero con amplitudes menores. \textcolor{blue}{ACA IBA A DECIR ALGO, PERO ME PARECE QUE NO PUEDO PORQUE EL ESTADO ENTRELAZADO QUIZAS TIENE ALGUNAS COSAS RARAS. Uno puede adelantarse un poco, y deducir como se comporta la fase geométrica en estos dos casos. Por un lado, en el caso que el estado inicial no este entrelazado, la dinámica es exactamente igual que en el caso de 1 átomo, entonces la fase geométrica es la misma que \ref{eq3:fg unitaria jcm}, en cambio, cuando el estado inicial es el entrelazado, como la dinámica es igual que antes pero con un medio de la probabilidad, y luego el átomo B no evoluciona, entonces es autoestado y no acumula fase geométrica. Por lo tanto se puede concluir que en el caso entrelazado la FG va a ser la mitad que en el caso no entrelazado.}

Para analizar mas en detalle la dinámica, y para poder realizar comparaciones cuando se complejice el problema, se puede realizar un estudio poblacional, y también podemos mirar las entropías relativas y otros observables importantes.
En primer lugar, el caso separable $\psi_0=\ket{eg0}$, es idéntico al caso de 1 átomo, ya que el átomo B no evoluciona por estar totalmente aislado del sistema. Lo único que se puede resaltar es que, si se traza sobre la cavidad, que es algo que es útil para observar el entrelazamiento entre los átomos, lo único destacable es que el estado es mixto, ya que la evolución temporal del sistema átomo A-átomo B consta del átomo B en el estado fundamental $\ket{g}$, y el átomo A oscila entre el estado excitado y fundamental. La amplitud de oscilación y el grado de mixing entre los estados depende del detunning, siendo el caso $\Delta=0$ el de oscilaciones coherentes entre estados, y al aumentar $\Delta$ se este comportamiento.
En segundo lugar, cuando el estado inicial de los átomos no es separable por estar entrelazados $\ket{\psi_0}=(\ket{eg0} + \ket{ge0})/\sqrt{2}$, entonces la dinámica es un poco diferente. La figura \ref{fig4:fig4:dinamica eg0 sim resonante} muestra el caso de $\Delta=0$, donde se observan las evoluciones de las diferentes partes del sistema.

\begin{figure}[h]
    \centering
    \begin{subfigure}{0.49\textwidth}
        \centering
        \includegraphics[width=\textwidth]{figuras/ch4/d eg0+ din ABC d=0.png}
        \caption{}
        \label{fig4:dinamica pob eg0 sim resonante}
    \end{subfigure}
    \hfill
    \begin{subfigure}{0.49\textwidth}
        \centering
        \includegraphics[width=\textwidth]{figuras/ch4/d eg0+ din svn d=0.png}
        \caption{}
        \label{fig4:dinamica svn eg0 sim resonante}
    \end{subfigure}
    \caption{\textcolor{red}{labels, ticks y legens chiquitos. unificar colores}Panel (a):Dinámica poblacional para el caso resonante $\Delta=0$ con el estado inicial entrelazado $\ket{\psi_0}=(\ket{eg0} + \ket{ge0})/\sqrt{2}$. Panel (b):Entropía de von Neuman del sistema total (negro con puntos), y de diferentes subsistemas. En rojo se muestra la entropía del sistema habiendo trazado parcialmente sobre la cavidad, y en azul habiendo trazado parcialmente sobre el átomo B.}
    \label{fig4:dinamica eg0 sim resonante}
\end{figure}

En la figura \ref{fig4:dinamica pob eg0 sim resonante} se muestran las poblaciones y las coherencias correspondientes a la condición inicial $\ket{\psi_0}=(\ket{eg0} + \ket{ge0})/\sqrt{2}$ en el caso resonante, y en la figura \ref{fig4:dinamica svn eg0 sim resonante} se muestra la entropía de Von Neuman, en función del tiempo $t/T$ con $T=2 \pi \Omega(n,j)$  . La entropía de von Neuman es una cantidad que esta definida según:
\begin{equation}\label{eq4:entropia von neuman}
    S=-\Tr(\rho \ln \rho)=-\sum_j \lambda_j \ln \lambda_j
\end{equation}
donde $\rho$ es la matriz densidad del sistema, y $\lambda_j$ son los autovalores de la matriz densidad. La entropía de Von Neuman sirve para determinar si un estado es puro o mixto, ya que $S(\rho)=0$ representa un estado puro, y $S(\rho)=\ln(N)$ representa un estado máximamente mixto, donde $N$ es la dimensión del espacio de Hilbert.
Vemos como el estado $\ket{ge0}$ no evoluciona, ya que en este caso, el átomo B contiene la única excitación y esta aislado. Pero la otra parte, si que evoluciona. Vemos la presencia de las mismas oscilaciones coherentes entre los estados $\ket{eg0}$ y $\ket{gg1}$. La diferencia principal es que en esta caso, el estado de los subsistemas es mixto. Esto se observa claramente en el gráfico de la entropía, pero también se puede deducir este comportamiento desde la figura \ref{fig4:dinamica pob eg0 sim resonante}, ya que a $t/T=0.5$, tenemos el estado $\ket{\psi(T/2)}=\ket{g}_A\otimes(\ket{e_B0_C}+\ket{g_B1_C})/\sqrt{2}$, que es separable solo en el átomo A, y los otros dos están totalmente entrelazados, y por lo tanto al tomar traza parcial tal que el átomo B y la cavidad estén separadas, este estado es máximamente mixto. Vemos como el entrelazamiento entre la cavidad y el átomo B, que están totalmente aislados, evoluciona indirectamente por medio del átomo A, y paradójicamente este queda desentrelazado del sistema para tiempos $t=(k-1/2)T\; ; \; k \in \mathbb{N}$. En este punto notamos algo muy importante, y es que la entropía de von Neuman solo sirve para estados puros. Cuando $t=0$, la entropía del subsistema AB es 0, porque es un estado puro, y esta máximamente entrelazado. Pero al evolucionar, el subsistema AB se hace mixto, y como se observa en la linea azul, la entropía del átomo B es siempre $\log 2$, que según la interpretación de la entropía de von Neuman es que esta siempre máximamente entrelazado. Este no es al caso, y la descripción falla porque el estado AB no es puro.

Entonces, ya que el entrelazamiento es un recurso muy importante y estudiado para las información cuántica, es necesario introducir una medida de entrelazamiento, para poder estudiarlo en este tipo de situaciones. Si bien la entropía de Von Neuman es útil en el caso de estados puros, cuando tenemos estados mixtos como se vio recién, o en el caso de tener un sistema abierto, esta medida ya no sirve. Una de las medidas mas utilizadas y con mayor aplicación es el \textit{Entanglement of Formation} ($E_F$) \cite{an intro to entanglement measures}, que coincide con la entropía de von Neuman para estados puros, y sirve para estados mixtos. El $E_F$ esta definido como
\begin{equation}
    E_F(\rho)=\text{inf}\left( \sum_i p_i E(\ketbra{\psi_i}{\psi_i}) : \rho = \sum_i p_i\ketbra{\psi_i}{\psi_i}\right)
\end{equation}
Esta medida representa el entrelazamiento promedio mínimo entre todas las posibles descomposiciones puras de $\rho$, donde $E(\ketbra{\psi_i}{\psi_i})=S(\tr_B{\ketbra{\psi_i}{\psi_i}})$ es la entropía de von Neuman, que es la medida que se utiliza para estados puros. Esta definición es general, pero en el caso presente, nos sirve una simplificación de esta medida que se obtiene si se estudia el entrelazamiento entre dos qu-bits, como lo son los átomos A y B. Esta medida es la concurrencia, y esta definida como
\begin{equation}
    C(\rho)=\text{max}\{0,\lambda_1-\lambda_2-\lambda_3-\lambda_4\}
    \label{ec4:concurrencia}
\end{equation}
donde los $\lambda_i$ son las raíces de los autovalores, en orden decreciente, de la matriz $\rho\sigma_y\otimes\sigma_y\rho^*\sigma_y\otimes\sigma_y$, donde $\rho*$ es el conjugado (sin transponer) de $\rho$. La concurrencia y la entropía de formación $E_F$ están relacionadas, y la concurrencia obtiene su interpretación a través de esta. Un estado máximamente entrelazado tiene $C(\rho)=1$ y un estado separable $C(\rho)=0$. 


\begin{figure}[H]
    \begin{minipage}[c]{0.67\textwidth}
        \includegraphics[width=\textwidth]{figuras/ch4/d eg0+ concu d=0.png}
    \end{minipage}\hfill
    \begin{minipage}[c]{0.3\textwidth}
    \caption{Concurrencia en el caso resonante para estado inicial $\ket{eg0}+\ket{ge0}$} 
    \label{fig4:concu eg0 sim}
  \end{minipage}
\end{figure}
En la figura \ref{fig4:concu eg0 sim} se observa la concurrencia entre los átomos AB, para el caso estudiado anteriormente. Como era de esperar, a $t=0$ el estado es máximamente entrelazado, y luego el entrelazamiento se pierde a $t=T/2$, donde el átomo B esta entrelazada con la cavidad. 

\subsection{Interacción átomo-átomo}

El siguiente paso es analizar el rol de las interacciones entre los átomos, aun manteniendo el apantallamiento $\alpha=0$. Para esto, se sigue utilizando las mismas condiciones iniciales y el átomo B seguirá sin interactuar con la cavidad, pero se considera ahora que la interacción entre átomos dadas por los parámetros $k$ y $J$ ahora serán distintos de cero. Para comenzar, en la figura \ref{fig4:k eg0 abc} se observa la evolución temporal para el estado inicial $\ket{\psi_0}=\ket{eg0}$, con $\Delta = 0$, $J=0$ pero $k=0.1g$. Recordemos que $k$ es la intensidad de la interacción $\sigma^{(1)}_+\sigma^{(2)}_-+\text{c.c.}$ (ver \ref{eq4:H}).

\begin{figure}[h]
    \begin{minipage}[c]{0.67\textwidth}
        \includegraphics[width=\textwidth]{figuras/ch4/k eg0 ABC.png}
    \end{minipage}\hfill
    \begin{minipage}[c]{0.3\textwidth}
    \caption{Dinámica poblacional para la condición inicial $\ket{\psi_0}=\ket{eg0}$, para los parámetros $\Delta=0$, $J=0$ y $k=0.1g$. Las lineas solidas se corresponden con las poblaciones de la matriz densidad total del sistema; en azul la probabilidad de encontrar al estado en el estado $\ket{gg1}$, en verde en $\ket{eg0}$, en rojo $\ket{ge0}$, y en negro $\ket{gg0}$. Las lineas rayadas son las coherencias entre estas poblaciones, la violeta entre $\ket{gg1}$ y $\ket{ge0}$, la celeste entre $\ket{eg0}$ y $\ket{gg1}$ y la amarilla entre $\ket{eg0}$ y $\ket{gg1}$.
         } \label{fig4:k eg0 abc}
  \end{minipage}
\end{figure}
Lo que sucede es que la excitación esta inicialmente en el átomo A, y como siempre, se observan oscilaciones entre los estados $\ket{eg0}$ y $\ket{gg1}$, la diferencia es que al haber interacciones entre los átomos, ahora la excitación inicial que esta en el átomo A, sufre dos procesos diferentes, primero la oscilación, y ademas, la interacción con el átomo B. Al tener la excitación el átomo A, una parte de esta se va hacia la cavidad, y la otra hacia el átomo B, excitándolo parcialmente. La amplitud de la oscilación depende de la intensidad de la interacción $k$. Si nos concentramos en la curva roja, vemos que su pendiente crece mientras que la probabilidad de $\ket{eg0}$ es mayor a la de $\ket{gg1}$, luego la amplitud crece, pero de manera desacelerada, hasta que la probabilidad del estado $\ket{eg0}$ es nula. En ese momento, ya no hay excitación que pasar del átomo A al B, y el proceso se revierte. Antes de analizar el entrelazamiento entre los átomos, se observa en la figura \ref{fig4:k eg0 sim abc} la dinámica para los mismos parámetros, pero para la condición inicial entrelazada $\ket{\psi_0}=(\ket{eg0}+\ket{ge0})/\sqrt{2}$:
\begin{figure}[h]
    \begin{minipage}[c]{0.67\textwidth}
        \includegraphics[width=\textwidth]{figuras/ch4/k eg0+ ABC.png}
    \end{minipage}\hfill
    \begin{minipage}[c]{0.3\textwidth}
    \caption{Dinámica poblacional para la condición inicial $\ket{\psi_0}=\ket{eg0+ge0}$, para los parámetros $\Delta=0$, $J=0$ y $k=0.1g$. Las coherencias y poblaciones tienen los mismos colores que la figura anterior \ref{fig4:k eg0 abc}
         } \label{fig4:k eg0 sim abc}
  \end{minipage}
\end{figure}
La dinámica en este caso presenta oscilaciones en la población de $\ket{ge0}$ con un periodo dos veces mas grande. Esto se debe a una \"pelea\" entre los estados $\ket{eg0}$ y $\ket{ge0}$, ya que tienen las excitaciones en diferentes átomos. Inicialmente, como los estados están entrelazados, no esta bien definido en cual de los dos átomos esta la excitación, entonces la interacción $k$ se anula y vemos que tiene pendiente 0. Entonces la dinámica inicial es igual que para $k=0$ y comienza a oscilar. Apenas baja la curva verde, la probabilidad de encontrar la excitación en el átomo B es mayor que la del átomo A, entonces lo que sucede es que el átomo B comienza a perder esta excitación y se la da lentamente al átomo A, y por lo tanto la oscilación del estado $\ket{eg0}$ no llega a tener amplitud nula en $t/T=0.5$. Luego, la evolución sigue su curso oscilante, y al llegar a $t=T$, vemos que la probabilidad de encontrar la excitación en el átomo A es mayor, y por lo tanto comienza a revertirse la situación, hasta completar el ciclo para $t=2T$. 

El entrelazamiento entre los átomos se analiza utilizando la concurrencia, como se muestra en la figura \ref{fig4:concu k}, donde \ref{fig4:concu k eg0} muestra la condición inicial separable $\ket{eg0}$, y \ref{fig4:concu k eg0 sim} el entrelazamiento para la condición inicial entrelazada $\ket{eg0+ge0}$. 
\begin{figure}[h]
    \centering
    \begin{subfigure}{0.49\textwidth}
        \includegraphics[width=\textwidth]{figuras/ch4/k eg0 concu.png}
        \caption{$\ket{eg0}$}
        \label{fig4:concu k eg0}
    \end{subfigure}
    \hfill
    \begin{subfigure}{0.49\textwidth}
        \includegraphics[width=\textwidth]{figuras/ch4/k eg0+ concu.png}
        \caption{$\ket{eg0+ge0}$}
        \label{fig4:concu k eg0 sim}
    \end{subfigure}
    \caption{Dinámica de entrelazamiento para $\Delta=0$, $J=0$ y $k=0.1g$}
    \label{fig4:concu k}
\end{figure}

Ahora vamos a ver $k=0$ y $J\neq 0$. En la figura \ref{fig4:j alpha0}, vemos que, si bien la dinámica es similar, los átomos no se entrelazan.

\begin{figure}[H]
    \centering
    \begin{subfigure}{0.49\textwidth}
        \includegraphics[width=\textwidth]{figuras/ch4/j eg0 abc.png}
        \caption{$\ket{eg0}$ Poblaciones}
        \label{fig4:pob j eg0}
    \end{subfigure}
    \hfill
    \begin{subfigure}{0.49\textwidth}
        \includegraphics[width=\textwidth]{figuras/ch4/j eg0+ge0 abc.png}
        \caption{$\ket{eg0+ge0}$ Poblaciones}
        \label{fig4:pob j eg0 sim}
    \end{subfigure}
    \vfill
    \begin{subfigure}{0.49\textwidth}
        \includegraphics[width=\textwidth]{figuras/ch4/j eg0 concu.png}
        \caption{$\ket{eg0}$ Concurrencia}
        \label{fig4:pob j eg0}
    \end{subfigure}
    \hfill
    \begin{subfigure}{0.49\textwidth}
        \includegraphics[width=\textwidth]{figuras/ch4/j eg0+ge0 concu.png}
        \caption{$\ket{eg0+ge0}$ Concurrencia}
        \label{fig4:pob j eg0 sim}
    \end{subfigure}
    \caption{$\Delta=0$, $J=0.5g$ y $k=0$}
    \label{fig4:j alpha0}
\end{figure}
Vemos que la diferencia principal entre la interacción tipo Isign ($J\sigma_z^{(1)}\sigma_z^{(2)}$) y la dipolar ($k\sigma_+^{(1)}\sigma_-^{(2)}+\text{c.c.}$), es que el segundo parece entrelazar los átomos, ya que en el primer caso, el efecto es separar los niveles de energía, pero en el segundo no solo eso, sino que también pasa excitaciones de un atomo al otro. Si bien esto nos sirve para entender intuitivamente el efecto, el problema de este análisis es que estamos asumiendo cosas no físicas mediante el apantallamiento y la asimetría que imponemos entre los dos átomos. Esto, lleva a estos análisis que en realidad no son correctos, ya que si miramos el Hamiltoniano del sistema sin apantallamiento \ref{eq4:H}, donde usamos la base con estados simétricos y antisimetricos \ref{ec4:base}, el efecto de ambos parámetros debería ser el mismo, ya que solo aparecen en la diagonal principal. Si bien la interacción $J$ actúa sobre todos los estados, y el $k$ solamente solo sobre los $\ket{egn\pm gen}$, su principal función es separar las energías de los estados de la base.
Entonces sera necesario retomar este análisis sin apantallamiento y con la base \ref{ec4:base}.
\subsection{Medio Kerr}
\label{sec4:medio kerr}
Ahora nos concentramos en el efecto del medio Kerr. Para esto, apagamos las interacciones interatómicas $k=J=0$, y ahora se modifica el medio a través del parámetro $\chi$. 
\begin{figure}[h]
    \centering
    \begin{subfigure}{0.49\textwidth}
        \includegraphics[width=\textwidth]{figuras/ch4/x eg0 abc.png}
        \caption{$\ket{eg0}$}
        \label{fig4:pob x eg0}
    \end{subfigure}
    \hfill
    \begin{subfigure}{0.49\textwidth}
        \includegraphics[width=\textwidth]{figuras/ch4/x eg0+ abc.png}
        \caption{$\ket{eg0+ge0}$}
        \label{fig4:pob x eg0 sim}
    \end{subfigure}
    \caption{Dinámica de poblaciones para $x=g$}
    \label{fig4:pob x}
\end{figure}

Al igual que en el caso de 1 átomo, se puede observar en las ecuaciones \ref{ec4:autoenergias} y \ref{ec4:parametros solucion}, la frecuencia depende del medio. En la figura \ref{fig4:pob x} el tiempo esta normalizado con la frecuencia, entonces no se nota el cambio. Pero lo que es necesario analizar, es como las oscilaciones no son totalmente coherentes, en el sentido de que la probabilidad del estado $\ket{gg1}$ nunca alcanza la amplitud inicial de la oscilación, como en el caso de $\chi=0$. Esto se debe a que el aumento de $\chi$ hace que las energías de ambos estados se separen, y por lo tanto hace que las transiciones entre los estados sea menos probable. Este comportamiento también se observa si el estado inicial se toma como $\ket{gg1}$. Es lógico estudiar el entrelazamiento en este caso. En la figura \ref{fig4:concu x} se muestran las concurrencias para ambas condiciones iniciales. Es interesante comparar la figura \ref{fig4:concu x eg0 sim} con la figura en el caso de $\chi=0$ para esta misma condición inicial, la figura \ref{fig4:concu eg0 sim}. En principio se puede pensar que el medio no lineal rompe con el entrelazamiento del sistema, pero como se ve al comparar estas figuras, la interpretación correcta es que el medio no hace mas que ralentizar el comportamiento preexistente de la cavidad, ya que en este caso, no destruye el entrelazamiento, sino que lo conserva por virtud de haber ralentizado las amplitudes de oscilación entre los dos estados dinámicos.
\begin{figure}[h]
    \centering
    \begin{subfigure}{0.49\textwidth}
        \includegraphics[width=\textwidth]{figuras/ch4/x eg0 concu.png}
        \caption{$\ket{eg0}$}
        \label{fig4:concu x eg0}
    \end{subfigure}
    \hfill
    \begin{subfigure}{0.49\textwidth}
        \includegraphics[width=\textwidth]{figuras/ch4/x eg0+ concu.png}
        \caption{$\ket{eg0+ge0}$}
        \label{fig4:concu x eg0 sim}
    \end{subfigure}
    \caption{Dinámica de entrelazamiento para $x=g$}
    \label{fig4:concu x}
\end{figure}
Cambien se puede intentar de recuperar el comportamiento visto en el modelo de 1 átomo, que el medio Kerr no es mas que un desplazamiento lateral en las frecuencias, ademas de modificar las amplitudes. Para esto se realiza otra evolución para $\chi=\Delta=\frac{g}{2}$, y se compara con el caso en que $\chi=\Delta=0$
\begin{figure}[h]
    \centering
    \begin{subfigure}{0.49\textwidth}
        \includegraphics[width=\textwidth]{figuras/ch4/d=x=0 eg0 abc.png}
        \caption{$\Delta=\chi=0$}
        \label{fig4:comparacion kerr pob 1}
    \end{subfigure}
    \hfill
    \begin{subfigure}{0.49\textwidth}
        \includegraphics[width=\textwidth]{figuras/ch4/d=x=0.5 eg0 abc.png}
        \caption{$\Delta=\chi=0.5g$}
        \label{fig4:comparacion ker pob 2}
    \end{subfigure}
    \caption{Dinámica de entrelazamiento para $x=g$}
    \label{fig4:comparacion d vs x}
\end{figure}
Vemos como se anula el efecto del medio, y la dinámica es la misma pero con un cambio en la frecuencia. Al igual que antes, aumenta la frecuencia

\subsection{Batidos}

Al complejizar el problema, comienzan a aparecer batidos, comportamiento que se atribuye a la modulación de dos procesos simultáneos. Por ejemplo, si observamos la evolución temporal con $\chi\neq0$ y $k\neq0$, entonces el primero disminuye la amplitud de oscilación de, los estados con mayor cantidad de fotones en la cavidad, que dentro del subespacio $N$ la jerarquía del medio sera favorecer a los estados $\ket{eg,N-1}$ y $\ket{ge,N-1}$ por sobre el $\ket{ggN}$. Por el contrario, se observo que en esta situación, el termino de interacción entre los átomos disminuye la amplitud del estado $\ket{ggN}$ como se vio en la sección anterior. Por lo tanto, si tenemos dos procesos que están en juego y sus efectos son similares, entonces es esperable que se observen oscilaciones moduladas. No vale la pena mostrar la dinámica de las poblaciones, porque no se pueden sacar conclusiones muy importantes, pero si podemos observar la trayectoria en la esfera de bloch, para dar una idea de la complejidad de la evolución.

\section{Dinámica sin apantallamiento}
\label{sec4:dinamica sin apantallamiento}

Al sacar el apantallamiento, es necesario utilizar la base mencionada anteriormente \ref{ec4:base}, ya que los átomos son indistinguibles y esta base es mas apropiada. Ademas, el Hamiltoniano desacopla los estados antisimetricos, facilitando la solución. Por lo tanto, se procede a estudiar la dinámica sacando el apantallamiento. Lo que nos interesa estudiar es el entrelazamiento entre las diferentes partes del sistema, y su dependencia con los parámetros. 

\subsection{Dinámica con disipación}
Lo primero que hay que mirar es la dependencia de la dinámica con el régimen de acoplamiento, esperamos un comportamiento igual al del caso de un átomo \ref{sec3:regimen acoplamiento}. Recordemos que el régimen de acoplamiento fuerte (SC) es el caso en donde la interacción entre cavidad y átomos es mayor a la interacción entre sistema y entorno.
\begin{figure}[h]
    \centering
    \begin{subfigure}{0.7\textwidth}
        \includegraphics[width=\textwidth]{figuras/ch4/sc vs wc eg0 sim j0.5.png}
        \caption{$\ket{eg0+ge0}$. $\Delta=\chi=k=0$, $J=0.5g$}
        \label{fig4:acoplamiento eg0 sim}
    \end{subfigure}
    \vfill
    \begin{subfigure}{0.7\textwidth}
        \includegraphics[width=\textwidth]{figuras/ch4/sc vs wc gg1 k=0.5.png}
        \caption{$\ket{gg1}$. $\Delta=\chi=J=0$, $k=0.5g$ }
        \label{fig4:acoplamiento gg1}
    \end{subfigure}
    \caption{Dependencia de las poblaciones con el régimen de acoplamiento, para $\Delta=J=\chi=0$ y $k=0.5g$, y para dos condiciones iniciales diferentes.}
    \label{fig4:regimen acoplamiento}
\end{figure}
En  la figura \ref{fig4:regimen acoplamiento} se  muestran las coherencias y las poblaciones, como se esperaba, estas tienen el mismo comportamiento que en el caso de 1 átomo. Notablemente, se puede ver el efecto de la interacción entre los átomos, como se separan las energías inicialmente las oscilaciones no logran la inversión total de población, solo una inversión parcial, y a tiempo largos la disipación hace que se tenga una mayor probabilidad de encontrar al sistema en el estado $\ket{eg0+ge0}$ ya que tiene menor energía. Eventualmente alcanza su estado estacionario. Lo que se recupera, ahora que ya no hay apantallamiento, es que ambos tipos de interacción ($J$ y $k$) generan entrelazamiento. Como es de esperarse, en ambos casos la concurrencia es oscilatoria por la naturaleza oscilante del problema, pero ahora, como el estado al que oscila

Nuevamente, nos concentraremos en el régimen SC. Si bien hasta ahora nos concentramos en estados con 1 excitación, y es interesante por sus implicancias y similitudes al modelo de 1 átomo, considerar estados con mayor cantidad de excitaciones hace a la riqueza del problema. Si solo consideramos $N=1$, tenemos 3 estados en el subespacio, de los cuales uno es el estado antisimetrico $\ket{eg0-ge0}$, que esta desconectado de los otros estados, y por lo tanto efectivamente se tiene un modelo de Jaynes-Cummings normal. Si vamos a $N=2$, ahora tenemos 4 estados en el subespacio, y 3 son relevantes. Por lo tanto, ahora veremos cuales son los efectos de las interacciones y la dinámica para estados iniciales en el subespacio de $N=2$. Para mantener el paralelismo, comenzaremos con el estado $\ket{eg1+ge1}$, pero como se vera, las condiciones iniciales cambian totalmente la dinámica de entrelazamiento del sistema. La enorme cantidad de posibilidades para elegir condiciones iniciales, hace que estudiar todos sea imposible, así que nos concentraremos en algunos.
Al tener 3 estados dinámicamente relevantes, tenemos 3 autoestados con sus respectivas autoenergías, y por lo tanto tenemos 3 frecuencias que compiten entre si, y son las 3 frecuencias de Rabi del sistema:
\begin{equation}
    \begin{aligned}
        \Omega^{(n)}_{12} &= E^{(n)}_2-E^{(n)}_1 \\
        \Omega^{(n)}_{23} &=E^{(n)}_3-E^{(n)}_2 \\
        \Omega^{(n)}_{31} &= E^{(n)}_1-E^{(n)}_3         
    \end{aligned}
    \label{ec4:frecuencias de rabi}
\end{equation}
Estas frecuencias se muestran en función del detunning $\Delta$ y para diferentes valores de $\chi$ y de $k-J$ en la figura \ref{fig4:frecuencias de rabi}.
\begin{figure}
    \centering
    \begin{subfigure}{\textwidth}
        \includegraphics[width=\textwidth]{figuras/ch4/frecuencias k=0.5g chilist.png}
        \caption{Frecuencias de Rabi en función del detunning, para diferentes valores de $\chi/g\in[0,5]$ y $k-J=0.5g$. A la izquierda $\Omega_{12}$ y a la derecha $\Omega_{23}$ y $\Omega_{13}$}
        \label{fig4:rabi chi}
    \end{subfigure}
    \vfill
    \begin{subfigure}{\textwidth}
        \includegraphics[width=\textwidth]{figuras/ch4/frecuencias x=0 klist.png}
        \caption{Frecuencias de Rabi en función del detunning, para diferentes valores de $k-J\in[0,5g]$ y $\chi=0$. A la izquierda $\Omega_{12}$ y a la derecha $\Omega_{23}$ y $\Omega_{13}$. Solo se muestra una de las ramas.}
        \label{fig4:rabi k}
    \end{subfigure}
    \caption{Frecuencias de Rabi en función del detunning $\Delta$ para diferentes valores de $\chi$, y $|k-J|$, donde las frecuencias de Rabi son $\Omega^{(2)}_{ij}=E^{(2)}_{j}-E^{(2)}_{i}$}
    \label{fig4:frecuencias de rabi}
\end{figure}

% \begin{figure}
%     \centering
%     \begin{subfigure}{0.3\textwidth}
%         \includegraphics[width=\textwidth]{figuras/ch4/omega12 k=0.png}
%         \caption{}
%         \label{fig4:rabi chi}
%     \end{subfigure}
%     \hfill
%     \begin{subfigure}{0.3\textwidth}
%         \includegraphics[width=\textwidth]{figuras/ch4/omega23 k=0.png}
%         \caption{}
%         \label{fig4:rabi chi}
%     \end{subfigure}
%     \hfill
%     \begin{subfigure}{0.3\textwidth}
%         \includegraphics[width=\textwidth]{figuras/ch4/omega31 k=0.png}
%         \caption{}
%         \label{fig4:rabi chi}
%     \end{subfigure}
%     \vfill
%     \begin{subfigure}{0.3\textwidth}
%         \includegraphics[width=\textwidth]{figuras/ch4/omega12 x=0.png}
%         \caption{}
%         \label{fig4:rabi chi}
%     \end{subfigure}
%     \hfill
%     \begin{subfigure}{0.3\textwidth}
%         \includegraphics[width=\textwidth]{figuras/ch4/omega23 x=0.png}
%         \caption{}
%         \label{fig4:rabi chi}
%     \end{subfigure}
%     \hfill
%     \begin{subfigure}{0.3\textwidth}
%         \includegraphics[width=\textwidth]{figuras/ch4/omega31 x=0.png}
%         \caption{}
%         \label{fig4:rabi chi}
%     \end{subfigure}
%     \caption{Frecuencias de Rabi en función del detunning, para diferentes valores de $\chi/g\in[0,5]$ y $k-J=0.5g$. A la izquierda $\Omega_{12}$ y a la derecha $\Omega_{23}$ y $\Omega_{13}$}
%     \label{}
% \end{figure}
En esta solo se muestra una de las dos ramas (se puede tener $\pm \Omega_{ij}$) por simplicidad. Lo que podemos concluir de esto es que la frecuencia $\Omega^(2)_{12}$ se comporta de manera similar a la frecuencia de Rabi del JC de 1 átomo, que es también igual a la $\Omega^(1)$; en ambos casos la diferencia de energía presenta un máximo que se desplaza lateralmente al aumentar el parámetro $\chi$, pero presenta una sutil diferencia ya que también el máximo aumenta en valor absoluto. También se observa en la figura \ref{fig4:rabi k}, a la izquierda, como depende esta frecuencia al aumentar $k-J$, y vemos que el máximo ya no se desplaza lateralmente, sino que solo aumenta en valor absoluto. Por otro lado, en los paneles derechos, se puede analizar que sucede con las frecuencias al aumentar $\chi$ (fig. \ref{fig4:rabi chi}) y $k-J$ (fig. \ref{fig4:rabi k}). Es interesante ver como al aumentar $\chi$, se comienza a observar un máximo y mínimo local para la frecuencia $\Omega^{(2)}_{23}$ (la superior). Similarmente, pero en la rama superior, se observa este mismo comportamiento al aumentar $k-J$.

La dinámica para el caso de $N\geq2$ es muy complicada, ya que ahora se tienen 3 frecuencias diferentes, y predecir que sucede para cada combinación de parámetros y para cada condición inicial se hace muy complicado, por lo tanto, el análisis para estos casos no puede ser muy profundo. Lo que se puede distinguir es que cuando $\chi,k-J>g$, se comienzan a observar estos mínimos y máximos locales en las frecuencias, y se puede intentar de ver cual es el efecto que tiene esto en el entrelazamiento de los átomos, y también se puede observar si hay alguna relación entre los parámetros, por ejemplo como se encontró para el caso de 1 átomo (y para el subespacio de N=1) que hay una clara relación entre el detunning $\Delta$ y el medio $\chi$.

\section{Dinámica de entrelazamiento}
Para estudiar la dinámica de entrelazamiento entre los dos átomos, nos centraremos en la concurrencia ($0\leq C_{AB} \leq 1$).
\textcolor{blue}{En primer lugar consideraremos una cavidad lineal, y finalmente veremos cual es el efecto del medio Kerr sobre el entrelazamiento.}
En esta sección siempre se estudia el estado entre los dos átomos $\rho_{AB}=\tr_C\{\rho\}$, y se usa como medida de entrelazamiento la concurrencia, definida por la ecuación \ref{ec4:concurrencia}.

Lo primero que tenemos que analizar es los efectos de las interacciones entre los átomos, como ya vimos,  vamos a definir dos regímenes, que llamaremos Strong Interacting (SI) y Weak Interacting (WI), refiriéndonos a la interacción entre los átomos con respecto a la cavidad. El SI sera cuando la interacción entre los átomos es fuerte en comparación con la cavidad, es decir $k-J>g$, y WI con $k,J<g$. Ya que no se definió un limite muy claro, trabajaremos con un valor representativo de cada régimen, $k-J=0.5g$ y $k-J=2.5g$ respectivamente, ademas de utilizar un parámetro del entorno $\gamma=0.25g$. Para ilustrar la las diferencias entre condiciones iniciales, y para seguir con el paralelismo con el caso de 1 átomo, se consideraran unicamente las siguientes condiciones iniciales:
\begin{itemize}
    \item $\ket{eg0+ge0}$ para seguir el paralelismo con el JCM de 1 átomo 
    \item $\ket{eg1+ge1}$ para comparar con el anterior y explorar el subespacio N=2
    \item $\ket{ee0+gg2}$ para ver otra condición inicial con $N=2$ en donde los atomos NO estan entrelazados
\end{itemize}

Para poder comparar el efecto de las dependencias entre si, todos los graficos mostrados en el resto del capitulo estaran caracterizados por un acoplamiento con el entorno de $\gamma=0.25g$, y todos los tiempos estaran normalizados segun un periodo que llamaremos $T_0=2\pi/\Omega_0 \, | \Omega_0=\Omega^{(1)}(\Delta=0,\chi=0,k-J=0)$.
\subsection{Dependencia con el detunning}
\subsubsection{\underline{Condicion inicial $\ket{eg0+ge0}$}}
\begin{figure}[h]
    \centering
    \begin{subfigure}{0.49\textwidth}
        \includegraphics[width=\textwidth]{figuras/ch4/concu/delta/eg0+ge0 k=0.0g x=0.0g J=0.0g gamma=0.25g concu delta uni.png}
        \caption{Dinámica sin perdidas}
        \label{fig4:concu detunning 0 uni}
    \end{subfigure}
    \hfill
    \begin{subfigure}{0.49\textwidth}
        \includegraphics[width=\textwidth]{figuras/ch4/concu/delta/eg0+ge0 k=0.0g x=0.0g J=0.0g gamma=0.25g concu delta dis.png}
        \caption{Dinámica con perdidas}
        \label{fig4:concu detunning 0 dis}
    \end{subfigure}
    \caption{Dinámica de entrelazamiento para el estado inicial $\ket{eg0+ge0}$, en función del detunning, y para $\chi=k-J=0$.}
    \label{fig4:concu detunning 0}
\end{figure}
En la figura \ref{fig4:concu detunning 0} se observa la evolución de la concurrencia para la condición inicial $\ket{eg0+ge0}$, cuyo entrelazamiento entre los átomos es máximo. El eje x es el tiempo, y el eje y es el detunning $\Delta/g$. En el panel \ref{fig4:concu detunning 0 uni} se observa el caso sin perdidas; lo primero que notamos es que la figura es simetrica sobre $\Delta=0$ (caso resonante), y en este caso las oscilaciones son de menor frecuencia pero de mayor amplitud. A medida que se aumenta el valor absoluto del detunning, entonces las oscilaciones son de mayor frecuencia y ademas de menor amplitud. Consecuentemente, al tomar la dinámica disipativa, en el panel \ref{fig4:concu detunning 0 dis}, se observa como las oscilaciones siguen estando, pero ahora su máximo disminuye con el tiempo. Esto es de esperarse, ya que la cavidad deja escapar fotones y el estado del sistema se hace mixto. Un estado mixto ya no es entrelazado, porque no hay coherencias, entonces se observa el deterioro del entrelazamiento a medida que el tiempo avanza. Hay otros dos comportamientos notables. Por un lado, se observa como el máximo del entrelazamiento decae mas lentamente para valores mas altos del detunning, esto se debe a la naturaleza da la dinámica poblacional; al aumentar el detunning, como se vio anteriormente, uno de los efectos principales es que las oscilaciones entre los estados tienen poca amplitud, y en este caso, esto implica que si bien el estado $\ket{gg1}$ tienen amplitudes no nula, principalmente la probabilidad esta concentrada en el estado inicial, que no sufre decoherencia por perdida de fotones, ya que no tiene fotones en la cavidad. Entonces, mientras mayor el detunning, menor la probabilidad de encontrar al sistema en el estado $\ket{gg1}$, y por lo tanto tiene pocas probabilidades de perder fotones. El otro efecto que es interesante y que aparece con frecuencia en este tipo de sistemas, es la muerte y reanimación súbita del entrelazamiento, que llamaremos SDE (Sudden Death Effect) y SBE (Sudden Birth Effect) por sus siglas en ingles. En el caso disipativo, hay marcadas zonas en negro, estas muestran como el entrelazamiento \textit{muere} durante un tiempo finito y revive luego. Este efecto es sorprendente, ya que uno espera que las coherencias, responsables en gran medida del entrelazamiento entre átomos, decaigan asintóticamente. Este efecto muestra que esto no es asi para algunos sitemas, y aun cuando las coherencias son distintas de cero, el entrelazamiento puede ser nulo durante un tiempo, y luego reaparecer subitamente. La razon por la que aparece este fenomeno es que se esta tomando traza parcial sobre la cavidad, lo que hace que algunas coherencias desaparezcan. Aun asi, no se entiende bien el porque de este efecto.

La pregunta natural que sigue es si cambiar los parametros $\chi$ y $k-J$, cambia la forma de la figura \ref{fig4:concu detunning 0}. En la figura \ref{fig4:concu detunning 0 params} se muestran nuevamente la dinamica de entrelazamiento en funcion del detunning, pero ahora cambiando alguno de los parametros:

\begin{figure}[h]
    \centering
    \begin{subfigure}{0.49\textwidth}
        \includegraphics[width=\textwidth]{figuras/ch4/concu/delta/eg0+ge0 k=0.0g x=0.1g J=0.0g gamma=0.25g concu delta dis.png}
        \caption{$\chi=0.1g$}
        \label{fig4:concu detunning x1}
    \end{subfigure}
    \hfill
    \begin{subfigure}{0.49\textwidth}
        \includegraphics[width=\textwidth]{figuras/ch4/concu/delta/eg0+ge0 k=0.0g x=5.0g J=0.0g gamma=0.25g concu delta dis.png}
        \caption{$\chi=5g$}
        \label{fig4:concu detunning x2}
    \end{subfigure}
    \vfill
    \begin{subfigure}{0.49\textwidth}
        \includegraphics[width=\textwidth]{figuras/ch4/concu/delta/eg0+ge0 k=0.5g x=0.0g J=0.0g gamma=0.25g concu delta dis.png}
        \caption{$k-J=0.5g$}
        \label{fig4:concu detunning k1}
    \end{subfigure}
    \hfill
    \begin{subfigure}{0.49\textwidth}
        \includegraphics[width=\textwidth]{figuras/ch4/concu/delta/eg0+ge0 k=2.5g x=0.0g J=0.0g gamma=0.25g concu delta dis.png}
        \caption{$k-J=2.5g$}
        \label{fig4:concu detunning k2}
    \end{subfigure}
    \caption{Concurrencia en funcion del tiempo (eje x) y el detunning $\Delta$ (eje y) para diferentes parametros (los que no estan nombrados son cero).}
    \label{fig4:concu detunning 0 params}
\end{figure}
En el panel \ref{fig4:concu detunning x1} y \ref{fig4:concu detunning x2} se muestran dos casos en donde se puso de manifiesto la no linealidad del medio, cuyos parametros son $\chi=0.1g$ y $\chi=5g$ respectivamente, ambos casos en presencia del entorno. Comparando con la figura \ref{fig4:concu detunning 0 dis}, vemos que la forma cambia al aumentar mucho las no linealidades, y la figura ya no es simetrica. De todas formas, cerca del centro el comportamiento es similar, y por lo tanto podria esperarse que este eje sera el caso coherente, que en analogia con la seccion \ref{sec3:fg disipacion}, puede ser una posible condicion de robustez para la fase geometrica, que analizaremos mas tarde. Vemos como al aumentar $\chi$, el eje de simetria tambien se desplaza en esa misma cantidad. En los paneles \ref{fig4:concu detunning k1} y \ref{fig4:concu detunning k2} se muestra el cambio al considerar acomplamiento entre los atomos, dados por una intensidad de $k-J=0.5g$ y $k-J=2.5g$ respectivamente. Nuevamente hay un desplazamiento, que se comportan igual que en el caso anterior, pero ahora el desplazamiento del eje de simetria es el doble que la intensidad de la interaccion. Esto no es tan sorprendente si se observa la expresion de la energia \ref{ec4:energias n1} \textcolor{red}{ec 4.9}, donde los parametros $\chi$ y $\Delta$ aparecen con un factor $1/2$, mientras que $k-J$ no. 

La simetria de la figura se rompe en una direccion, y se ve como hay en ambos casos una franja donde el entrelazamiento parace mantenerse durante un periodo mas largo de tiempo. Ademas, se observa como en el panel \ref{fig4:concu detunning k2} que se corresponde con el caso $k-J=2.5g$, en la parte superior de la imagen se ve como el entrelazamiento persiste mayor tiempo que su contraparte de menor interaccion (panel \ref{fig4:concu detunning k1}) con $k-J=0.5g$.

Basicamente, el subespacio de $N=1$ se comporta igual que el Jaynes-Cummings de 1 atomo, pero donde se tiene un estado \textit{inerte}, el estado $\ket{eg0-ge0}$ que no interactua con los otros dos. Esto puede servir para, usando como condicion inicial el estado $\ket{\psi_0}=\cos\theta\ket{eg0}+\sin\theta\ket{ge0}$ que es combinacion lineal de los estados $\ket{eg0\pm ge0}$, donde el estado simetrico evoluciona como siempre, y el antisimetrico no evoluciona, teniendo asi una manera de mantener el entrelazamiento siempre por arriba de un cierto valor, que depende del valor de $\theta$, aun en presencia de decoherencia, ya que el estado $\ket{eg0-ge0}$ es un estado que no sufre perdidas.

\subsubsection{\underline{Condicion inicial $\ket{eg1+ge1}$}}
Ahora se estudia la dinamica para el estado simetrico en el subespacio de $N=2$. Igual que en el caso anterior se tiene en la figura \ref{fig4:concu detunning 1} la evolucion unitaria a la izquierda, y la disipativa a la derecha. Se observa como en este subespacio hay mas estructura, y si bien sigue siendo simetrico con respecto al eje $\Delta=0$, ahora la forma es mas complicada, y ademas en el caso unitario tambien observamos regiones que presentan SDE, cosa que para $N=1$ no se observaba. Al igual que antes, al aumentar el detunning las oscilaciones son de mayor frecuencia y menor amplitud, conservando mejor el entrelazamiento. La razon es la misma: si bien ahora el estado $\ket{eg1+ge1}$ si puede perder fotones, al hacerlo cae al estado $\ket{eg0+ge0}$, que tiene el mismo entrelazamiento, y los otros dos estados del subespacio $\ket{gg2}$ decae al $\ket{gg1}$, y $\ket{ee0}$ no tiene fotones asi que no pierde excitaciones. Por lo tanto, en el caso de alta desintonia, el comportamiento es similar al de anterior. Las oscilaciones no tienen mucha amplitud y por lo tanto la probabilidad esta concentrada casi toda en el estado $\ket{eg1+ge1}$, que al perder un foton decae al estado $\ket{eg0+ge0}$ cuyo entrelazamiento es el mismo.
\begin{figure}[H]
    \centering
    \begin{subfigure}{0.49\textwidth}
        \includegraphics[width=\textwidth]{figuras/ch4/concu/delta/eg1+ge1 k=0.0g x=0.0g J=0.0g gamma=0.25g concu delta uni.png}
        \caption{Dinámica sin perdidas}
        \label{fig4:concu detunning 1 uni}
    \end{subfigure}
    \hfill
    \begin{subfigure}{0.49\textwidth}
        \includegraphics[width=\textwidth]{figuras/ch4/concu/delta/eg1+ge1 k=0.0g x=0.0g J=0.0g gamma=0.25g concu delta dis.png}
        \caption{Dinámica con perdidas}
        \label{fig4:concu detunning 1 dis}
    \end{subfigure}
    \caption{Dinámica de entrelazamiento para el estado inicial $\ket{eg0+ge0}$, en función del detunning, y para $\chi=k-J=0$.}
    \label{fig4:concu detunning 1}
\end{figure}

Lo interesante de este subespacio de mayor excitacion, es que como se observo anteriormente en las figuras \ref{fig4:frecuencias de rabi}, las energias presentan maximos y minimos locales, que esperamos que cambien drasticamente la estructura de las dependencias en los parametros al aumentar sus valores. 
\begin{figure}[h]
    \centering
    \begin{subfigure}{0.49\textwidth}
        \includegraphics[width=\textwidth]{figuras/ch4/concu/delta/eg1+ge1 k=0.0g x=0.1g J=0.0g gamma=0.25g concu delta dis.png}
        \caption{$\chi=0.1g$}
        \label{fig4:concu detunning 1 x1}
    \end{subfigure}
    \hfill
    \begin{subfigure}{0.49\textwidth}
        \includegraphics[width=\textwidth]{figuras/ch4/concu/delta/eg1+ge1 k=0.0g x=5.0g J=0.0g gamma=0.25g concu delta dis.png}
        \caption{$\chi=5g$}
        \label{fig4:concu detunning 1 x2}
    \end{subfigure}
    \vfill
    \begin{subfigure}{0.49\textwidth}
        \includegraphics[width=\textwidth]{figuras/ch4/concu/delta/eg1+ge1 k=0.5g x=0.0g J=0.0g gamma=0.25g concu delta dis.png}
        \caption{$k-J=0.5g$}
        \label{fig4:concu detunning 1 k1}
    \end{subfigure}
    \hfill
    \begin{subfigure}{0.49\textwidth}
        \includegraphics[width=\textwidth]{figuras/ch4/concu/delta/eg1+ge1 k=2.5g x=0.0g J=0.0g gamma=0.25g concu delta dis.png}
        \caption{$k-J=2.5g$}
        \label{fig4:concu detunning 1 k2}
    \end{subfigure}
    \caption{}
    \label{fig4:concu detunning 1 params}
\end{figure}
En la figura \ref{fig4:concu detunning 1 params} se muestran diferentes casos; en la primera fila se muestran casos con medio Kerr cuyos parametros son $\chi=0.1g$ y $\chi=5g$ respectiamente. El primer caso, no se observa ningun cambio significativo con repecto al caso $\chi=0$, el cambio es muy pequeño para ser notado, y la estructura queda igual. Pero al aumentar mucho la no linealidad del medio, se comienza a observar el desdoblamiento en las energias, y los maximos y minimos locales de energia comienzan a notarse. En la figura \ref{fig4:concu detunning 1 x2} se ve como hay dos zonas en donde las oscilaciones tienen gran amplitud y poca frecuencia, similar al caso \textit{resonante}. Esto se puede explicar con el desdoblamiento de energias, parece que hay dos minimos en las frecuencias de Rabi que participan en este proceso, que parecen estar en $\Delta_1=5g$ y $\Delta_2=15g$ aproximadamente. Haciendo una analogia con el caso de 1 atomo, donde veiamos que la condicion de robustez se daba para $\Delta=\chi(2n-1)$ (ver \ref{sec3:medio kerr}) y recordando que esta condicion sale de la diferencia entre las contribuciones diagonales del Hamiltoniano, entonces en este caso, podemos hacer lo mismo y obtener dos condiciones, la primera se obtiene con $\chi n^2-\chi(n-1)^2=\chi(2n-1)$ que es igual a la del caso anterior, y la segunda con $\chi(n-1)^2-\chi(n-2)^2=\chi(2n-3)$. Sustituyendo $n=2$, obtenemos que la condicion se da para $\Delta=\chi$ y $\Delta=3\chi$, que es justo donde se obtienen los minimos. Si bien el segundo minimo de $3\chi$ parece ser menos pronunciado, esta analogia funciona bien.

Si hacemos esta resta pero ahora con $\chi=0$ y nos concentramos en $k-J$, vemos que la condicion se da cuando $\Delta=\pm 2(k-J)$, que es justo lo que observamos en la figura \ref{fig4:concu detunning 1 k2}. En general entonces, podemos decir que esta condicion de menor frecuencia y mayor amplitud de oscilacion se da cuando las energias de los estados estan cerca de la degeneracion, y como tenemos 3 estados que interactuan siempre por medio del estado del medio que es el $\ket{egn+gen}$, entonces puede haber una degeneracion entre el primero y segundo estado, o entre el segundo y tercero, que se dan en el caso general cuando

\begin{equation}
    \Delta-\chi(2n-1)+2(k-J)=0
    \label{ec4:condicion 1}
\end{equation}
que se corresponde con la degeneracion entre las energias del estado $\ket{ggn}$ y $(\ket{eg,n-1}+\ket{ge,n-1})/\sqrt{2}$, y en el otro caso
\begin{equation}
    \Delta-\chi(2n-3)-2(k-J)=0
    \label{ec4:condicion 2}
\end{equation}
que se corresponde con los estados $(\ket{eg,n-1}+\ket{ge,n-1})/\sqrt{2}$ y $\ket{ee,n-2}$. Tambien, se puede dar que los tres estados estan degenerados si se cumplen las dos condiciones simultaneamente.
Algunas expresiones que surgen de pedir que los tres estados sean degenerados son sencillas: $\Delta+2\chi(n-1)=0$ o $\chi-2(k-J)=0$.

\subsubsection{\underline{Condicion inicial $\ket{ee0+gg2}$}}
Esta condicion inicial no tiene entrelazamiento incial, ya que al tomar traza parcial sobre la cavidad, vemos como uno de los estados de la superposicion tiene 0 fotones, y el otro 2, entonces el resultado de trazar sobre la cavidad en el instante inicial es que los atomos se encuentren en el estado maximamente mixto $\frac{1}{2}(\ketbra{ee}{ee}+\ketbra{gg}{gg})$, que no es entrelazado.

\subsection{Dependencia con el medio $\chi$}
\subsubsection{\underline{Condicion inicial $\ket{eg0+ge0}$}}
En la figura \ref{fig4:concu x 0} se muestra la dinamica de entrelazamiento en funcion de la no linealidad del medio. No hay difrencias mayores con la dependencia en $\Delta$ (fig \ref{fig4:concu detunning 0}).
\begin{figure}[h!]
    \centering
    \begin{subfigure}{0.49\textwidth}
        \includegraphics[width=\textwidth]{figuras/ch4/concu/chi/eg0+ge0 d=0.0g k=0.0g J=0.0g gamma=0.25g concu chi uni.png}
        \caption{Dinámica sin perdidas}
        \label{fig4:concu x 0 uni}
    \end{subfigure}
    \hfill
    \begin{subfigure}{0.49\textwidth}
        \includegraphics[width=\textwidth]{figuras/ch4/concu/chi/eg0+ge0 d=0.0g k=0.0g J=0.0g gamma=0.25g concu chi dis.png}
        \caption{Dinámica con perdidas}
        \label{fig4:concu x 0 dis}
    \end{subfigure}
    \caption{Dinámica de entrelazamiento para el estado inicial $\ket{eg0+ge0}$, en función del detunning, y para $\chi=k-J=0$.}
    \label{fig4:concu x 0}
\end{figure}

Esto nos reafirma que en el espacio de $N=1$, el medio Kerr es muy similar al detunning. 

\begin{figure}[h!]
    \centering
    \begin{subfigure}{0.49\textwidth}
        \includegraphics[width=\textwidth]{figuras/ch4/concu/chi/eg0+ge0 d=1.0g k=0.0g J=0.0g gamma=0.25g concu chi dis.png}
        \caption{$\Delta=1g$}
        \label{fig4:concu x d1}
    \end{subfigure}
    \hfill
    \begin{subfigure}{0.49\textwidth}
        \includegraphics[width=\textwidth]{figuras/ch4/concu/chi/eg0+ge0 d=5.0g k=0.0g J=0.0g gamma=0.25g concu chi dis.png}
        \caption{$\Delta=5g$}
        \label{fig4:concu x d2}
    \end{subfigure}
    \vfill
    \begin{subfigure}{0.49\textwidth}
        \includegraphics[width=\textwidth]{figuras/ch4/concu/chi/eg0+ge0 d=0.0g k=0.5g J=0.0g gamma=0.25g concu chi dis.png}
        \caption{$k-J=0.5g$}
        \label{fig4:concu x k1}
    \end{subfigure}
    \hfill
    \begin{subfigure}{0.49\textwidth}
        \includegraphics[width=\textwidth]{figuras/ch4/concu/chi/eg0+ge0 d=0.0g k=2.5g J=0.0g gamma=0.25g concu chi dis.png}
        \caption{$k-J=2.5g$}
        \label{fig4:concu x k2}
    \end{subfigure}
    \caption{Entrelazamiento $\ket{eg0+ge0}$}
    \label{fig4:concu x params}
\end{figure}
\newpage
\subsubsection{\underline{Condicion inicial $\ket{eg1+ge1}$}}
Para la condicion inicial $\ket{eg1+ge1}$, vemos que hay una diferencia en comparacion con el caso del detunning: comparando las imagenes \ref{fig4:concu x 1} y \ref{fig4:concu detunning 1} vemos que hay una diferencia en la estructura. Esto significa que ya no estan en igualdad de condiciones el detunning y el medio, como era el caso de el subespacio de $N=1$.
\begin{figure}[h!]
    \centering
    \begin{subfigure}{0.49\textwidth}
        \includegraphics[width=\textwidth]{figuras/ch4/concu/chi/eg1+ge1 d=0.0g k=0.0g J=0.0g gamma=0.25g concu chi uni.png}
        \caption{Dinámica sin perdidas}
        \label{fig4:concu x 1 uni}
    \end{subfigure}
    \hfill
    \begin{subfigure}{0.49\textwidth}
        \includegraphics[width=\textwidth]{figuras/ch4/concu/chi/eg1+ge1 d=0.0g k=0.0g J=0.0g gamma=0.25g concu chi dis.png}
        \caption{Dinámica con perdidas}
        \label{fig4:concu x 1 dis}
    \end{subfigure}
    \caption{Dinámica de entrelazamiento para el estado inicial $\ket{eg1+ge1}$, en función del medio Kerr, y para $\Delta=k-J=0$.}
    \label{fig4:concu x 1}
\end{figure}
Aparte de esto, los demas comportamientos no son inesperados. Al aumentar $\chi$, el entrelazamiento inicial se conserva ya que las oscilaciones son de menor amplitud; ademas, al agregar disipacion, las oscilaciones decaen en amplitud y las zonas donde hay presente SDE se agrandan. Tambien, al aumentar la no linealidad del medio, la frecuencia de oscilacion aumenta haciendo que el entrelazamiento sea mas robusto ante el efecto del entorno.
\begin{figure}[h!]
    \centering
    \begin{subfigure}{0.49\textwidth}
        \includegraphics[width=\textwidth]{figuras/ch4/concu/chi/eg1+ge1 d=1.0g k=0.0g J=0.0g gamma=0.25g concu chi dis.png}
        \caption{$\Delta=1g$}
        \label{fig4:concu x 1 d1}
    \end{subfigure}
    \hfill
    \begin{subfigure}{0.49\textwidth}
        \includegraphics[width=\textwidth]{figuras/ch4/concu/chi/eg1+ge1 d=5.0g k=0.0g J=0.0g gamma=0.25g concu chi dis.png}
        \caption{$\Delta=5g$}
        \label{fig4:concu x 1 d2}
    \end{subfigure}
    \vfill
    \begin{subfigure}{0.49\textwidth}
        \includegraphics[width=\textwidth]{figuras/ch4/concu/chi/eg1+ge1 d=0.0g k=0.5g J=0.0g gamma=0.25g concu chi dis.png}
        \caption{$k-J=0.5g$}
        \label{fig4:concu x 1 k1}
    \end{subfigure}
    \hfill
    \begin{subfigure}{0.49\textwidth}
        \includegraphics[width=\textwidth]{figuras/ch4/concu/chi/eg1+ge1 d=0.0g k=2.5g J=0.0g gamma=0.25g concu chi dis.png}
        \caption{$k-J=2.5g$}
        \label{fig4:concu x 1 k2}
    \end{subfigure}
    \caption{Dinamica de entrelazamiento para $\ket{eg1+ge1}$ en funcion del medio Kerr, para diferentes valores de $\Delta$ y $k-J$.}
    \label{fig4:concu x params 1}
\end{figure}
Al cambiar algunos de los parametros para ver como afectan el entrelazamiento en funcion de $\chi$, se muestran en las figuras \ref{fig4:concu x 1 d1} y \ref{fig4:concu x 1 d2} como afecta aumentar el detunning. En primer lugar, aumentar a $\Delta=g$ no afecta sustancialmente la estructura (comparar con \ref{fig4:concu x 1 dis}), el efecto principal es desplazar hacia arriba la figura. Pero al aumentar mucho el detunning como en el caso de $\Delta=5g$, observamos que no solo que la frecuencia cambia (poco pero cambia, en el primer caso para llegar a $t/T_0=2$ se necesitan 3 oscilaciones pero para el segundo 4), como se esperaba, sino que tambien hay nuevametne una separacion en dos regiones. Esto nuevamente se atribuye a que la diferencia de energias tiene maximos y minimos locales, haciendo que en estas regiones las oscilaciones sean de mayor amplitud. Es interesante que ahora, si se observan las figuras \ref{fig4:concu x 1 k1} y \ref{fig4:concu x 1 k2}, por un lado, \textcolor{red}{parace que la disipacion afecta menos, y por otro lado no ocurre la separacion en dos regiones al aumentar la interaccion $k-J=2.5g$. Esto es interesante, ya que parece que aumentar la interaccion entre los atmos hace que el entrelazamiento sea mas duradero ante el efecto del entorno. A este punto se volvera mas adelante}.
\subsubsection{\underline{Condicion inicial $\ket{ee0+gg2}$}}

\newpage
\subsection{Dependencia con la interaccion entre atomos $k-J$}
Ahora nos concentraremos en la interaccion entre los atomos. A diferencia de los otros dos parametros, esta interaccion solo depende los dos atomos y por lo tanto se esperan comportamientos diferentes. 
\subsubsection{\underline{Condicion inicial $\ket{eg0+ge0}$}}
\begin{figure}[h!]
    \centering
    \begin{subfigure}{0.49\textwidth}
        \includegraphics[width=\textwidth]{figuras/ch4/concu/k/eg0+ge0 d=0.0g x=0.0g J=15.0g gamma=0.25g concu k uni.png}
        \caption{Dinámica sin perdidas}
        \label{fig4:concu k 0 uni}
    \end{subfigure}
    \hfill
    \begin{subfigure}{0.49\textwidth}
        \includegraphics[width=\textwidth]{figuras/ch4/concu/k/eg0+ge0 d=0.0g x=0.0g J=15.0g gamma=0.25g concu k dis.png}
        \caption{Dinámica con perdidas}
        \label{fig4:concu k 0 dis}
    \end{subfigure}
    \caption{Dinámica de entrelazamiento para el estado inicial $\ket{eg0+ge0}$, en función del detunning, y para $\chi=k-J=0$.}
    \label{fig4:concu k 0}
\end{figure}
Desde el comienzo ya se puede ver como al aumentar el valor absoluto de la interaccion el comportamiento es el mismo que en los otros dos casos, donde aumenta la frecuecnia y disminuye la amplitud conservando el entrelazamiento inicial, pero en es te caso este comportemianto se ve acentuado. 


\begin{figure}[h!]
    \centering
    \begin{subfigure}{0.49\textwidth}
        \includegraphics[width=\textwidth]{figuras/ch4/concu/k/eg0+ge0 d=1.0g x=0.0g J=15.0g gamma=0.25g concu k dis.png}
        \caption{$\Delta=1g$}
        \label{fig4:concu k d1}
    \end{subfigure}
    \hfill
    \begin{subfigure}{0.49\textwidth}
        \includegraphics[width=\textwidth]{figuras/ch4/concu/k/eg0+ge0 d=5.0g x=0.0g J=15.0g gamma=0.25g concu k dis.png}
        \caption{$\Delta=5g$}
        \label{fig4:concu k d2}
    \end{subfigure}
    \vfill
    \begin{subfigure}{0.49\textwidth}
        \includegraphics[width=\textwidth]{figuras/ch4/concu/k/eg0+ge0 d=0.0g x=0.5g J=15.0g gamma=0.25g concu k dis.png}
        \caption{$\chi=0.5g$}
        \label{fig4:concu k x1}
    \end{subfigure}
    \hfill
    \begin{subfigure}{0.49\textwidth}
        \includegraphics[width=\textwidth]{figuras/ch4/concu/k/eg0+ge0 d=0.0g x=5.0g J=15.0g gamma=0.25g concu k dis.png}
        \caption{$\chi=5g$}
        \label{fig4:concu k x2}
    \end{subfigure}
    \caption{}
    \label{fig4:concu k params}
\end{figure}

La interaccion entre los atomos conserva mejor el entrelazamiento. Esto tambien se observa cuando se agrega la disipacion, en comparacion con las otras interacciones, en la figura \ref{fig4:concu k 0 uni} se ve como se conserva mejor el entrelazamiento.

En la figura \ref{fig4:concu k params} se observan los efectos que tienen los otros dos parametros en la dinamica. Por un lado, vemos que aumentar el detunning \ref{fig4:concu k d1} desplaza el centro hacia abajo pero no parece haber mucha diferencia. Al seguir aumentando la desintonia \ref{fig4:concu k d2}, el centro se sigue desplazando y podemos ver que este se desplaza la mitad que lo que aumenta $\Delta$. Ademas, la estructura pierde la simetria y se observa una franja en la parte duperior donde el entrelazamiento se conserva mejor. Este valor parece corresponderse con $+\Delta/2$. Esto puede deberse a la condicion que se encontro anteriormente (\ref{ec4:condicion 1} y \ref{ec4:condicion 2}) que en este caso se resume en que la condicion de degeneracion se de para $k-J=\pm\Delta/2$. Es interesante que en este caso una de las condiciones presenta zonas de SDE, y la otra en cambio conserva mejor el entrelazamiento que las demas zonas. Por otro lado, al cambiar las no linealidades, vemos que el efecto es exactamente el mismo, pero el desplazamiento es ahora en el otro sentido, y las condiciones de degeneracion siguen siendo las mismas, a menos de un signo.
\subsubsection{\underline{Condicion inicial $\ket{eg1+ge1}$}}
Al considerar el espacio de $N=2$, lo primero que resalta es que la zona central que presenta SDE ahora es mas ancha. Por otro lado, al agragar disipacion, las diferencias parecen desaparecer con respecto a su contraparte de $N=1$. Esto probablemente se debe a que en presencia de disipacion, para tiempos largos, el estado del sistema sera similar. Mas aun considerando que se esta ignorando la dinamica de la cavidad al observar el entrelazamiento entre los atomos.
\begin{figure}[h!]
    \centering
    \begin{subfigure}{0.49\textwidth}
        \includegraphics[width=\textwidth]{figuras/ch4/concu/k/eg1+ge1 d=0.0g x=0.0g J=15.0g gamma=0.25g concu k uni.png}
        \caption{Dinámica sin perdidas}
        \label{fig4:concu k 1 uni}
    \end{subfigure}
    \hfill
    \begin{subfigure}{0.49\textwidth}
        \includegraphics[width=\textwidth]{figuras/ch4/concu/k/eg1+ge1 d=0.0g x=0.0g J=15.0g gamma=0.25g concu k dis.png}
        \caption{Dinámica con perdidas}
        \label{fig4:concu k 1 dis}
    \end{subfigure}
    \caption{Dinámica de entrelazamiento para el estado inicial $\ket{eg0+ge0}$, en función del detunning, y para $\chi=k-J=0$.}
    \label{fig4:concu k 1}
\end{figure}
La mayor diferencia entre ambos casos aparece cuando observamos los efectos de los parametros $\Delta$ y $\chi$. Este caso es interesante porque nos ayuda a ver concretamente como, al tener ahora 3 estados dinamicos, el efecto del detunning y del medio son diferentes, que en el caso de 1 atomo (o en el subespacio de $N=1$) por lo que se vio antes son lo mismo. En la figura \ref{fig4:concu k params 1}\ref{sub@fig4:concu k 1 d2} y \ref{sub@fig4:concu k 1 x2} muestra la clara diferencia entre estos dos efectos. Para el detunning aparecen 2 regiones, que nuevamente son correctamente predichas por las condiciones \ref{ec4:condicion 1} y \ref{ec4:condicion 2} (considerando $\chi=0$), y para el caso del medio (panel \ref{sub@fig4:concu k 1 x2}) si traducimos las condiciones para $\Delta=0$ obtenemos que la condicion de degeneracion se da para $k-J=\frac{3}{2}\chi=7.5g$ y $k-J=\frac{-\chi}{2}=-2.5g$, que es justamente donde se observan las regiones con mayor estructura.
Esto nos dice que fundamentalmente, la degeneracion de estados lleva a una mayor coherencia entre las oscilaciones del entrelazamiento. Si bien esto es cierto, puede ser que este entrelazamiento muera subitamente, o que sea mas robusto que otras combinaciones de parametros. Esto se ve claramente en estas dos figuras (paneles \ref{sub@fig4:concu k 1 d2} y \ref{sub@fig4:concu k 1 x2}), donde se pudo predecir correctamente que habia zonas mas coherentes, pero en ambos casos tenemos una de las zonas que presenta SDE y el entrelazamiento muere, y la segunda zona donde las oscilaciones perduran en el tiempo mas que en cualquier otra combinacion de parametros. 
\begin{figure}[h!]
    \centering
    \begin{subfigure}{0.49\textwidth}
        \includegraphics[width=\textwidth]{figuras/ch4/concu/k/eg1+ge1 d=1.0g x=0.0g J=15.0g gamma=0.25g concu k dis.png}
        \caption{$\Delta=1g$}
        \label{fig4:concu k 1 d1}
    \end{subfigure}
    \hfill
    \begin{subfigure}{0.49\textwidth}
        \includegraphics[width=\textwidth]{figuras/ch4/concu/k/eg1+ge1 d=5.0g x=0.0g J=15.0g gamma=0.25g concu k dis.png}
        \caption{$\Delta=5g$}
        \label{fig4:concu k 1 d2}
    \end{subfigure}
    \vfill
    \begin{subfigure}{0.49\textwidth}
        \includegraphics[width=\textwidth]{figuras/ch4/concu/k/eg1+ge1 d=0.0g x=0.5g J=15.0g gamma=0.25g concu k dis.png}
        \caption{$\chi=0.5g$}
        \label{fig4:concu k 1 x1}
    \end{subfigure}
    \hfill
    \begin{subfigure}{0.49\textwidth}
        \includegraphics[width=\textwidth]{figuras/ch4/concu/k/eg1+ge1 d=0.0g x=5.0g J=15.0g gamma=0.25g concu k dis.png}
        \caption{$\chi=5g$}
        \label{fig4:concu k 1 x2}
    \end{subfigure}
    \caption{}
    \label{fig4:concu k params 1}
\end{figure}
\subsubsection{\underline{Condicion inicial $\ket{3er0}$}}


\subsection{Acoplamiento Buck-Sukumar}
Se repitio el estudio utilizando el acoplamiento no lineal entre la cavidad y los atomos, este acoplamiento es mayor mientras mas excitaciones tenga la cavidad, concretamente va como $\sqrt{n_C}$. 
La conclusion de este estudio es que en el caso de $N=1$, es todo exactamente igual. Para $N=2$, hay algunas pequeñas diferencias, pero son casi insignificantes y no tienen consecuencias mayores sobre el entrelazamiento de los atomos. 

Por lo tanto, no justifica mostrar nada, solo mencionar que en los casos observados no afecta en casi nada.

\section{Conclusiones del capitulo 4}
En este capitulo se analizo la dinamica del sistema, primero recuperamos el caso de 1 atomo \ref{sec4:dinamica apantallamiento} apantallando uno de los dos atomos, situacion que no es fisica pero sirve tambien para estudiar el efecto de los parametros del problema \ref{sec4:dinamica sin apantallamiento}. Luego nos concentramos en la dinamica de entrelazamiento entre los dos atomos, ignorando la cavidad. Este estudio resulto util, ya que por un lado se concluye que el subespacio de $N=1$ se comporta igual que el modelo de Jaynes-Cummings de 1 atomo, y donde se observo que el detunning $\Delta$ y la no linealidad del medio Kerr $\chi$ juegan un mismo rol. Pero para el caso de $N=2$ esto ya no es cierto, y encontramos una dinamica con mucha mas estructura. Ademas, se pudo utilizar una analogia que nos llevo a estudiar los casos de degeneracion energetica entre estados, y estas condiciones nos sirvieron para predecir correctamente zonas de entrelazamiento particulares, donde la combinacion particular de estos parametros daba lugar a estructura en la figura. Vimos que si bien estas condiciones logran predecir, no se encontro una herramienta concreta para predecir si esta zona presentara muerte del enterlazamiento, o robustez ante el efecto del entorno. Ademas, se concluye que de los parametros del problema ($\Delta$, $\chi$ y $k-J$) aumentar la interaccion entre los atomos es la mas efectiva para presevar el entrelazamiento, por su contrario, aumentar las no linealidades ($\chi$) hace que el entrelazamiento muera mas rapidamente. Esto nos dice que nuestra intuicion fisica sobre estos sistemas esta correcta. Las no linealidades del medio son malas, y aumentar la interaccion entre los atomos aumenta el entrelazamiento entre estos. Para cerrar entonces, observemos una ultima situacion, que podria ser en un experimento real.

