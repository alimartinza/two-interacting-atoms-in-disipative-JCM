\chapter{Fase Geometrica}
\label{ch:fg}

%CAMBIAR ESTO PARA PERSONALIZARLO A MI GUSTO
\pagestyle{fancy}
\fancyhf{}
\fancyhead[LE]{\nouppercase{\rightmark\hfill}}
\fancyhead[RO]{\nouppercase{\leftmark\hfill}}
\fancyfoot[LE,RO]{\hfill\thepage\hfill}

Este cap\'itulo presenta el objeto de estudio del trabajo. La fase geometrica es un observable que promete en el ambito de la informaci\'on cu\'antica, ya que como se ver\'a m\'as adelante, recupera informaci\'on sobre la trayectoria del sistema en el espacio de Hilbert, y en algunos casos se observa que est\'a relacionado con el entrelazamiento. 
El cap\'itulo esta estructurado de manera que en primer lugar se tratar\'a una descripción general de las fases geom\'etricas (FG) en el contexto de sistemas aislados, descritos consecuentemente mediante estados puros. Analizar este caso antes de centrar la antención en sistemas cuanticos abiertos permitira asilimar nociones y ganar intuición sobre las fases geométricas en el marco de una teoría formalmente más simple. A lo largo del capítulo se trabajaran expresiones validas bajo ciertas hipotesis, partiendo del caso menos general, y llegando al caso más general conocido hasta el momento, aunque la aplicación de las fases geometricas a sistemas abiertos no llego a un consenso unánime. Por lo tanto, al final del capitulo se presentará una propuesta particular, la cual se usará en los próximos capitulos.

\section{R\'egimen adiab\'atico y fase de Berry} \label{sec2:adiabatico}
La fase de berry \ref{} ludmi 1 es un fenomeno fundamental relacionado con el teorema adiab\'atico. Esta representa la fase acumulada pot el autoestado de un Hamiltoniano $H(t)$ que var\'ia lentamente en un ciclo, que esta relacionada con el circuito descrito por $H(t)$ en un dado espacio de par\'ametros. \newline
Para ver esto, se considera un Hamiltoniano $H(R(t))$ que depende explcitamente del tiempo a travez de un parámetro $R=(R_1,R_2,\dots)$. Dado esta Hamiltoniano, formalmente se pueden encontrar los autoestados instantaneos del sistema $\ket{\psi_n(R(t))}$ que satisfacen
\begin{equation}
    H(R(t))\ket{\psi_n(R(t))}=E_n(R(t))\ket{\psi_n(R(t))}
\end{equation}
, suponiendo ademas que los autovalores satisfacen $E_1<E_2<\dots$ de forma que no hay degeneraci\'on. Se considera que la evoluci\'on temporal de un estado cualquiera $\ket{\psi(t)}$ esta dada por la ecuación de Schr\"odinger
\begin{equation}
    i \hbar \ket{\dot \psi(t)}=H(R(t))\ket{\psi(t)}
\end{equation}
Desarrollando el estado en funci\'on de los autoestados instantaneos del Hamiltoniano, se puede resolver formalmente el problema
\begin{equation}
    \ket{\psi(t)} = \sum_n c_n(t)\ket{\psi_n(R(t))}
\end{equation}

los coeficientes \( c_n(t) \) satisfacen:
\[
i \hbar \dot{c}_n(t) = \left( E_n - i \hbar \langle \psi_n | \dot{\psi}_n \rangle \right) c_n(t) - i \hbar \sum_{m \neq n} \langle \psi_n | \dot{\psi}_m \rangle c_m(t).
\]

En el régimen adiabático, donde el Hamiltoniano cambia lentamente en comparación con las escalas internas del sistema, se desprecia el término de acoplamiento cruzado:
\[
\dot{c}_n(t) \approx -\frac{i}{\hbar} \left( E_n - i \hbar \langle \psi_n | \dot{\psi}_n \rangle \right) c_n(t).
\]

El estado resultante es:
\[
| \psi(t) \rangle = e^{-\frac{i}{\hbar} \int_0^t E_n(R(t')) \, dt'} e^{i \phi_n(t)} | \psi_n(R(t)) \rangle,
\]
donde \( \phi_n(t) = i \int_0^t \langle \psi_n(R(t')) | \nabla_R | \psi_n(R(t')) \rangle \cdot \dot{R}(t') \, dt' \) es la fase geométrica acumulada.

Para circuitos cerrados en el espacio de parámetros, la fase geométrica se expresa como:
\begin{equation}\label{eq2:fg berry}
    \phi_n(C) = i \oint_C \langle \psi_n(R) | \nabla_R | \psi_n(R) \rangle \cdot dR,    
\end{equation}

independiente de la velocidad con que se recorre el circuito. Sin embargo, la hipotesis de este resultado es que la veolcidad de la evoluci\'on sea suficientemente lenta para que se puedan despreciar las transiciones no adiab\'aticas a otros niveles de energ\'ia, por lo tanto este resultado no es totalmente independiente de la velocidad con la que se recorre el circuito en el espacio de parámetros.

\section{Fase de Aharonov-Anandan}\label{sec2:fase AA}

La formulación de Aharonov y Anandan permite definir una fase geométrica que es independiente de la evolución adiabática. Su propuesta se basa únicamente en la trayectoria del estado en el espacio proyectivo de rayos, sin referencia explícita al Hamiltoniano.

Considérese el espacio de Hilbert \( H \), y dentro de este, el subespacio \( N_0 \) que contiene vectores normalizados \( | \psi \rangle \). El espacio proyectivo \( P \) se define como el conjunto de clases de equivalencia bajo la relación \( | \psi \rangle \sim e^{i\alpha} | \psi \rangle \), estas colecciones $\xi = \{e^{i\alpha}\ket{\psi} \; ; \; 0 \leq \alpha 2\pi\}$ denominadas rayos, agrupan en un \'unico elemento (la clase) todos los objetos equivalentes. Cada clase de equivalencia se denomina un rayo, y el mapeo \( \Pi : N_0 \to P \) proyecta un vector al rayo correspondiente.

Durante una evolución cíclica, el estado al tiempo inicial \( | \psi(0) \rangle \) y al tiempo final \( | \psi(T) \rangle \) pertenecen al mismo rayo, por lo que:
\[
| \psi(T) \rangle = e^{i\phi} | \psi(0) \rangle.
\]
los estados solo pueden diferir en una fase total $\phi$. Para determinar la fase geométrica, se descompone \( \phi \) en dos contribuciones: una parte dinámica y una parte geométrica.

La relación entre el estado físico \( | \psi(t) \rangle \) y su clase de equivalencia \( \xi \in P \) se escribe como:
\[
| \psi(t) \rangle = e^{i f(t)} | \xi(t) \rangle,
\]
donde \( f(t) \) es una función que recoge la fase acumulada. Sustituyendo esta relación en la ecuación de Schrödinger:
\[
i \hbar \frac{\partial}{\partial t} | \psi(t) \rangle = H | \psi(t) \rangle,
\]
se obtiene una ecuación para \( f(t) \):
\[
\hbar \dot{f}(t) = -\langle \xi(t) | H | \xi(t) \rangle + i \hbar \langle \xi(t) | \dot{\xi}(t) \rangle.
\]

La fase total acumulada entre los tiempos \( 0 \) y \( T \) es:
\[
\phi = f(T) - f(0) = -\frac{1}{\hbar} \int_0^T \langle \xi(t) | H | \xi(t) \rangle \, dt + \int_0^T i \langle \xi(t) | \dot{\xi}(t) \rangle \, dt.
\]

Aquí, el primer término es la fase dinámica:
\[
\phi_{\text{din}} = -\frac{1}{\hbar} \int_0^T \langle \xi(t) | H | \xi(t) \rangle \, dt = -\frac{1}{\hbar}\int_0^T dt \, \bra{\psi(t)}H\ket{\psi(t)},
\]
y el segundo término corresponde a la fase geométrica:
\begin{equation} \label{eq:fg AA}
    \phi_{\text{AA}} = \int_0^T i \langle \xi(t) | \dot{\xi}(t) \rangle \, dt.
\end{equation}

Esta última expresión muestra que la fase geométrica depende únicamente de la trayectoria en el espacio proyectivo \( P \) y no del Hamiltoniano o la velocidad de evolución. Al ser independiente de estos factores, refleja una propiedad puramente geométrica de la curva trazada por el estado en \( P \).



\subsection{Interpretación Geométrica y caso no-cíclico}
En esta sección se mostrará la interpretación geométrica y la generalización al caso no cíclico, demostrada por Samuel y Bhandari \cite{}ludim 4. Esta definición no requiere de la condición de ciclo cerrado, y tampoco requiere que el estado conserve su norma, como por ejemplo en una medición y colapso de la función de onda. Para esto es necesario dotar al espacio de Hilbert de geometría donde entonces la fase surge de la estructura del espacio. \newline
Para darle estructura al espacio, lo que ya hicimos antes es considerar un fibrado, donde definimos una clase de equivalencia para estados que difieren en una fase global. Para darle mayor estructura tenemos que introducir el concepto de conexión, que nos permitirá comparar elementos pertenecientes a fibras distintas mediante una regla de transporte paralelo. Esta regla nos dice que
\begin{equation} \label{eq2:transporte paralelo}
    \text{Im} \braket{\psi(t)}{\dot \psi(t)}=0
\end{equation}

Considérese una curva \( C: t \in [0, T] \to \ket{\psi(t)} \) sobre \( N_0 \), horizontal, y su vector tangente $\ket{\dot{\psi}(t)}/\braket{\psi(t)}{\psi(t)}$. La conexión natural
\begin{equation}
A = \frac{\text{Im} \bra{\psi(t)} \dot{\psi}(t) \rangle}{\bra{\psi(t)} \psi(t) \rangle},
\end{equation}
transforma, frente a transformaciónes \( U(1) \) de gauge \( \ket{\psi(t)} \to e^{i\alpha(t)} \ket{\psi(t)} \), según
\begin{equation} \label{eq2:transformación de gauge}
A \to A + \dot{\alpha}(t).
\end{equation}

Dado que \( C \) es horizontal por definición, la ley de transporte paralelo de la ecuación \ref{eq2:transporte paralelo} impone que la conexión se anule a lo largo de la trayectoria del estado que le da origen. Si el vector de estado \( \ket{\psi(t)} \) está, además, asociado a una evolución cíclica en el sentido de Aharonov-Anandan, entonces retorna al rayo inicial en algún instante \( T \).

Considérese, en este escenario, la integral de la conexión \( A \) sobre el camino construido a partir de la curva \( \ket{\psi(t)} ; t \in [0, T] \), cerrada uniendo \( \ket{\psi(T)} \) con \( \ket{\psi(0)} \) sobre el rayo. Como se ha discutido, la curva \( \ket{\psi(t)} \) es horizontal por definición y, por lo tanto, la conexión se anula \( A = 0 \) sobre ella. Por otra parte, la integral sobre el tramo vertical que cierra el camino da como resultado la diferencia de fase entre \( \ket{\psi(T)} \) y \( \ket{\psi(0)} \):
\begin{equation}
\oint A dl_{N_0} = \int_C A + \int_{\text{rayo}} A = \text{arg} \bra{\psi(0)} \psi(T) \rangle.
\end{equation}

Es decir, la integral sobre el camino total (cerrado), es la diferencia de fase total entre el estado inicial y final. Por otra parte, la integral de la conexión \( A \) sobre una curva cerrada en \( N_0 \) es invariante por efecto de la ley de transformación \ref{eq2:transformación de gauge}. La holonomía de la curva \( C \subset P \) asociada a la conexión \( A \) es entonces:
\begin{equation}
g(C) = e^{i \oint_C A} = e^{i\phi_{\text{AA}}}.
\end{equation}
En el caso de una evolución no cíclica, el vector que describe el sistema no vuelve a su rayo de partida. Para este caso se establece una manera de comparar estados de diferentes fibras. Dicha comparación se hace a travez de la fase de \textit{Pancharatnam} \ref{}66 ludmi, definida para dos estados no-ortogonales cualesquiera como
\begin{equation}
    \phi_P = \arg \braket{\psi_1}{\psi_2}
\end{equation}
Para hacer la generalización al caso no-ciclico, tenemos que dar un concepto de distancia, y para esto tenemos que hablar de lineas geodesicas. No vamos a meternos en detalle en esto, pero lo importante es que la fase en el caso no ciclico consiste de la diferencia entre la fase dinámica y la dase de Pancharatnam
\begin{equation}
    \psi_{SB}=-\psi_P-\frac{1}{\hbar}\int_0^Tdt\bra{\psi(t)}H\ket{\psi(t)}
\end{equation}
Este método se puede utilizar para generalizar al caso no unitario, en el sentido de un estado puro que no conserva su norma. Este tipo de evolución puede suceder cuanto estamos teniendo en cuenta mediciones en el sistema, colapsos de la función de onda no conservan la norma segun la regla de colapso de la mecánica cuántica. En este caso, si consideramos el estado inicial $\ket{\psi_0}$ sobre el cual se realizan mediciones sucesivas, de forma tal que la N-esima proyección es otra vez al estado inicial, el estado funal del sistema esta dado por
\begin{equation}
    \ket{\psi_0}\braket{\psi_0}{\psi_{N-1}}\dots\braket{\psi_2}{\psi_1}\braket{\psi_1}{\psi_0}.
\end{equation}
Según el criterio de Pancharatnam los estados inicial y final tienen una diferencia de fase bien definida, dado por el argumento del número complejo que acompaña al estado $\ket{\psi_0}$.


\section{Enfoque Cinemático}\label{sec2:cinematico}

En la mayoría de las discusiones sobre la fase geométrica, el punto de partida es la ecuación de Schrödinger para algún sistema cuántico particular caracterizado por un dado Hamiltoniano. Sin embargo, la fase geométrica es consecuencia de la cinemática cuántica, esto es, independiente del detalle respecto del origen dinámico de la trayectoria descrita en el espacio de estados físicos. Mukunda y Simon \ref{} 5 y 67 ludmi resaltaron la independencia de la fase geométrica respecto del origen dinámico de la evolución proponiendo un enfoque cinemático en el cual la trayectoria descrita en el espacio de estados físicos es el concepto fundamental para la fase geométrica. En su desarrollo, se parte de la consideración de una curva uniparamétrica y suave \( C \subset N_0 \), conformada por una dada secuencia de estados \( \ket{\psi(t)} \):
\begin{equation}
C = \{ \ket{\psi(t)} \in N_0 \mid t \in [0, T] \subset \mathbb{R} \},
\end{equation}
donde no se hace ninguna suposición respecto de si \( C \) es una curva abierta o cerrada, ni del origen dinámico de la secuencia de estados. Se observa luego detenidamente la cantidad \( \bra{\psi(t)} \dot{\psi}(t) \rangle \) construida a partir de esta curva. La condición de unitariedad implica que esta cantidad sea imaginaria pura, lo que puede escribirse como
\begin{equation}
\bra{\psi(t)} \dot{\psi}(t) \rangle = i \, \text{Im} \bra{\psi(t)} \dot{\psi}(t) \rangle.
\end{equation}

Por otra parte, aplicando una transformación \( U(1) \) de gauge
\begin{equation} \label{eq2:transformacion u1}
C \to C': \ket{\psi'(t)} = e^{i\alpha(t)} \ket{\psi(t)}, \quad t \in [0, T],
\end{equation}
la cantidad analizada transforma según
\begin{equation}
    \text{Im} \bra{\psi(t)} \dot{\psi}(t) \rangle \rightarrow  \text{Im} \bra{\psi(t)} \dot{\psi}(t) \rangle + \dot{\alpha}(t).
\end{equation}

Lo que queremos conseguir es una funcional que sea invariante ante transformaciónes $U(1)$ \ref{eq2:transformación u1}, es decir, toma mismos valores para curvas $C$ y $C'$
\begin{equation} \label{eq2:fg cinematica unitaria}
    \psi_u[C] \def \arg \braket{\psi(0)}{\psi(T)} - \Im \int_0^T dt \braket{\psi(t)}{\dot \psi(t)}
\end{equation}
Tenemos permitido definir este funcional de la curva $C$ en el espacio de rayos, ya que es invariante ante reparametrizaciones. Algo importante de remarcar es que, si aplicamos una transformación unitaria arbitraria a nuestro estado, entonces al cambiar el Hamiltoniano tambien cambiará la curva que describe el estado inicial en el espacio de Hilbert, y por lo tanto se puede mostrar que la fase geometrica cambia. Por suerte, en el caso que la transformación no depende del tiempo, entonces se demuestra que la fase no cambia. 
\newline
\textcolor{red}{Hasta ahora solo tratamos con sistemas aislados. Antes de pasar a sistemas abiertos, vamos a analizar un ejemplo sencillo utilizando las diferentes definiciones, para ganar intuición y encontrar algunas explicaciones interesantes a comportamientos caracteristicos de este observable.}

\section{Ejemplo de aplicacion: Sistema de dos niveles en un campo magn\'etico}\label{sec2:ejemplos}

\textcolor{red}{LO PONGO O NO LO PONGO?}

\section{Fases geométricas en sistemas abiertos}\label{sec2:sistemas abiertos}
Las secciones anteriores tratan la fase geometrica en diferentes casos, ascendientes en generalidad ya que se logra relajar condiciones e hipotesis, y se llego a una expresion general que satisface propiedades importantes, como invarianza antre transformaciones de fase global $U(1)$ y a reparametrizaciones monotonas, tambien dependen unicamente de la trayectoria descirta por el estado fisico en ele spacio de rayos y no del Hamiltoniano que genera dicha trayectoria, y finalmente son interpretables en terminos puramente geometricos. 

Sin embargo, estamos asumiendo que el estado es puro durante toda su evolucion, restriccion que es una idealizacion y experimentalmente es necesario tener en cuenta que todo sistema fisico esta en contacto con un entorno, y se requiere entonces una descripcion en terminos de estados mixtos y evoluciones no unitarias. La definicion de una fase geometrica que aplique en tal escenario es todavia un problema cuya solucion todavia no llego a un concenso unanime. Muchos esfuerzos notables [\cite{6},\cite{8}-\cite{10}] se concentraron en definir la fase geometrica acumulada por un estado mixto, incluso existen reportes experimentales de detecciones \cite{73}. Otra ruta explirada considera el efecto del entorno como correciones que permitan mantener las nociones de fase geometrica del caso unitartio. Trabajos de este tipo introcucen el efecto del entorno mediante un hamiltoniano no hermicito \cite{32,33}, y otros estudian modificaciones a la fase de Berry por ruido clasico en el campo magnetico \cite{38}, o por un entorno cuantico \cite{39,40}, tanto desde lo teorico como lo experimental \cite{74,75}.

El marco en el cual una fase geometrica para sistemas cuanticos abiertos debe definirse es le siguiente: se suponeq ue el efecto del entorno en el sistema de interes es tal que, bajo aproximaciones adecuadas, el sistema puede tratarse \textit{efectovamente} como un sistema aislado que experimenta un tipo de evolucion lineal no unitaria:
\begin{equation}
    \Sigma:\rho(0)\rightarrow\Sigma_t[\rho(0)] \equiv \rho(t)
\end{equation}
que da cuenta tanto de la dinamica interna del sistema como de su interaccion con el entorno, y satisface una ecuacion maestra.Una consecuencia de este enfoque es que, en el caso general, un estado inicial puro evoluciona en un estado mixto $\rho(t)$. El operador densidad que representa el estado del sistema admite una dscomposiucion $\{ \ket{\psi_k(t)},\omega_k(t)\}$ en estados puros $\ket{\psi_k(t)}$ pesados con probabilidades $\omega_k(t)$, que permite expresarla como
\begin{equation}
    \rho(t)=\sum_k\omega_k(t)\ketbra{\psi_k(t)}{\psi_k(t)}
\end{equation}
. La asociacion $\rho(t)\rightarrow \{ \ket{\psi_k(t)},\omega_k(t)\}$ entre el operador densidad y el \textit{ensamble} de estados $\{\ket{\psi_k(t)}\}$ no es uno-a-uno, sino uno-a-muchos, lo que significa que en general existen diferentes ensambles, con diferentes estados y diferentes pesos, que sin embargo tienen la misma matriz densidad. Esto imposibilita la distincion entre estas situaciones solamente con la informacion que proporciona la matriz densidad.

Una estrategia recurrente entre la literatura que aborda el problema de asociar una fase gemoetrica a un estado mixto $\rho(t)$ es descomponer formalnete la matriz densidad en una mezcla estadistica como la de la ecuacion enterior, y aplicar la fase unitaria \ref{eq2.34} sobre cada elemento de la mezcla para asociar una fase a $\rho(t)$. Esto fue propuesto, desde una descripcion en terminos de operadores de saltos en \cite{32,33} y posteriormente en \cite{34-37}. En una aproximacion diferente al problema, en Tong et al. \cite{31} se propone una definicion de fase geometrica que se vale de una purificacion del estado, pero resulta independiente de la eleccion que se utilize para purificar. La siguiente seccion desarrolla esta propuesta en particular.
\subsection{Enfoque cinematico en sistemas abiuertos}
La introduccion teorica concluye con esta seccion, siguiento la propuesta de Tong et al. \cite{31} para la fase geometrica en sistemas cuanticos abiertos. Para esto, se considera un sistema y el espacio de Hilbert $\mathcal{H}$ de dimension $N$ asociado al mismo. La evolucion del estado puede describirse como una curva $C \subset \mathcal{P}$
\begin{equation}
    C:t\in[0,T] \rightarrow \rho(t) = \sum_{k=1}^N\omega_k(t)\ketbra{\psi_k(t)}{\psi_k(t)}
\end{equation}
donde $\omega_k(t)\geq 0$ y $\ket{\psi_k(t)}$ son los autovalores y autoestados, respectivamente, de la matriz densidad $\rho(t)$ del sistema
\begin{equation}
    \phi_g[C]=\arg \left( \sum_{k=1}^{N} \sqrt{\omega_k(0)\omega_k(t)} \braket{\psi_k(0)}{\psi_k(T)}e^{-\int_{0}^{T}dt\braket{\psi_k(t)}{\dot\psi_k(t)}} \right)
\end{equation}\label{ec2:fg general}

\begin{equation}
    \phi_g[C]=\arg{\braket{\psi(0)}{\psi_+(T)}}-\Im \int_{0}^{T}dt \braket{\psi_+(t)}{\dot\psi_+(t)}
\end{equation}\label{ec2:fg general puro}


