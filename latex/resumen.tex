\newenvironment{abstract}%
{\thispagestyle{empty} \cleardoublepage\null \thispagestyle{empty} \vfill\begin{center}
\bfseries \abstractname \end{center} }%
{\thispagestyle{empty} \vfill\null }


\selectlanguage{spanish}
\begin{abstract}
El modelo de Janyes-Cummings de un \'atomo es un ejemplo paradigm\'atico en la teor\'ia de los fundamentos e informaci\'on cu\'antica, ya que describe de 
manera sencilla la interacci\'on entre fotones y materia de manera puramente cu\'antica. Para extender este modelo, en el presente trabajo se consideran 
dos \'atomos interactuantes, inmersos en una cavidad que presenta no-linearidades y un medio tipo Kerr. En particular, se analiz\'o la dinamica, la 
entropia y otros observables considerando el sistema aislado, y tambien en presencia de decoherencia. Adem\'as, se estudi\'o la fase geometrica en ambos
casos.







\end{abstract}

%\selectlanguage{english}
%\begin{abstract}



%\end{abstract}
\selectlanguage{spanish}



