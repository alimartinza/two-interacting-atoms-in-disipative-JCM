\label{ch:referencias}
\renewcommand\bibname{Referencias}

\bibliographystyle{unsrt}
%\bibliographystyle{ieeetr}
\bibliography{biblio}


\begin{thebibliography}{9}
%cap1 intro

%cap2 geofase

%cap3 jcm 1 atomo

\bibitem{JCoriginal}E. T. Jaynes and F. W. Cummings, "Comparison of quantum and semiclassical radiation theories with application to the beam maser," in Proceedings of the IEEE, vol. 51, no. 1, pp. 89-109, Jan. 1963, doi: 10.1109/PROC.1963.1664.

\bibitem{Haroche2006}Haroche, Serge, and J-M. Raimond. Exploring the quantum: atoms, cavities, and photons. Oxford university press, 2006.

\bibitem{Viotti2022}Viotti, Ludmila, Fernando C. Lombardo, and Paula I. Villar. "Geometric phase in a dissipative Jaynes-Cummings model: Theoretical explanation for resonance robustness." Physical Review A 105.2 (2022): 022218.

\bibitem{Khitrova2006}Khitrova G., Gibbs H., Kira M., Koch S. W., and Scherer A. Nature physics, 2(2):81–90,2006.

\bibitem{Laussy2009}Laussy F. P., Del Valle E., and Tejedor C. Physical Review B, 79(23):235325, 2009.

\bibitem{DelValle2009} Del Valle E., Laussy F. P., and Tejedor C. Physical Review B, 79(23):235326, 2009.

\bibitem{Carmi1989}Carmichael H., Brecha R., Raizen M., Kimble H., and Rice P. Physical Review A, 40(10):5516,1989.

\bibitem{Yamamoto2003}Yamamoto Y., Tassone F., and Cao H. Semiconductor cavity quantum electrodynamics,volume 169. Springer, 2003.

\bibitem{Laussy2008}Laussy F. P., Del Valle E., and Tejedor C. Physical review letters, 101(8):083601, 2008.

\bibitem{Vera2009} Vera C. A., Quesada N., Vinck-Posada H., and Rodríguez B. A. Journal of Physics: Condensed Matter, 21(39):395603, 2009.

\bibitem{Lodhal2015}Lodahl P., Mahmoodian S., and Stobbe S. Reviews of Modern Physics, 87(2):347, 2015.
%cap4 dinamica entrelazamiento
\bibitem{Santos2016}O. de los Santos-Sánchez, C. González-Gutiérrez, J. Récamier. Nonlinear Jaynes-Cummings model for two interacting two-level atoms. arXiv:1607.03216 [quant-ph]

\bibitem{Plenio2006}Martin B. Plenio and S. Virmani. An introduction to entanglement measures. ArXiv quant-ph/0504163, 2016. doi: https://doi.org/10.48550/arXiv.quant-ph/0504163

%cap5 fg

%unused
\bibitem{Berry1984}Berry, Michael Victor. "Quantal phase factors accompanying adiabatic changes." Proceedings of the Royal Society of London. A. Mathematical and Physical Sciences 392.1802 (1984): 45-57.

\bibitem{Anandan1992}Anandan, Jeeva. "The geometric phase." Nature 360.6402 (1992): 307-313.

\bibitem{Tomita1986}Tomita, Akira, and Raymond Y. Chiao. "Observation of Berry's topological phase by use of an optical fiber." Physical review letters 57.8 (1986): 937.

\bibitem{Vedral2003}Carollo, Angelo, et al. "Geometric phase in open systems." Physical review letters 90.16 (2003): 160402.

\bibitem{Sjöqvist2008}Sjöqvist, Erik. "A new phase in quantum computation." Physics 1 (2008): 35.

\bibitem{Sjöqvist1997}Sjöqvist, Erik, and Magnus Hedström. "Noncyclic geometric phase, coherent states, and the time-dependent variational principle: application to coupled electron-nuclear dynamics." Physical Review A 56.5 (1997): 3417.

\bibitem{Jain1998}Jain, Sudhir R., and Arun K. Pati. "Adiabatic geometric phases and response functions." Physical review letters 80.4 (1998): 650.

\bibitem{Pati1999}Pati, Arun Kumar. "Quantum superposition of multiple clones and the novel cloning machine." Physical review letters 83.14 (1999): 2849.

\bibitem{Zanardi1999}Zanardi, Paolo, and Mario Rasetti. "Holonomic quantum computation." Physics Letters A 264.2-3 (1999): 94-99.

\bibitem{Pachos1999}Pachos, Jiannis, Paolo Zanardi, and Mario Rasetti. "Non-Abelian Berry connections for quantum computation." Physical Review A 61.1 (1999): 010305.

\bibitem{Pachos2001}Pachos, Jiannis, and Paolo Zanardi. "Quantum holonomies for quantum computing." International Journal of Modern Physics B 15.09 (2001): 1257-1285.

\bibitem{Aharonov1987}Aharonov, Yakir, and J. Anandan. "Phase change during a cyclic quantum evolution." Physical Review Letters 58.16 (1987): 1593.

\bibitem{Samuel1988}Samuel, Joseph, and Rajendra Bhandari. "General setting for Berry's phase." Physical Review Letters 60.23 (1988): 2339.


\bibitem{Pancharatnam1956}Pancharatnam, Shivaramakrishnan. "Generalized theory of interference, and its applications: Part I. Coherent pencils." Proceedings of the Indian Academy of Sciences-Section A. Vol. 44. No. 5. New Delhi: Springer India, 1956.

\bibitem{Mukunda1993}Mukunda, N., and R. Simon. "Quantum kinematic approach to the geometric phase. I. General formalism." Annals of Physics 228.2 (1993): 205-268.

\bibitem{Mostafazadeh1999}Mostafazadeh, Ali. "Noncyclic geometric phase and its non-Abelian generalization." Journal of Physics A: Mathematical and General 32.46 (1999): 8157.

\bibitem{Sjöqvist2000}Sjöqvist, Erik, et al. "Geometric phases for mixed states in interferometry." Physical Review Letters 85.14 (2000): 2845.


\bibitem{Kimble1998}Kimble, H. Jeff. "Strong interactions of single atoms and photons in cavity QED." Physica Scripta 1998.T76 (1998): 127.

\bibitem{Blais2020}Blais, Alexandre, Steven M. Girvin, and William D. Oliver. "Quantum information processing and quantum optics with circuit quantum electrodynamics." Nature Physics 16.3 (2020): 247-256.

\bibitem{Clarke2008}Clarke, John, and Frank K. Wilhelm. "Superconducting quantum bits." Nature 453.7198 (2008): 1031-1042.

\bibitem{Kjaergaard2019}Krantz, Philip, et al. "A quantum engineer's guide to superconducting qubits." Applied physics reviews 6.2 (2019): 021318.

\bibitem{Krantz22019}Krantz, Philip, et al. "A quantum engineer's guide to superconducting qubits." Applied physics reviews 6.2 (2019): 021318.

\bibitem{Wendin2017}Wendin, Göran. "Quantum information processing with superconducting circuits: a review." Reports on Progress in Physics 80.10 (2017): 106001.

\bibitem{Clerk2020}Clerk, A. A., et al. "Hybrid quantum systems with circuit quantum electrodynamics." Nature Physics 16.3 (2020): 257-267.

\bibitem{Kubo2010}Kubo, Y., et al. "Strong coupling of a spin ensemble to a superconducting resonator." Physical review letters 105.14 (2010): 140502.

\bibitem{Aspelmeyer2014}Aspelmeyer, Markus, Tobias J. Kippenberg, and Florian Marquardt. "Cavity optomechanics." Reviews of Modern Physics 86.4 (2014): 1391.

\bibitem{Burkard2020}Burkard, Guido, et al. "Superconductor–semiconductor hybrid-circuit quantum electrodynamics." Nature Reviews Physics 2.3 (2020): 129-140.

\bibitem{Lachance2019}Lachance-Quirion, Dany, et al. "Hybrid quantum systems based on magnonics." Applied Physics Express 12.7 (2019): 070101.

\bibitem{Wong2005}Wong, Hon Man, Kai Ming Cheng, and M-C. Chu. "Quantum geometric phase between orthogonal states." Physical review letters 94.7 (2005): 070406.

\bibitem{Uhlmann1986}Uhlmann, Armin. "Parallel transport and “quantum holonomy” along density operators." Reports on Mathematical Physics 24.2 (1986): 229-240.



\bibitem{Nigg2012}Nigg, Simon E., et al. "Black-box superconducting circuit quantization." Physical Review Letters 108.24 (2012): 240502.

\bibitem{Nakamura1999}Nakamura, Yasunobu, Yu A. Pashkin, and J. S. Tsai. "Coherent control of macroscopic quantum states in a single-Cooper-pair box." nature 398.6730 (1999): 786-788.

\bibitem{Paik2011}Paik, Hanhee, et al. "Observation of high coherence in Josephson junction qubits measured in a three-dimensional circuit QED architecture." Physical Review Letters 107.24 (2011): 240501.

\bibitem{Rigetti2012}Rigetti, Chad, et al. "Superconducting qubit in a waveguide cavity with a coherence time approaching 0.1 ms." Physical Review B 86.10 (2012): 100506.

\bibitem{Barends}Barends, Rami, et al. "Superconducting quantum circuits at the surface code threshold for fault tolerance." Nature 508.7497 (2014): 500-503.

\bibitem{Chang2013}Chang, Josephine B., et al. "Improved superconducting qubit coherence using titanium nitride." Applied Physics Letters 103.1 (2013): 012602.

\bibitem{Josephson1962}Josephson, B. D., 1962, Physics Letters 1(7), 251


\bibitem{Schoelkopf2003}Schoelkopf, R. J., et al. "Qubits as spectrometers of quantum noise." Quantum noise in mesoscopic physics (2003): 175-203.

\bibitem{Rempe1987}Rempe, Gerhard, Herbert Walther, and Norbert Klein. "Observation of quantum collapse and revival in a one-atom maser." Physical review letters 58.4 (1987): 353.

\bibitem{Koch2007}Koch, Jens, et al. "Charge-insensitive qubit design derived from the Cooper pair box." Physical Review A 76.4 (2007): 042319.

\bibitem{Jones2000} Jones, Jonathan A., et al. "Geometric quantum computation using nuclear magnetic resonance." Nature 403.6772 (2000): 869-871.

\bibitem{Krantz2019} Krantz, Philip, et al. "A quantum engineer's guide to superconducting qubits." Applied physics reviews 6.2 (2019): 021318.

\bibitem{Xiang2013}Xiang, Ze-Liang, et al. "Hybrid quantum circuits: Superconducting circuits interacting with other quantum systems." Reviews of Modern Physics 85.2 (2013): 623. 

\bibitem{Lombardo2006} Lombardo, Fernando C., and Paula I. Villar. "Geometric phases in open systems: A model to study how they are corrected by decoherence." Physical Review A 74.4 (2006): 042311.

\bibitem{Tong2004}Tong, D. M., et al. "Kinematic approach to the mixed state geometric phase in nonunitary evolution." Physical review letters 93.8 (2004): 080405. 

\bibitem{Ripoll2022}Ripoll, Juan José García. Quantum Information and Quantum Optics with Superconducting Circuits. Cambridge University Press, 2022.

\bibitem{Jaynes1963} Jaynes, Edwin T., and Frederick W. Cummings. "Comparison of quantum and semiclassical radiation theories with application to the beam maser." Proceedings of the IEEE 51.1 (1963): 89-109.

\bibitem{Griffiths2005}Griffiths, David J., and Darrell Schroeter. Instructor's Solutions Manual: Introduction to Quantum Mechanics. Pearson education, 2005.

\bibitem{Tang1995} Tang, Zhong. "Approach to a generalized Jaynes-Cummings model and the geometric phase." Physical Review A 52.5 (1995): 3448.

\bibitem{Yu2008} Yu, Ge, et al. "Geometric phase in a two energy level Jaynes-Cummings model with imaginary photon process." International Journal of Theoretical Physics 47 (2008): 2279-2284.

\bibitem{Breuer2002} Breuer, Heinz-Peter, and Francesco Petruccione. The theory of open quantum systems. Oxford University Press on Demand, 2002.

\bibliographystyle{ieeetr}
%\bibliography{biblio}

\end{thebibliography}
