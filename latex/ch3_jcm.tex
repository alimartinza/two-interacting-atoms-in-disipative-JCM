\chapter{Modelo de Jaynes-Cummings}
\label{ch:jcm}

%CAMBIAR ESTO PARA PERSONALIZARLO A MI GUSTO
\pagestyle{fancy}
\fancyhf{}
\fancyhead[LE]{\nouppercase{\rightmark\hfill}}
\fancyhead[RO]{\nouppercase{\leftmark\hfill}}
\fancyfoot[LE,RO]{\hfill\thepage\hfill}

En este capitulo analizaremos en profundidad la dinamica y los aspectos teoricos mas importantes 
del modelo de Jaynes-Cummings, abordando el problema tanto desde un lado teorico, como desde
el lado computacional, necesario para resolver la dinamica en sistemas abiertos.
Primero se trabajara en el modelo de un atomo en una cavidad, se analizaran los casos importantes,
y se explicara la dinamica del problema. Esto es importante para comprender conceptualmente como
interactuan fundamentalmente la materia y la luz, y nos sirve para conseguir buena intuicion del
problema de dos atomos. Tambien se ver\'a la influencia del entorno sobre la cavidad, permitiendo
perdida (o absorcion) de fotones, y tambien el bombeo coherente que puede excitar espontaneamente
al atomo. \newline
Luego, se analizara el problema para dos atomos, primero en el caso que estos no interactuan
directamente entre si, sino que lo hace indirectamente a travez de la cavidad. La comparativa entre
esta situacion y la mas comun, donde los atomos interactuan mediante sus espines o sus momentos
dipolares, es muy rica porque nos permite discernir con claridad cual es el efecto de la cavidad
y cual de la interaccion entre los atomos a la hora de entrelazarse e intercambiar energia.
El problema de dos atomos tiene una peculiaridad al elegir las condiciones iniciales, ya que la
dinamica depende de esta eleccion, y hay muchas diferentes configuraciones interesantes, por un lado
por la gran dimension del espacio, y por otro lado, esta la posibilidad de jugar con las simetrias.
Surge asi la pregunta de si es importante, o si tiene sentido, teniendo dos atomos indistinguibles
en una cavidad, que la condicion inicial sea asimetrica ante intercambio. 

\section{JCM de un atomo}
Comencemos entonces por el paradigmatico modelo de 1 atomo. 